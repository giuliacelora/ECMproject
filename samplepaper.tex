% This is samplepaper.tex, a sample chapter demonstrating the
% LLNCS macro package for Springer Computer Science proceedings;
% Version 2.20 of 2017/10/04
%
\documentclass[runningheads]{llncs}
%
usepackage{amsmath}
\usepackage{booktabs} % For pretty tables
\usepackage{caption} % For caption spacing
\usepackage{graphicx}
\usepackage{pgfplots}
\usepackage[all]{nowidow}
\usepackage[utf8]{inputenc}
\usepackage{tikz}
\usetikzlibrary{er,positioning,bayesnet}
\usepackage{multicol}
\usepackage{algpseudocode,algorithm,algorithmicx}
\usepackage{minted}
\usepackage{hyperref}
\usepackage[inline]{enumitem} % Horizontal lists
% Used for displaying a sample figure. If possible, figure files should
% be included in EPS format.
%
% If you use the hyperref package, please uncomment the following line
% to display URLs in blue roman font according to Springer's eBook style:
% \renewcommand\UrlFont{\color{blue}\rmfamily}

\newcommand{\card}[1]{\left\vert{#1}\right\vert}
\newcommand*\Let[2]{\State #1 $\gets$ #2}

\pgfplotsset{compat=1.14}

\renewcommand{\topfraction}{0.85}
\renewcommand{\bottomfraction}{0.85}
\renewcommand{\textfraction}{0.15}
\renewcommand{\floatpagefraction}{0.8}
\renewcommand{\textfraction}{0.1}
\setlength{\floatsep}{3pt plus 1pt minus 1pt}
\setlength{\textfloatsep}{3pt plus 1pt minus 1pt}
\setlength{\intextsep}{3pt plus 1pt minus 1pt}
\setlength{\abovecaptionskip}{2pt plus 1pt minus 1pt}


\begin{document}
%
\title{A thermodynamically consistent multi-phase model of Extracellular Matrix}
%
\titlerunning{Mechanical Properties of ECM}
% If the paper title is too long for the running head, you can set
% an abbreviated paper title here
%
\author{Giulia Laura Celora}
%
%\authorrunning{F. Author et al.}
% First names are abbreviated in the running head.
% If there are more than two authors, 'et al.' is used.
%
\institute{Mathematical Institute University of Oxford}
%
\maketitle              % typeset the header of the contribution
%
\begin{abstract}

\end{abstract}
%
%
%
\section{Introduction}
There are now several studies supporting the central role of mechanical stimulus in tissue morphogenesis and homeostasis, alongside with biochemical signalling. \texit{In vitro} studies have shown that ECM rigidity and shear stresses can alone promote the transition to malignant phenotype of normal cells and consequently the growth of malignant masses. 
\bibliographystyle{splncs04}
\bibliography{biblio}
%
\end{document}
