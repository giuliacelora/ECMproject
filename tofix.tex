\section{Model B: Separating the Volumetric Deformation.}
\label{modelB}
The derivation of the governing equation for the second model proposed follow the same steps as model A, with only few changes. The conservation laws (Section \ref{conslaw}) are still valid as they do not depend on the specific kinetics model chosen. As discussed in Section \ref{kin}, we use multiplicative decomposition to isolate the different contribution to the strain:
\begin{equation}
\F= \bar{\F} \F_{vol}= J^{1/3} \bar{\F}_e \bar{\F}_v,\label{mol2}
\end{equation}
where we have used the fact that $\F_{vol}=J^{1/3}\mathbb{I}$, with $J=\det \F$. Where both $\bar{\F}_e$ and $\bar{\F}_v$ preserve the ECM volume. Consequently, analogously to Equation~(\ref{Jv}), we have that:
\begin{equation}
\det \bar{\F}_v = 1.
\end{equation}
\begin{figure}
	\Large
	\def\svgwidth{1\linewidth}
	\input{latex/images/modelB2.pdf_tex}
	\caption{Multiplicative decomposition of Model B}
\end{figure}

Besides the viscous velocity gradient $\bar{\LL}_v=\dot{\bar{\F}}_v\bar{\F}_v^{-1}$, we also define the total velocity gradient $\LL=\dot{\F}\F^{-1}$. These are the two variables we will be using to describe the kinetics of the solid phase deformation. 

Focusing on the reversible exchange of energy and mass between the system and the environment, we have that Equation~(\ref{M}) and~(\ref{W}) still hold, so that the energy imbalance inequality can still be written as Equation~(\ref{temp2}). Note however that the constitutive function of the free energy is now of the form:
\begin{equation}
\psi=\psi(\bar{\F},\bar{\F}_e, J , C_s,C_i,\mathbf{H}),\label{aptemp1}
\end{equation}
where $J$ needs now to be introduced to the list of independent variables. 

If we now take the time derivative of $J$, $\bar{\F}$ and $\bar{\F}_e$, we have that:

\begin{gather}
\dot{J} = J (\mathbb{I}:\LL),\\
\dot{\bar{\F}} = J^{-1/3} \LL \F - \frac{1}{3} J^{-1/3} (\mathbb{I}:\LL) \F \\
\dot{\bar{\F}}_e = \text{DEV}[\LL] \bar{\F}_e -\bar{\F}_e \bar{\LL}_v,\label{aptemp2}
\end{gather}
so that the dynamics of the system can be fully described in terms of $\LL$ and $\bar{\LL}_v$. If we combine Equations~(\ref{aptemp1})-(\ref{aptemp2}) with the energy inequality~(\ref{temp2}), we obtain:
\begin{equation}
\begin{aligned}
\color{blue} \left(\frac{\partial \psi}{\partial C_s}-\mu_s+p v\right)\color{black}\dot{C}_s+ \color{blue}\sum_i\left(\frac{\partial \psi}{\partial C_i} + e\Phi z_i-\mu_i\right)\color{black} \dot{C}_i
+\color{blue}\left(\frac{\partial \psi}{\partial \mathbf{H}}-\mathbf{E}\right) \color{black} \cdot \dot{\mathbf{H}}\\
+\color{blue} \left(\text{DEV}\left[J^{-1/3}\frac{\partial \psi}{\partial \bar{\F}}\F^T + \frac{\partial \psi}{\partial \bar{\F}_e}\bar{\F}_e^{T}\right]- \mathbb{S}\F^T +J\left(\frac{\partial \psi}{\partial J} - p\right)\mathbb{I}\right)\color{black}:\LL\\
+ \sum_m \nabla_0 \,\mu_m \cdot \mathbf{J}_m - \text{DEV}[\mathbb{\bar{M}}_e]:\mathbb{\bar{L}}_v\leq 0 . \label{ineq2}
\end{aligned}
\end{equation}

Moving on to the form of the free energy $\psi$ (see Section \ref{freeenergy}), Equations~(\ref{psi1})-(\ref{psi4}) still holds. On the other hand, the last term, i.e. the strain energy $\psi_5$, needs to be update so to have a term explicitly related to the volumetric deformation, which we assume to be perfectly elastic. From a rheological point of view, as illustrated in Figure \ref{fig1B}, this is equivalent to add a spring in series to the \textquotedblleft circuit'' in Figure~\ref{fig1A}.

\begin{figure}
	\begin{subfigure}{0.32\textwidth}
		\centering
		\large
		\def\svgwidth{0.9\linewidth}
		\input{latex/images/modelB1.pdf_tex}
		\caption{}
		\label{fig1B}
	\end{subfigure}
	\hspace{20mm}
	\begin{subtable}{0.375\textwidth}
		\hspace{-15mm}
		\begin{tabular}{|c | c | c|}	
			\hline
			\multirow{2}{*}{\textbf{ Element } }& \textbf{ Constitutive } & \multirow{2}{*}{\textbf{ Deformation }} \\
			& \textbf{Properties} &\\
			\hline	
			\multirow{2}{*}{ spring $vol$} & Isotropic  & 	\multirow{2}{*}{  volumetric }\\
			&Neo-Hookean spring&\\
			\hline
			\multirow{2}{*}{ spring 1 } & Isotropic  & 	\multirow{2}{*}{  deviatoric }\\
			&Neo-Hookean spring&\\
			\hline
			\multirow{2}{*}{ spring 2 } & Isotropic  & \multirow{2}{*}{  deviatoric }\\
			&Neo-Hookean spring &\\ 
			\hline
			\multirow{2}{*}{dashpot}  & Isotropic  & 	\multirow{2}{*}{deviatoric}\\
			& Linear dashpot & \\
			\hline
		\end{tabular}
		\caption{}
		\label{tabB}
	\end{subtable}
	\caption{(a) Schematic representation of the non-linear rheological model for ECM; (b) Table summarizing the major properties of the model components.}
\end{figure}

As in Section \ref{freeenergy}, we can decompose strain energy as the sum of contributions from each element in the rheological model:

\begin{equation}
\psi_5 = \psi_{5.1}(\bar{\F}) + \psi_{5.2}(\bar{\F}_e) + \psi_{vol}(J).
\end{equation}

which based on the constitutive properties list in Table~(\ref{tabB}) have the form:

\begin{eqnarray}
\psi_{5.1}(\bar{\F}) = \frac{G^B_1}{2} \left(\bar{\F}:\bar{\F} - 3\right),\\
\psi_{5.2}(\bar{\F}_e) = \frac{G^B_2}{2} \left(\bar{\F}_e:\bar{\F}_e - 3 \right)\\
\psi_{vol}(J) = \frac{G_{vol}}{2}\left[3(J^{2/3}-1) -\ln J^2\right].
\end{eqnarray}

Following the same argument as in Appendix~(\ref{stateequation}), we have that the term highlighted in blue in Equation~(\ref{ineq2}) must be identically zero. This leads to following set of governing equations:
\begin{gather}
\begin{aligned}
\mu_s = p v + \mu_s^0 + kT&\left[\ln \frac{C_s v_s}{1+C_s v_s} + \frac{1}{1+C_sv_s}\right.\\
&\left.\ \ \ \ \ \ +\frac{\chi}{(1+C_s v_s)^2}-\sum_i \frac{C_i}{C_s}\right], 
\end{aligned}\\[2.5mm]
\mu_i = \mu^0_i + e\Phi z_i + kT \ln \frac{C_i}{C_s},\\
\mathbf{E} = \frac{1}{\epsilon J} \F^T \F\, \mathbf{H}\, \\[3mm]
\begin{aligned}
\mathbb{T}= \left(G_{vol}\dfrac{J^{2/3}-1}{J}-p\right) \mathbb{I} + \frac{G^B_1}{1+C_sv_s}\text{DEV}[\bar{\B}] + \frac{G^B_2}{1+C_sv}\text{DEV}[\bar{\B}_e]\\
+ \gamma \left[\frac{1}{2} |\nabla C_s|^2\mathbb{I} - \nabla C_s \otimes \nabla C_s\right]+ \epsilon \left[\frac{1}{2} \,|\nabla \Phi|^2\mathbb{I} -\nabla \Phi \otimes \nabla \Phi\right],\label{sys2B}
\end{aligned}
\end{gather}

When looking at how energy is dissipated, the derivation in Section~(\ref{ent}) and Appendix~(\ref{apenergy}) is still valid, with the only [requirement] of replacing $\LL_v\leftrightarrow \bar{\LL}$ and $\mathbb{B}_e\leftrightarrow\mathbb{\bar{B}}_e$. If we assume that the time of relaxation $\tau_R$ is independent of the model used, this yield to a system of governing equations analogous to the System~(\ref{gov2})- (\ref{Be}):
coupled to the governing equations:
\begin{gather}
\partial_t C_s=\nabla_0 \cdot\left[K C_s \F^{-1}\left(c_s\nabla \mu_s +\sum_i \frac{D_i}{D^0_i} c_i \nabla \mu_i\right)\right],\\
\partial_t C_i= \nabla_0\cdot\left[\frac{D_i}{k_B T}C_i\F^{-1}\nabla \mu_i -\frac{D_i}{D^0_i} \frac{C_i}{C_s} \mathbf{J}_s\right],\\
\dot{\bar{\B}}_e = \bar{\B}_e \bar{\LL}^T +\bar{\LL} \bar{\B}_e -\frac{1}{\tau_R} \bar{\B}_e \text{DEV}\left[\bar{\B}_e\right]\label{BeB}
\end{gather}

To sum up, model A and model B are similar with only exception of the state equation for the Cauchy stress $\mathbb{T}$ and the governing equation of the dashpot.
\section{Simulation Parameter.}
\label{para}
Throughout the work, unless differently specified, we will use the parameters in Table \ref{Tab1} will be considered to be constant so to reflect the condition in the experiment by Netti et~al. \cite{Netti,ecm2}.
\begin{table}[h!]
	\vspace{4mm}
	\centering
	\begin{tabular}{||c c c||}
		\hline\addlinespace[2pt]
		Symbol  & Value& Unit\\
		\hline\addlinespace[5pt]
		$\qquad C_f\qquad$  & $3.947\times 10^{23}$& $\qquad\text{m}^{-3}\quad$\\
		$\qquad v_s\qquad$  & $3\times 10^{-29}$& $\qquad\text{m}^3\quad$ \\
		$\qquad z_f\qquad$ & $-4$& -\\
		$\qquad k_B\qquad$ & $1.38 \times 10^{-23}$& $\qquad\text{J}/\text{K}\quad$\\
		$\qquad T\qquad$ &$295$ &K\\
		$\qquad c^i_0\qquad$ & $9.27\times 10^{25}$& $\qquad\text{m}^{-3}\quad$\\
		\hline
	\end{tabular}
	\vspace{2mm}
	\caption{Parameters adopted in the simulations as estimated in \cite{ecm2} in reference to the experiment by Netti et~al. \cite{Netti}.}
	\label{Tab1}
\end{table}

\section{Free Swelling}
\label{apfree}
In the case of free swelling, due to the symmetry of $\F$, the tensors $\F_e$ and $\bar{F}_e$ is of the similar form, $\F_e=\lambda_e \mathbb{I}$. Consequently, based on Equation~(\ref{apBe}) in Appendix \ref{apenergy}, we have that the viscous contribution vanish so that $\B_e=\B= \lambda^2 \mathbb{I}$. Substituting this result and the boundary condition~(\ref{free1})-(\ref{free2}) into Equations (\ref{sys1})-(\ref{sys2}) and (\ref{sys3}), setting to zero all spatial derivative, we obtain:
\begin{gather}
p_A = \frac{G^A_1+G^A_2}{1+C_sv_s}(\lambda^2-1),\label{presA}\\
\Pi^{n}_A = \frac{k_BT}{v_s} \left[\ln \frac{C_s v_s}{1+C_s v_s} + \frac{1}{1+C_sv_s} +\frac{\chi}{(1+C_s v_s)^2}\right],\\
\Pi^{ion}_A = k_B T \sum_i \left(\frac{C_i}{v_sC_s}-c^0\right),\\
0 = \frac{v_s}{k_BT} (p_A+\Pi^{n}_A-\Pi^{ion}_A), \\[2mm]
0 = \pm\frac{e}{k_B T} \phi  + \ln \frac{C_\pm}{C_s v_s c_\pm^0},\qquad i=1,\ldots,N,\\[2.5mm]
Q = e\left(C_+-C_-+z_f C_{f}\right)=0.\label{electron}
\end{gather}
where $\Pi^{n}$ and $\Pi^{ion}_A$ are the osmotic pressures due to the mixing of the polymer network with the solvent and the imbalance of ions inside and outside the ECM. 
Note that at equilibrium the system reaches a balance between the mechanical pressure $p_A$ and the osmotic pressures. Moreover, the electro-neutrality condition is naturally imposed, Equation~(\ref{electron}). 
Note that Equation~(\ref{eqion}) corresponds to the well known Donnan Equilibrium \cite{DROZDOVph}. Equation~(\ref{eqF}) instead implicitly defines the concentration $C_s$ and thus the final swelling volume. As expected, the latter can be controlled by changing the concentration of ions in the bath. We also notice that, free swelling experiment, are not sufficient to differentiate the elastic properties of the two branches as the two behave equivalently.