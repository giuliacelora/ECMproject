\section{Introduction}

There are several studies supporting the central role of mechanical stimuli in tissue morphogenesis and homeostasis \cite{ex1,ex2}. In tissues, cells are mainly surrounded by extracellular matrix (ECM), a soft porous media made up of  networks of polymer chains and proteins. \textit{In vitro} studies have shown that ECM rigidity and shear stresses can alone promote malignant phenotypes in a population of initially normal cells, impact on cell proliferation and differentiation \cite{ex3}. Further experiments on solid tumour development have proven that this is often associated to a stiffening of the tissue compared to the surrounding healthy one \cite{ex4}, which results in the exposure of cells to higher compressive stresses as well as favouring the collapse of blood vessels and impeding the diffusion of substances in the extra-cellular environment ultimately decreasing the efficacy of numerous therapies \cite{ecm2}. Based on such evidence, it is now widely accepted that, unlike originally thought, biological process are not simply regulated by biochemical signals but by the complex interplay of mechanical and chemical stimuli.
 
Given the different physical nature and scale of phenomena involved, coupling micro-environment and cell behaviours is a problem of high complexity. This requires understanding processes occurring at different temporal and spatial scales and how they interplay to determine the macroscopic behaviour of a tissue, whether healthy or damaged. If we can learn to tune its properties, as cells already do, this could led to the development of novel therapies and completely change our approach to  drug design. In order for this to be possible, alongside experiments, it is necessary to develop a theoretical framework able to capture both the biology and physics involved and which is consistent with the known universal laws of Nature \cite{NET}. 

With the development of new experimental techniques such as Atomic Force Microscopy (AFM), the local mechanical properties of a material can be measured with atomic precision \cite{viscoporo}. When tested at this scale, soft tissues and the ECM in particular have been found to be visco-elastic \cite{ex5}. Purely elastic solids only store energy when deformed, viscoelastic material instead exhibit a time-dependent response as part of the energy is dissipated in the deformation process. While a large amount of literature focuses on the elastic properties of ECM, it remains unclear the role of viscosity in determining cell behaviour. However, the recent efforts to develop synthetic ECM, i.e. hydrogels, with tunable viscoelasticity, have now opened new research opportunities \cite{viscocell}. 

Despite the progress in experimental techniques, theoretical studies of viscoelastic soft materials remain limited. While experiments rapidly progress, most of the literature on mathematical modelling for soft matter has completely neglected viscous dissipation \cite{Article1}. Whether this assumption might be valid for certain applications, the empirical studies previously mentioned highlight the need of including this component in the study of living tissues. Our works aims to develop a continuum mathematical model of the extracellular matrix which is consistent with the laws of thermodynamics, which accounts for its poro-visco-elastic properties and the coupling of mechanical, transport and electrical phenomena. At our present knowledge, there is no previous work in the literature capturing all these aspects. In \cite{ecm1,ecm2} Xue et al.~ develop a nonlinear poroelastic theory for ECM, which couples all three physical phenomena but does not include viscous dissipation. In \cite{Jeru}, the authors couple mechano-electrophysiological effects including the viscous dissipation but neglect transport; Caccavo et al.~ \cite{Article1} propose a poro-viscoelastic model for neutral hydrogel, thus excluding electrical effects.  Following these previous work, we will derive our model in the framework of linear non-equilibrium thermodynamics \cite{NET}, multi-phase modelling and Biot's poroelastic theory of continuum \cite{Biot}. 

Despite the large number of studies that have characterised the poro-elastic and visco-elastic properties of ECM independently, little is known about their combined effect. In the literature, two main constitutive models have been presented, but never rigorously compared.  Instead of arbitrarily choosing one of the two, we here rely on both approaches, with the aim of identifying their differences and investigating experimental result which would allow us to experimentally test which one best describes the behaviour of soft tissues. From this point of view, our results are more widely applicable to the study of polyelectrolyte gels, which are largely applied as biomedical devices and as a synthetic equivalent of ECM.

Our work is organized as follows: in Section \ref{secNET} we start with a brief overview of Classical Irreversible Thermodynamics. After presenting the composition of the ECM, Section \ref{modeldev} will be focusing on the derivation of the governing equation for the deformation and swelling of ECM. [... FOLLOWING SECTIONS TO UPDATE AS I WRITE.]