\section{Conclusion and Further Work.}

Our work has been motivated by the recent evidences of the visco-elastic behaviour of soft tissue and in particular the extracellular matrix (ECM). Experimental studies have shown that in tumour the ECM undergoes stiffening, which relates to poor efficacy of chemotherapy and positively correlates with the formation of metastasis. On the other known, little is known on the effect of visco-elasticity on the cell micro-environment and whether this might change between healthy and damaged tissues. However, as synthetic ECM of tunable visco-elasticity can be now produced, there has been recently a growing interested in investigating this aspect of the cell environment.

In this study, we develop a novel thermodynamically consistent model for the extracellular matrix, and more generally applicable to polyelectrolyte gels, which accounts simultaneously for swelling, polarization, transport and viscous deformation phenomena. Despite having being separately analysed, there has been no previous model presented in the literature that couples all of these aspects. In particular, there are only a limited number of studies which incorporate viscous effects, which is the aspect we are more interested in. 

Our model is derived in the framework of Linear non Equilibrium Themodynamics, which have been commonly used in the realm of soft matter to derive models for complex materials where multiple physical process needs to be considered. In Section \ref{modeldev}, we have derived the model based on constitutive laws and universal physical principles, such as conservation of mass and momentum and the second law of thermodynamics. Unlike for poro-elastic model commonly proposed to describe soft tissue, the introduction of viscous (or also plastic) effect requires to specify a decomposition for the deformation tensor $\F$. Despite this approach have been used in the past few decades in different fields of continuum mechanics, there is no systematic or unified rules that help in the choice. 

Starting from two of the most common decomposition in the framework of large deformation, we have investigated how the two affect the material behaviours. As a first analysis, we have been focusing on equilibrium experiment. Even in this simplified situations, we have shown that the two models can differ. In particular, we have identified the set of experimental data we will need to validate the two models propose. The first natural extension to this work is thus perform such experiment and define the model that better describe the data recorded. Once identified the most promising model, more accurate analyses of the dynamics of the system can be performed, with particular focus on how viscosity can affect the diffusion of substances in the ECM. 