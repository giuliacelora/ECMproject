\section{Conclusion and Further Work.}

Our work has been motivated by the recent evidences of the visco-elastic behaviour of soft tissue and in particular of the extracellular matrix (ECM). Experimental studies have shown that in tumour the ECM undergoes stiffening, which relates to poor efficacy of chemotherapy and positively correlates with the formation of metastasis. On the other known, little is known on the effect of visco-elasticity on the cell micro-environment and the consequence this might have on healthy and damaged tissues. However, as synthetic ECM with tunable visco-elasticity can be now produced, there has been a growing interested in investigating also this aspect of the cell environment \cite{stanford}.

This study focuses on the development of a theory for poro-visco-elastic materials, which applies to soft tissue and it is more generally applicable to polyelectrolyte gels. Our model is derived in the framework of Linear non Equilibrium Themodynamics, which have been commonly used in the realm of soft matter to derive models for complex materials where multiple physical process needs to be considered. In Section \ref{modeldev}, we have derived the model based on constitutive laws and universal physical principles, such as conservation of mass and momentum and the second law of thermodynamics. Unlike for poro-elastic model commonly proposed to describe soft tissue, the introduction of viscous (or also plastic) effect requires to specify a decomposition for the deformation tensor $\F$. Despite this approach have been used in the past few decades in different fields of continuum mechanics, there is no systematic or unified rules that help in the choice. 

Starting from two of the most common decomposition in the framework of large deformation, we have investigated how the two reflect on the material behaviours. As a standard in the literature of hydrogels and soft tissues \cite{DROSDOVph,Article1,ecm2}, we test our models in two different settings: free swelling and confined compression test. These are indeed typical experiment conducted to estimate the elastic properties of soft materials. Our analysis shows that while the two models agree in the free-swelling, they result in different equilibrium behaviour in the confined test. Despite predicting similar qualitative behaviours, the models are quantitatively different. As shown in Section~\ref{data} using experimental data from previous literature \cite{Netti}, validation of models on compressing data can be misleading and inaccurate. We does conclude that in order to test our model, both free-swelling and constraint compression data are needed in order to identify the model more suited to describe ECM in the realm of large deformations. Consequently, the first natural extension to this work is performing such experiments and, based on the results, select the most appropriate model for ECM. After that, more accurate analyses on the dynamics of the system can be performed, with particular emphasis on the interplay between viscosity, pressure and diffusion of substances in the ECM. Having a model that well describe the macroscopic behaviour of soft-tissue, we can use numerical \text{in silico} and \text{in vitro} AFM experiments, to downscale the model and understand how measurement at the micro-scale relates to the tissue dynamics. 