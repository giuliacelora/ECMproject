\section{Conclusion and Further Work.}

Our work has been motivated by the recent evidence of the visco-elastic behaviour of soft tissue and in particular of the extracellular matrix (ECM). Experimental studies have shown that in tumours, the ECM undergoes stiffening \cite{ecm2}, which relates to poor efficacy of chemotherapy and positively correlates with the formation of metastasis. On the other hand, little is known on the effect of visco-elasticity on the cell micro-environment and the consequence this might have on healthy and damaged tissues. As synthetic ECM with tunable visco-elasticity are now produced, there has been a growing interest in investigating also this aspect of the cell microenvironment \cite{viscocell}.

We have developed a thermodynamically consistent poro-visco-elastic model for polyelectrolyte gels, and more specifically for the extracellular matrix. To the best of our knowledge, this is the first time visco-elasticity is introduced in the framework of linear non-equilibrium thermodynamics to describe polyelectrolytes, which are usually modelled as purely hyper-elastic and porous materials. Unlike in poro-elastic models, the introduction of viscous (or also plastic) effects requires specification of a multiplicative decomposition of the deformation tensor $\F$, depending on how viscous and elastic components of the deformation couple. 

Starting from two of the most common decompositions in the framework of large deformations, we have investigated how the two reflect on the behaviours of a material. As standard in the literature of hydrogels and soft tissues \cite{Article1,DROZDOVph,ecm2}, we test our models on two different settings: free swelling and confined compression tests. Our analysis shows that while the two models agree in the free-swelling, they differ in the case of compression test. As shown in Section~\ref{data} using experimental data from previous literature \cite{Netti}, the two models have similar qualitative behaviour, but differ by several orders of magnitude in their quantitative predictions. As the compression test alone is not sufficient in model validation and selection, we propose a two step protocol which includes both free-swelling and constraint compression experiments. The first natural extension of this work is thus performing such experiments on ECM samples. Having selected the most appropriate model, the interplay of viscosity, pressure and diffusion phenomena can be investigated by analysing the dynamics of the model and comparing it with temporal data from non-static experiments. In particular numerical simulation of \text{in silico} together with \textit{in vitro} AFM experiments can be used to better understand how measurement at the micro-scale relates to the tissue dynamics so as to get new insights into the micro-environment cell experience. 