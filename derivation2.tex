\subsection{Energy Balance Inequality.}
\label{sec_ine}

As mentioned in Section \ref{secNET}, in a thermodynamically consistent model, the free energy, $\psi$, can not be chosen arbitrarily but needs to satisfy the energy imbalance inequality,~(\ref{energyin}). In this section, we focus on the right-hand side of the inequality which specifies how the system exchanges energy and mass with the environment. Considering a control volume $R$ in the reference configuration $\mathcal{B}_0$, the system exchanges mass due to the diffusion of each mobile species, so that $M(R)$ is given by:
\begin{equation}
M(R)= \sum\limits_{m=s,1,\ldots,N} - \int_{\partial R} \mu_m \,\mathbf{J}_m \cdot \mathbf{n} \label{M},
\end{equation}
where $\mathbf{n}$ is the unit normal vector to the surface $\partial R$ and $\mu_m$ is the chemical potential associated with each species. Widely used in the thermodynamics of mixtures, the chemical potential is a measure of the rate of change in free energy associated with adding one more molecule to a unit volume.

Following \cite{DROZDOVph}, the term $W(R)$, i.e. the rate of work done on the system, is decomposed into two contributions:
\begin{equation}
W(R) = -\int_{\partial R} \Phi\, \dot{\mathbf{H}}\cdot \mathbf{n} +  \int_{\partial R}\mathbb{S}\mathbf{n} \cdot \dot{\mathbf{u}},\label{W}
\end{equation}
which stands for the rate of electrical and mechanical work respectively. Substituting this result back into the formula~(\ref{energyin}) and applying the divergence theorem we obtain the following inequality:
\begin{equation}
\int_R \dot{\psi} - \mathbf{E}\cdot \dot{\mathbf{H}} \, + \, \sum\limits_{i=1}^{N} \left[e \Phi  z_i \dot{C}_i+ \nabla_0 \left(\mu_i \mathbf{J}_i \right)\right] + \nabla_0 (\mu_s \mathbf{J}_s-\mathbb{S}^T\mathbf{\dot{u}}) \leq 0\,. 
\end{equation}

Since the volume $R$ can be chosen arbitrarily, the inequality must hold also locally:
\begin{equation}
\dot{\psi} - \mathbf{E}\cdot \dot{\mathbf{H}} \, + \, \sum\limits_{i=1}^{N} \left[e \Phi  z_i \dot{C}_i+ \nabla_0 \left(\mu_i \mathbf{J}_i \right)\right] + \nabla_0 (\mu_s \mathbf{J}_s -\mathbb{S}^T\mathbf{\dot{u}}) \leq 0. 
\end{equation}
Further accounting for Equations~(\ref{consmass})-(\ref{consmom}), we obtain that:
\begin{equation}
\begin{aligned}
\dot{\psi} - \mathbf{E}\cdot \dot{\mathbf{H}} \, + \, \sum\limits_{i=1}^{N} \left[e \Phi  z_i - \mu_i\right] \dot{C}_i - \mu_s \,\dot{C}_s -\mathbb{S}:\dot{\F}+ \sum\limits_{m} \nabla_0 \, \mu_m \cdot \mathbf{J}_m \leq 0.
\end{aligned}
\label{temp2} 
\end{equation}

As exhaustively discussed in previous studies \cite{Plasto,GURTIN}, the energy inequality imposes restrictions on the constitutive equation of the free energy $\psi$. Adapting their results to our specific problem, we have that:
\begin{equation}
\psi = \psi (\F,\F_e, C_s, C_i, \mathbf{H}), \label{temp1}
\end{equation}
which precludes any explicit dependency of $\psi$ on the chemical potentials $\mu_m$ or the viscous deformation gradient $\F_v$. By differentiating the incompressibility condition~(\ref{inc}) and~(\ref{Jv}), we obtain:

\begin{gather}
v_s\dot{C_s} - J \F^{-T}:\dot{\F} =0, \label{temp3}\\
\mathbb{I}:\LL_v=0. \label{temp4}
\end{gather}

If we now substitute~(\ref{temp1}) and~(\ref{lv}) into~(\ref{temp2}), and include the constraint~(\ref{temp3})-(\ref{temp4}) using as Lagrange multipliers $p$ and $p_v$ respectively, we are left with the augmented form of the energy imbalance inequality:
\begin{equation}
\begin{aligned}
 \color{blue}{\left(\frac{\partial \psi}{\partial C_s}-\mu_s+p v_s\right)}\color{black}\dot{C}_s+\color{blue}\left(\frac{\partial \psi}{\partial \mathbf{H}}-\mathbf{E}\right) \cdot \color{black}\dot{\mathbf{H}}+ \sum_i\color{blue}\left(\frac{\partial \psi}{\partial C_i} + e\Phi z_i-\mu_i\right) \color{black}\dot{C}_i \\
+ \color{blue} \left(\frac{\partial \psi}{\partial \F} + \frac{\partial \psi}{\partial\F_e}\F_v^{-1}- \mathbb{S} - p J \F^{-T}\right): \color{black}\dot{\F}+ \sum_m \nabla_0 \,\mu_m \cdot \mathbf{J}_m \\
- \underbrace{\left(\F_e^T\frac{\partial \psi}{\partial \F_e}-p_v\mathbb{I}\right)}_{\text{DEV}[\mathbb{M}_e]}:\mathbb{L}_v\leq 0 , \label{ineq}
\end{aligned}
\end{equation}
where $\mathbb{M}_e$ is the effective Mandel stress. 
\subsection{Construction of the Free Energy.}
\label{freeenergy}
Having the general form of $\psi$, Equation~(\ref{temp1}), it remains to construct its precise form. Following a standard approach in $\psi$-depending modeling, we assume that the total free energy can be additively decomposed with each physical mechanism contributing independently. We here consider five distinct contributions:

\begin{enumerate}
	{\indentitem\item[\textbullet] the energy of the electric field $\psi_1$;}
	{\indentitem \item[\textbullet] the energy of solvent and solutes' molecules not interacting with the solid phase $\psi_2$;}
	{\indentitem\item[\textbullet] the energy of mixing the solid phase with the solution, $\psi_3$;}
	{\indentitem\item[\textbullet] the energy of mixing the solvent with the solutes in solution, $\psi_4$;}
	{\indentitem\item[\textbullet] the energy of the solid phase not interacting with the solution, $\psi_5$.}
\end{enumerate}

Assuming the solid phase to be an ideal and linear dielectric material, with constant permittivity $\epsilon$, the free energy of polarization reads \cite{DROZDOV+,Reviewpolyel}:
\begin{gather}
\psi_1 = \frac{1}{2\epsilon J} \mathbf{H}\F^T \cdot \F \mathbf{H}.\label{psi1}
\end{gather}

The specific energy density $\psi_2$ has the standard form:
\begin{equation}
\psi_2 = \sum\limits_{m} \mu^0_m C_m,
\end{equation} 
where $\mu^0_m$ denotes the chemical potential of non interacting solvent and ions molecules. According to Flory-Huggins theory \cite{flory,hug} of mixtures, the mixing energy is given by:
\begin{equation}
\psi_3 = \frac{k_B T J}{v_s} \left(\phi_f \ln \phi_f + \chi \phi_f \phi_n\right),\label{mix}
\end{equation}
where $k_B$ is the Boltzmann's constant, $T$ is the temperature and $\chi$ is the Flory-Huggins parameter, which is a measure of the enthalpy of mixing. Note that Xue et al.\cite{ecm1,ecm2} use a different approach. In these studies, the authors assume only the mixing of GAGs with solvent, while neglecting the collagen. Since we are considering GAGs and the collagen network as as a unique solid phase and we could not find any evidence that collagen does not mix with water, we have chosen the more general form~(\ref{mix}).

As the interstitial fluid is well approximated by a dilute solution, the contribution $\psi_4$ reads \cite{Reviewpolyel,ecm1,ecm2}:

\begin{equation}
\psi_4 = k_B T \sum\limits_{i=1}^{N} C_i \left(\ln \frac{C_i}{ C_s}-1\right).\label{psi4}
\end{equation}

Finally, we need to specify the strain energy $\psi_5$. As mentioned in the introduction, the Standard Linear Solid (SLS), see Figure \ref{SLS}, is commonly used to describe soft material in the regime of small deformations. However, when account for large deformation, as in the case of swelling, soft material exhibit a non-linear behaviour \cite{floryprinciples}. For this reason we consider the model in Figure \ref{fig1A}, which is a generalization of the 1D SLS to 3D problems with non-linear elastic response. 

\begin{figure}
	\begin{subfigure}{0.32\textwidth}
		\centering
		\large
		\def\svgwidth{0.9\linewidth}
		\input{latex/images/modelA1.pdf_tex}
		\caption{}
		\label{fig1A}
	\end{subfigure}
	\hspace{20mm}
	\begin{subtable}{0.375\textwidth}
		\hspace{-15mm}
		\begin{tabular}{|c | c | c|}	
			\hline
			\multirow{2}{*}{\textbf{ Element } }& \textbf{ Constitutive } & \multirow{2}{*}{\textbf{ Deformation }} \\
			& \textbf{Properties} &\\
			\hline	
			\multirow{2}{*}{ spring 1 } & Isotropic  & volumetric\\
			&Neo-Hookean spring& + deviatoric\\
			\hline
			\multirow{2}{*}{ spring 2 } & Isotropic  & volumetric\\
			&Neo-Hookean spring &+ deviatoric\\ 
			\hline
			\multirow{2}{*}{dashpot}  & Isotropic  & 	\multirow{2}{*}{deviatoric}\\
			& Linear dashpot & \\
			\hline
		\end{tabular}
		\caption{}
	\end{subtable}
	\caption{(a) Schematic representation of the non-linear rheological model for ECM; (b) Table summarizing the major properties of the model components.}
\end{figure}

The strain energy can thus be decomposed into the sum of the contributions from spring $1$ and spring $2$:

\begin{equation}
\psi_5 = \psi_{5.1}(\F) + \psi_{5.2}(\F_e).
\end{equation}

As in \cite{ecm2}, we consider the spring to be isotropic and hyper-elastic (Neo-Hookean) which are characterised by the following form of the free-energy:

\begin{eqnarray}
\psi_{5.1}(\F) = \frac{G^A_1}{2} \left(\F:\F - 3 -2 \ln J\right),\\
\psi_{5.2}(\F_e) = \frac{G^A_2}{2} \left(\F_e:\F_e - 3 -2 \ln J_{e}\right),\label{hyp}
\end{eqnarray}
where $G_{1,2}$ stands for the shear modulus associated with each spring, $J_e= \det \F_e$. As derived in \cite{floryprinciples}, the hyper-elastic model~(\ref{hyp}) can be derived in the framework of statistical mechanics based on the microscopic properties of a polymer network. This assumes the network to consist of Gaussian chains and affine deformation. Other thermodynamically consistent forms of the stretching energy have been proposed in the literature \cite{BERGSTROM1998931,boyce2,doibook} using different network models.

\subsection{Entropy Production $\sigma$.}
\label{ent}

Having specified how the system interacts with its environment, we can now discuss how it dissipates energy. As mentioned in Section~(\ref{kin}), there are two contributions: transport (diffusion of solvent and solutes) and viscosity. The thermodynamic fluxes \footnote{See Section \ref{secNET}} associated with these two phenomena are $\mathbf{J}_m$, $m=s,1,\ldots,N$, and $\LL_v$. Consequently, using Equation~(\ref{2law}), we obtain:

\begin{equation}
\sigma = \sum_m \zeta_m \cdot \mathbf{J}_m + \zeta_v : \LL_v,
\label{dis}
\end{equation}
where $\zeta$s represent the thermodynamic forces associated with each flux. On the other hand, the rates $\Lambda =\left\{\dot{C}_s,\dot{C}_i,\mathbf{\dot{H}},\dot{\F}\right\}$ describe the evolution of reversible process\footnote{energy and entropy in reversible processes are state functions; consequently the state of the system does only depend on the value of the thermodynamic variables and not the rate at which they change.} and thus do not belong to the set of constitutive physical variables\footnote{see Appendix \ref{stateequation}}. As discussed in detail in Appendix \ref{stateequation}, this implies that energy imbalance inequality~(\ref{ineq}) is linear in the variables $\Lambda$, so that the terms highlighted in blue in Equation~(\ref{ineq}) must be identically zero. This leads to the hold system of state equations (see Appendix \ref{stateequation}):
\begin{gather}
\begin{aligned}
\mu_s = p v_s + \mu_s^0 + k_BT&\left[\ln \frac{C_s v_s}{1+C_s v_s} + \frac{1}{1+C_sv_s}\right.\\
&\left.\ \ \ \ \ \ +\frac{\chi}{(1+C_s v_s)^2}-\sum_i \frac{C_i}{C_s}\right], 
\end{aligned}\label{gov1}\\[2.5mm]
\mu_i = \mu^0_i + e\Phi z_i + kT \ln \frac{C_i}{C_s},\label{mu}\\
-\epsilon J \nabla^2 \Phi = Q\, ,\label{sys2}
\end{gather}
\begin{gather}
\begin{aligned}
\mathbb{T}= -p \mathbb{I}+ \frac{G^A_1}{1+C_sv_s}\left(\mathbb{B}-\mathbb{I}\right) + \frac{G^A_2}{1+C_sv_s}\left(\mathbb{B}_e-\mathbb{I}\right) \\
+ \underbrace{\epsilon \left[\frac{1}{2} \,|\nabla \Phi|^2\mathbb{I} -\nabla \Phi \otimes \nabla \Phi\right]}_{\mathbb{T}^{Max}},
\end{aligned}
\label{sys3}
\end{gather}
where $\mathbb{T}^{max}$ is the component of the stress tensor due to the presence of the electric field. As discussed in Section \ref{secNET}, in the framework of linear non-equilibrium thermodynamics, when considering isothermal transformations, the second law of thermodynamics can be rewritten as Equations~(\ref{eqCIT}). Using the same argument as in Section \ref{sec_ine}, we can rewrite Equations~(\ref{eqCIT}) in differential form, and substituting Equations~(\ref{gov1})-(\ref{sys3}), we obtain:
\begin{equation}
\begin{aligned}
\sigma = -  \sum_m T^{-1} \nabla_0 \,\mu_m \cdot \mathbf{J}_m + T^{-1}\mathbb{M}_e:\mathbb{L}_v.\label{EQen}
\end{aligned} 
\end{equation}

Equating Equation~(\ref{EQen}) and~(\ref{dis}), it is natural to identify that the thermodynamics forces as:
\begin{gather}
\zeta_m = -\frac{1}{T} \nabla_0 \,\mu_m, \label{vflow1}\\
\zeta_v = \frac{1}{T} \text{DEV}[\mathbb{M}_e] = \frac{G^A_2}{T} \text{DEV}[\mathbb{C}_e].
\end{gather}
Assuming to be in a regime of linear non-equilibrium thermodynamics, we can use the identity~(\ref{lin}) to linearly couple fluxes and forces. However, considering the symmetry constraint imposed by \textit{Curie's law}\footnote{Macroscopic causes can not have more elements of symmetry than the effect they cause \cite{CIT}} , there can be no coupling between fluxes and forces of a different tensorial nature. Consequently, we are left with the following force-flux relation:
\begin{gather}
\LL_v = L_{vv} \zeta_v=\frac{L_{vv} G^A_2}{T} \text{DEV}[\mathbb{C}_e],\label{vflow2}\\
\mathbf{J}_m = \sum_{k=s,1,\ldots,N} L_{mk}\zeta_k= -\sum_{k=s,1,\ldots,N} \frac{L_{mk}}{T} \nabla_0 \,\mu_k. \label{dif}
\end{gather}

%Combining Equation~(\ref{vflow1}) and (\ref{vflow2}), and imposing that condition~(\ref{Jv}) is satisfied, we can characterise the viscous flow by the following relation:
%\begin{equation}
%\LL_v = L_{vv}T^{-1}\text{DEV}\left[\F_e^T\frac{\partial \psi}{\partial \F_e}\right] = \eta^{-1}\text{DEV}\left[\F_e^T\frac{\partial \psi}{\partial \F_e}\right] ,
%\end{equation}
%where $\eta$ represent the viscosity of the material and $\text{DEV}\left[\cdot\right] = \cdot-1/3\, \text{tr}(\cdot)$ is the deviatoric component of the tensor in the brackets. 

%The dissipative contribution due to the relative movement of phases has been largely studied in the literature \cite{ecm1,ecm2}. Starting from Equation~(\ref{dif}) and standard arguments we can get to the following definition for the fluxes:
As described in Appendix \ref{apenergy}, starting from Equations~(\ref{vflow1})-(\ref{dif}) and with common consideration from the theory of mixture, we can derive the following system of time dependent equations:
\begin{eqnarray}
\partial_t C_s=\nabla_0 \cdot\left[K \F^{-1}\left(C_s\nabla \mu_s +\sum_i \frac{D_i}{D^0_i} C_i \nabla \mu_i\right)\right],\label{gov2}\\
\partial_t C_i= \nabla_0\cdot\left[\frac{D_i}{k_B T}C_i\F^{-1}\nabla \mu_i -\frac{D_i}{D^0_i} \frac{C_i}{C_s} \mathbf{J}_s\right],\label{gov3} \\
\dot{\mathbb{B}}_e =\LL\mathbb{B}_e + \mathbb{B}_e \LL^T - \frac{1}{\tau_R} \,\mathbb{B}_e\text{DEV}[\mathbb{B}_e].\label{Be}
\end{eqnarray}
where the parameters (see the Glossary) are macroscopic phenomenological coefficients. To sum up, Table~(\ref{summary}) lists the variables and corresponding governing equation for model A as well as model B, which we recall has been derived in Appendix \ref{modelB}. In order to have a close solution, the proper boundary and initial conditions needs also to be assigned but these will be specified depending on the specific problem considered.

\vspace{3mm}
\begin{table}
	\centering
	\begin{tabular}{c|c|c|c}
		\hline\addlinespace[2pt]
		Variable &  \hspace{1pt} Model A \hspace{1pt} & \hspace{1pt} Model B\hspace{1pt} & \hspace{4pt}Equation Type\hspace{4pt}\\
		\hline
		\hline\addlinespace[2.5pt]
		Solvent Concentration &  \multicolumn{2}{c|}{$C_s$~(\ref{long})}& Governing \\[2.5pt]
		Chemical Potential of & \multicolumn{2}{c|}{ $\mu_s$~(\ref{gov1})}& State \\
		the solvent &\multicolumn{2}{c|}{}&\\[2pt]
		Ionic Concentrations & \multicolumn{2}{c|}{$C_i$~(\ref{long2})}& Governing \\[3pt]
		Chemical Potential of & \multicolumn{2}{c|}{$\mu_i$~(\ref{mu}) }& State \\
		the ions &\multicolumn{2}{c|}{ }&\\[2pt]
		\hline\addlinespace[2pt]
		Local Volumetric Change & \multicolumn{2}{c|}{$J$~(\ref{inc})}& State \\[2.5pt]
		Cauchy Stress Tensor & 	\multicolumn{2}{c|}{$\mathbb{T}$~(\ref{consmom})}& State \\[2.5pt]
		Deformation Gradient & $\F$~(\ref{dec1})& $\F$~(\ref{dec2})& State \\[2.5pt]
		Pore Pressure & $p$~(\ref{sys3}) & $p$~\ref{sys2B})& State \\[2.5pt]
		Left-Cauchy Tensor& \multirow{2}{*}{$\mathbb{B}_e$~(\ref{Be})} & \multirow{2}{*}{$\mathbb{\bar{B}}_e$~(\ref{BeB})}&\multirow{2}{*}{Governing}\\
		for spring 2&&&\\
		Viscous Velocity & \multirow{2}{*}{$\LL_v$~(\ref{Lv1})}&  \multirow{2}{*}{$\bar{\LL}_v$~(\ref{Lv2})}&\multirow{2}{*}{Governing }\\
		Gradient Tensor&&&\\
		\hline\addlinespace[2pt]
		Electric Field & \multicolumn{2}{c|}{$\Phi$~(\ref{sys2})}& State \\
		Local charge & \multicolumn{2}{c|}{$Q$~(\ref{Q})}& State\\
		\hline 
		\hline
	\end{tabular}
	\vspace{3mm}
	\caption{List of Variables involved in the problem, with reference to the corresponding governing equations. We denote by state equation, the constitutive equations that defines a variable in terms of the state of the system. On the other hand, governing equations explicitly contain the time variable $t$ as they describe the evolution of a non-reversible process.}
	\label{summary}
\end{table}

\section{Summary of the Evolution Equation.}

Having derived the full model for both decomposition of $\F$, we discuss more in details their physical interpretation, with particular interest at the governing equations. We first rewrite the solvent chemical potential, Equation~(\ref{mu}), as:
\begin{gather}
\mu_s = \mu^0_s + k_B T \left(\frac{p v_s}{k_BT} +\Pi_{osm}-\sum_i \frac{C_i}{C_s}\right)\label{mu2},\\
\Pi_{osm}=\ln \frac{C_s v_s}{1+C_s v_s} + \frac{1}{1+C_sv_s}+\frac{\chi}{(1+C_s v_s)^2},
\end{gather}
where $p$ represents the pore pressure and $\Pi_{osm}$ is the osmotic pressure of the solution. If we now substitute into Equation~(\ref{gov2}) the chemical potentials~(\ref{mu2})-(\ref{mu}), this yields to:
\begin{equation}
\begin{aligned}
\partial_t C_s=\nabla_0 \cdot\left\{K\F^{-1}\left[C_s v_s\nabla p +\sum_i \frac{D_i}{D^0_i} C_i e z_i \nabla \Phi.+ k_B T\color{teal} \left(C_s \nabla \Pi_{osm}+  \right.\right.\right.\\
\left.\left.\left.\color{teal} \sum_i\left(1-\frac{D_iC_i}{D^0_iC_s}\right) \nabla C_s- \sum_i\left(1-\frac{D_i}{D^0_i}\right) \nabla C_i \right)\color{black}\right]\right\}.\label{long}
\end{aligned}
\end{equation}

The above equation shows that the solvent transport is driven by the hydrostatic pressure gradient, the osmotic pressure gradient (blue-green term in Equation~(\ref{long})) and the electric potential gradient\footnote{equivalent to the electric field.}. If we consider the limit $v_sC_s\rightarrow\infty$, Equation~(\ref{long}) reduces to Darcy's law for the flow in a porous media. The model for polyelectrolytes proposed by Hong \cite{Reviewpolyel} corresponds instead to the limit $D^0_i\rightarrow\infty$, where mobile species can move freely in pure solution and the friction is due only to relative motion with the solid phase.

Similarly we can rewrite Equation~(\ref{gov3}) as:
\begin{equation}
\scriptsize
\partial_t C_i = \nabla_0 \cdot\left[D_i\F^{-1}\left(\underbrace{\nabla C_i}_{\text{diffusion}} +\underbrace{\frac{eC_iz_i}{k_B T} \nabla \Phi}_{\text{electric}}\right)-\underbrace{\frac{D_i C_i}{C_s}\F^{-1}\nabla C_s}_{\text{osmotic pressure}}-\underbrace{\frac{D_i C_i}{D^0_iC_s}\mathbf{J}_s}_{\text{advection}}\right].\label{long2}
\end{equation}

In the case of ions, the driving forces of transport are the diffusion of ions, the electric field, the osmotic pressure due to the mixing with the solvent and the advection term (due to the relative movement of ions with respect to the solvent). Again in the limit $D^0_i\rightarrow\infty$, we recover the well-known model proposed by Hong \cite{Reviewpolyel} for polyelectrolytes. In the dilute limit, i.e. when the concentration of ions is much smaller than the solvent  concentration $C_i<<C_s$, we can drop both the osmotic and advection term so as to recover the well-known Nernst-Planck equation \cite[see Equation (6.67)]{Reviewpolyel}.

As mentioned in the introduction, we are particularly interested in the visco-elastic contribution to the system kinetics. Even though in the transport Equation~(\ref{long})-(\ref{long2}) there is no direct reference to it, the transport is indirectly coupled to the poro-relaxation through the pressure gradient. Taking the divergence of Equation~(\ref{sys3}) and using the conservation of momentum~(\ref{consmom}), we can identify different components of the pressure gradient:

\begin{equation}
\nabla p= \nabla \cdot \mathbb{T}^{Max} + G_1^A\underbrace{ \nabla \cdot \left[\frac{\B-\mathbb{I}}{1+v_sC_s}\right]}_{\substack{\text{swelling}\\\text{+ deviatoric}\\\text{ deformation}}} + \,\color{red} G_2^A \underbrace{\nabla \cdot\left[\frac{\B_e-\mathbb{I}}{1+v_sC_s}\right]}_{\color{black}\substack{\text{swelling}\\\text{+ deviatoric}\\\text{deformation}\\\text{+ viscous relaxation}}}\color{black}.
\end{equation}

Unlike the standard Neo-Hookean model for polyelectrolytes, we have an additional term highlighted in red, which accounts for the energy stored in the second spring of model A. Its time evolution, as determined by Equation~(\ref{Be}), is driven by both the macroscopic (both volumetric and deviatoric) deformation and the relaxation dynamics which captures the viscous nature of the solid phase:

\begin{equation}
\dot{\mathbb{B}}_e=\underbrace{\LL\mathbb{B}_e + \mathbb{B}_e \LL^T}_{\substack{\text{swelling}\\\text{+ deviatoric}\\\text{deformation}}} - \underbrace{\frac{1}{\tau_R} \,\mathbb{B}_e\text{DEV}[\mathbb{B}_e]}_{\substack{\text{viscous}\\\text{relaxation}}}.\label{Be2}
\end{equation}


Despite the introduction of non-linearities, there is an apparent analogy between the above equation and the ODE describing the standard linear solid, see Equation~(1) in the Introduction. The term $\dot{\epsilon} \leftrightarrow \LL$ is related to the strain experienced by spring 1, while $\epsilon_e  \leftrightarrow \mathbb{B}_e$ is the variable describing the strain on the string in the Maxwell branch. In the limit $\tau_R\rightarrow \infty$, we have that $\mathbb{B}_e\equiv\mathbb{B}$. We thus recover the standard Neo-Hookean hyper-elastic model for hydrogels with shear modulus given by $G^A_1+G^A_2$. Similar result are obtained for model B. The governing equation of model B is equivalent to~(\ref{Be2}), except that the relaxation is now not influenced by volumetric deformations:
\begin{equation}
\dot{\bar{\mathbb{B}}}_e=\underbrace{\bar{\LL}\bar{\mathbb{B}}_e + \bar{\mathbb{B}}_e \bar{\LL}^T}_{\substack{\text{deviatoric}\\\text{deformation}}} - \underbrace{\frac{1}{\tau_R} \,\bar{\mathbb{B}}_e\text{DEV}[\bar{\mathbb{B}}_e]}_{\substack{\text{viscous}\\\text{relaxation}}}.
\end{equation}
Looking instead at the pressure gradient, we see that the volumetric and deviatoric contribution are now decoupled:
\begin{equation}
\begin{aligned}
\nabla p=\nabla \cdot \mathbb{T}^{Max} + G_1^B\underbrace{ \nabla \cdot \left[\frac{\text{DEV}[\bar{\mathbb{B}}]}{1+v_sC_s}\right]}_{\substack{\text{+ deviatoric}\\\text{ deformation}}}+ G_{vol}\underbrace{\nabla \left(\frac{J^{2/3}-1}{J}\right)}_{swelling} \\
+ \,\color{red} G_2^B \underbrace{\nabla \cdot\left[\frac{\text{DEV}[\bar{\mathbb{B}}_e]}{1+v_sC_s}\right]}_{\color{black}\substack{\text{ deviatoric}\\\text{deformation}\\\text{+ viscous relaxation}}}\color{black}.
\end{aligned}
\end{equation}

The two models are equivalent when $\F$ is iso-volumentric, i.e. $\det \F=1$. In the case of purely volumetric deformation, i.e. $\F=\F_{vol}=J^{1/3}\mathbb{I}$, we have instead that the two models are equivalent given that $G_{vol}\equiv G_A +G_B$. However, as we will be discussing in the next section, differences emerge when the deformation gradient has both an hydrostatic and a non-zero deviatoric component. 

