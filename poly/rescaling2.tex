\documentclass[12pt]{extarticle}
\usepackage[utf8]{inputenc}
\usepackage[utf8]{inputenc}
\usepackage{tikz}
\usetikzlibrary{er,positioning,bayesnet}
\usepackage{multicol}
\usepackage{xcolor}
\usepackage{algpseudocode,algorithm,algorithmicx}
\usepackage{hyperref}
\usepackage[inline]{enumitem} % Horizontal lists
\usepackage{cite}
\usepackage{amsmath}
\usepackage{amssymb}
\usepackage{graphicx}
\usepackage{multirow}
\usepackage{booktabs}
\usepackage[labelfont=bf,font=scriptsize]{caption}
\usepackage[font=scriptsize]{subcaption}
\usepackage{mhchem}
\usepackage[makeroom]{cancel}

\newcommand{\indentitem}{\setlength\itemindent{20pt}}
\newcommand{\F}{\ensuremath{\mathbb{F}}}
\newcommand{\B}{\ensuremath{\mathbb{B}}}
\newcommand{\LL}{\ensuremath{\mathbb{L}}}
\renewcommand{\d}{\ensuremath{\text{d}}}
\title{Debye Layer Analysis.}
\author{}

\begin{document}
\section{Non-dimensionalization of the problem.}
\subsection{The Gel.}
We assume the following form for the diffusion coefficients:
\begin{equation}
D_i=D_i^0,\quad k = \frac{D_0(1+v_sC_s)^\beta}{k_B T c_s} \Rightarrow K= \frac{D_0(1+v_sC_s)^\beta}{k_B T}.
\end{equation}
Using the following re-scaling for the variables in the model:
\begin{equation*}
\begin{aligned}
k_BT\hat{\mu} + \mu_0= \mu, \qquad \hat{C}_i = vC_i, \qquad \hat{\Phi} = \frac{\Phi e}{k_B T}, \qquad  G_1\hat{p}= p\\
G_1\hat{\mathbb{T}}=\mathbb{T}, \qquad\hat{\mathbf{x}} L =\mathbf{x}, \qquad \hat{t}\tau=t, \qquad e\hat{Q} =v Q, \qquad \hat{K} = \frac{k_BT}{D_0}K\\
\hat{\mathbf{j}}=\frac{vL}{D_0}\mathbf{j}, \qquad \hat{\LL}= \tau \LL, \qquad \tau=\frac{L^2}{D_0}
\end{aligned}
\end{equation*}
where $L$ is the characteristic size of the gel. We identify the following non-dimensional parameter:
\begin{equation*}
\beta=\frac{L_d}{L}, \qquad L_d= \sqrt{\frac{\epsilon k_B Tv}{L^2e^2}}, \qquad \omega= \frac{L_{int}}{L}, \qquad L_{int}=\sqrt{\frac{\gamma_0}{vk_BT}}, \qquad \mathcal{G}=\frac{vG}{k_BT}.
\end{equation*}
We thus identify three length scales which we consider to be related as follows:
\begin{equation}
L_d<<L, \quad L_d<<L_{int} \Rightarrow \beta<<\omega
\end{equation}

Using the eulerian framework the model can be thus written as:

\begin{gather}
\begin{aligned}
\mu_s = p \mathcal{G}- \omega\nabla^2c_s + \ln c_s+\frac{\chi(1-c_s)+1}{J},
\end{aligned}\\[2.5mm]
\mu_\pm = p \mathcal{G} \pm \Phi + \ln (c_\pm) +\frac{1-\chi c_s}{J},\\ 
-\beta^2 \nabla^2 \Phi = c_+-c_-+z_fc_f\, 
\end{gather}
\begin{gather}
\begin{aligned}
\mathbb{T}= -p \mathbb{I} +  \omega^2\left[\left(\frac{|\nabla c_s|^2}{2}+c_s\nabla^2c_s\right)\mathbb{I} - \nabla c_s \otimes \nabla c_s\right]\\
+ \frac{\beta}{\mathcal{G}} \left[\frac{1}{2} \,|\nabla \Phi|^2\mathbb{I} -\nabla \Phi \otimes \nabla \Phi\right]
\end{aligned}\\
\nabla \cdot \mathbb{T}=\mathbf{0}\\
\partial_t c_s + \nabla \cdot(c_s \mathbf{v}_n)=- \nabla \cdot\mathbf{j}_s,\\
\partial_t c_\pm + \nabla \cdot(c_\pm \mathbf{v}_n)= -\nabla\cdot\mathbf{j}_\pm,\label{flux1.1}\\
\mathbf{j}_s =-c_sK  \left(\nabla \mu_s +\sum_i \frac{c_i}{c_s} \nabla \mu_i\right)\\
\mathbf{j}_\pm= - \frac{D_\pm c_\pm}{D_0}\nabla \mu_\pm + \frac{c_\pm}{c_s}\mathbf{j}_s
\end{gather}
where $c_\ell=C_\ell/J$ is the concentration of the $\ell$-th component in the mixture.
Expanding the solution in terms of $\beta$ at the leading order we recover electro-neutrality:
\begin{equation}
c_- - c_+=z_f c_f. \label{neu1.1}
\end{equation}
Subtracting equations~(\ref{flux1.1}) and using~(\ref{neu1.1}):
\begin{equation}
\partial_t c_f +\nabla \cdot(c_f \mathbf{v}_n)= \nabla\cdot(\mathbf{j}_+-\mathbf{j}_-)
\end{equation}
Using the fact that $c_f=C_f/J$ with $C_f$ being constant and the additional equality:
\begin{equation}
\partial_t J + \mathbf{v}_n \nabla J=J\nabla \cdot\mathbf{v}_n
\end{equation}
we obtain the condition:
\begin{equation}
\frac{\nabla C_f}{J}\mathbf{v}_n= \nabla\cdot(\mathbf{j}_+-\mathbf{j}_-).
\end{equation}
If we further assume that $C_f$ is uniform then:
\begin{equation}
\nabla\cdot(\mathbf{j}_+-\mathbf{j}_-)= 0.
\end{equation}
Consequently at the leading order, the 1D problem in the gel we have:
\begin{gather}
\begin{aligned}
\mu_s = p \mathcal{G}  - \omega^2\partial_{zz} c_s+ \ln c_s+\frac{\chi(1-c_s)+1}{J},
\end{aligned}\\
\mu_\pm = p \mathcal{G} \pm \Phi + \ln (c_\pm) +\frac{1-\chi c_s}{J} ,\\
T_z= -p +\frac{\omega^2}{2\mathcal{G}} \left[c_s\partial_{zz}c_s-\frac{(\partial_z c_s)^2}{2}\right] +\frac{J^2-1}{J},\\
\partial_z T_z=0\\
C_- = C_+ + z_f C_f,\\
\partial_t c_s + \partial_z (c_s v_n)=- \partial_z j_s,\\
\partial_t c_+ +\partial_z (c_+ v_n)= -\partial_zj_+,\\
0= \partial_z(j_+-j_-) \Rightarrow j_+=j_-,\\
j_s =-c_sK  \left(\partial_z\mu_s +\sum_i \frac{c_i}{c_s} \partial_z \mu_i\right)\\
j_+= - \frac{D_+ c_+}{D_0}\partial_z \mu_+ + \frac{c_+}{c_s}j_s
\end{gather}

\subsection{The bath.}
We consider the bath to occupy a domain $\Omega(t)=(h(t),\ell(t))$; therefore the following system of equations holds:
\begin{gather}
\begin{aligned}
\mu_s = p \mathcal{G} + \ln c_s, 
\end{aligned}\\[2.5mm]
\mu_\pm = p \mathcal{G} \pm \Phi + \ln c_\pm ,\\
-\epsilon_r\beta^2\partial_{zz} \Phi = c_+-c_-\, \\
T_z= -p-\frac{\epsilon_r\beta^2 }{2\mathcal{G}} (\partial_z \Phi)^2,\\
\partial_z T_z=0\\
v=c_sv_s +c_+v_++c_-v_-=V(t),\\
\partial_z v = 0,\\
\partial_t c_s + \partial_z (c_sV(t))=-\partial_z j_s,\\
\partial_t c_\pm +\partial_z(c_\pm V(t))= -\partial_z j_\pm,\label{fluxes1.2}\\
j_s =c_s\left(v_s-V(t)\right)=-j_+-j_-,\\
j_\pm= - \frac{D_\pm c_\pm}{D_0}\partial_z \mu_\pm+\frac{c_\pm}{c_s}j_s,\\
\frac{d\ell}{dt}=\left.\color{red}v\right|_{\ell(t)}=V(t),
\end{gather}
where the fluxes are of the form $j_m=c_m (v_m-v)$. Note that the fluxes can again be derived using the friction relations as for the gel:
\begin{eqnarray}
-c_+\nabla \mu_+ = \sum_{m} h_{+m} j_{m} = f_{+-} \left(v_+ - v_-\right) + f_{+s} \left(v_+-v_s\right),\\
-c_-\nabla \mu_- = \sum_{m} h_{-m} j_{m} = f_{-+} \left(v_- - v_+\right) + f_{-s} \left(v_--v_s\right),
\end{eqnarray}
with the friction coefficients are related to the coefficients $h_{ij}$ as follows:
\begin{eqnarray}
f_{+-}= h_{++}c_+c_- - h_{+-}c_-c_+-h_{+-}c_-c_s+h_{+s}c_sc_-,\\
f_{-+}= h_{--}c_+c_- - h_{-+}c_-c_+-h_{-+}c_+c_s+h_{-s}c_+c_s,\\
f_{+s} = h_{++}c_+c_s - h_{+s}c_sc_++h_{+-}c_-c_s-h_{+s}c_sc_-,\\
f_{-s} = h_{--}c_-c_s - h_{-s}c_sc_-+h_{-+}c_+c_s-h_{-s}c_sc_+.
\end{eqnarray}
where we have $f_{+-}=f_{-+}=0$ and $f_{is}$ defined as in the bath.
In order to simplify the above system, we now re-define the ionic fluxes with respect to the solvent velocity, i.e. $\bar{j}_\pm=c_\pm\left(v_\pm-v_s\right)$. Hence we have:
\begin{gather}
\mu_\pm = p \mathcal{G} \pm \Phi + \ln c_\pm ,\\
-\epsilon_r\beta^2\partial_{zz} \Phi = c_+-c_-\, \\
T_z= -p-\frac{\epsilon_r\beta^2 }{2\mathcal{G}} (\partial_z \Phi)^2,\\
\partial_z T_z=0\\
\color{red}
\partial_t c_\pm +\partial_z(c_\pm v_s)= -\partial_z \bar{j}_\pm,\\
\bar{j}_\pm= - \frac{D_\pm c_\pm}{D_0}\partial_z \mu_\pm\\
v_s =-\bar{j}_+-\bar{j}_-+V(t),\\
\frac{d\ell}{dt}=V(t)
\end{gather}
When looking at the leading order in $\beta$, we again obtain the electro-neutrality condition:
\begin{equation}
c_-=c_+=c.
\end{equation}
Again subtracting~(\ref{fluxes1.2}), dropping the bar notation, and using the above condition we have:
\begin{equation}
\partial_z(j_+-j_-)=0.\label{temp1.2}
\end{equation}
At the leading order we also have:
\begin{equation}
T_z =-p \quad \Rightarrow p=const=0.
\end{equation}
So that the fluxes can be written as:
\begin{gather}
\frac{d\ell}{dt}=V(t),\\
j_+= - \frac{D_+}{D_0}\left(c\partial_z\Phi + \partial_z c\right),\\
j_-= - \frac{D_-}{D_0}\left(-c\partial_z\Phi + \partial_z c\right),
\end{gather}
So that~(\ref{temp1.2}) can be rewritten as:
\begin{gather}
c\partial_z\Phi+\frac{D_+-D_-}{D_++D_-}\partial_zc=G(t),\\
\partial_t c + \partial_z\left(cv_s\right)=\frac{2D_+D_-}{D_0(D_++D_-)}\partial_{zz}c,\\
v_s=\frac{D_+-D_-}{D_0}G(t)+V(t)+\frac{(D_-+D_+)(D_0-D_++D_-)}{D^2_0}\partial_z c.
\end{gather}
Note that in the simplified case $D_-=D_+$ the above set reduces to:
\begin{gather}
\partial_z\left(c\partial_z\Phi\right)=0\\
\partial_t c +\partial_z(cv_s)= \frac{D}{D_0} \partial_{zz} c,\\
v_s = V(t)+ \frac{2D}{D_0}\partial_z c,
\end{gather}
and we denote $\Phi^{bath}_0=\lim_{z\rightarrow h(t)^+} \Phi$, $c^{bath}_0=\lim_{z\rightarrow h(t)^+} c$.
\section{Inner Layer.}
The interface between the gel and the bath is moving in eulerian coordinates. This is identified by $z=h(t)$ where:
\begin{equation}
h(t)-1=\int_0^1{C^{gel}_s+C^{gel}_++C^{gel}_-}dZ = \int_0^{h(t)} c^{gel}_s+c^{gel}_++c^{gel}_- dz.
\end{equation}
Taking the time-derivative of the above equation we have:
\begin{equation}
\begin{aligned}
\frac{d h}{dt}=\int_0^{h(t)}\partial_z\left(c^{gel}_s+c^{gel}_++c^{gel}_-\right)dz + \frac{d h}{dt}\sum_m c^{gel}_m(h(t),t)\\
\Rightarrow \frac{d h}{dt}=-\frac{\sum_m c_m v_m}{1-\sum_m c_\alpha}=-\phi_n^{-1}\sum_m c_m v_m=v_n(h(t),t)
\end{aligned}
\end{equation}
where $phi_n$ is the volume fraction of the network. In order to move in the inner layer we now consider the following change of variable:
\begin{equation}
z= h(t)+ \beta \xi
\end{equation}
so that the derivative in time and space become:
\begin{equation}
\frac{\partial}{\partial z} = \beta^{-1}\frac{\partial}{\partial \xi}, \qquad \frac{\partial}{\partial t}= \frac{\partial}{\partial t} - \beta^{-1} \frac{d h}{dt} \frac{\partial}{\partial \xi}
\end{equation}
Before moving to analyse the bath, we need to impose the boundary condition at the interface $h(t)$:
\begin{equation}
\begin{aligned}
\left[\phi\right]^+_-=0, \qquad \left[\mu_i\right]^+_-=0, \qquad \epsilon_r \partial_z \phi^+=\partial_z \phi^-\\
\left[T_z\right]^+_-=0, \qquad \left[c_i\left(v_i-\frac{dh}{dt}\right)\right]^+_-=0, \qquad\left.\partial_z C_s \right|_{h^-(t)}=0.
\end{aligned}
\end{equation}
\subsection{Bath.}
We now study the inner layer  in the bath.
%\begin{gather}
%\begin{aligned}
%\mu_s = p \mathcal{G} + \ln c_s, 
%\end{aligned}\\[2.5mm]
%\mu_\pm = p \mathcal{G} \pm \Phi + \ln c_\pm ,\\
%-\epsilon_r\partial_{\xi\xi} \Phi = c_+-c_-\, \label{Poi2.1}\\
%T_z= -p-\frac{\epsilon_r }{2\mathcal{G}} (\partial_z \Phi)^2,\\
%\partial_\xi T_z=0\\
%\beta \partial_t c_s -\frac{dh}{dt}\frac{\partial c_s}{\partial \xi}+\partial_\xi(c_s v)= -\partial_\xi j_s,\\
%\beta \partial_t c_\pm -\frac{dh}{dt}\frac{\partial c_\pm}{\partial \xi}+\partial_\xi(c_\pm v)= -\partial_\xi j_\pm,\\
%\beta j_s =-c_s\frac{D^{bath}}{D_0}  \left(\partial_\xi\mu_s +\sum_i \frac{c_i}{c_s} \partial_\xi \mu_i\right)\\
%\beta j_\pm= - \frac{D_\pm c_\pm}{D_0}\partial_\xi \mu_\pm+\beta\frac{c_\pm}{c_s}j_s
%\end{gather}
\begin{gather}
\mu_s = p \mathcal{G} + \ln c_s, \\
\mu_\pm = p \mathcal{G} \pm \Phi + \ln c_\pm ,\\
-\epsilon_r\partial_{\xi\xi} \Phi = c_+-c_-\, \label{Poi2.1}\\
T_z= -p-\frac{\epsilon_r }{2\mathcal{G}} (\partial_z \Phi)^2,\\
\partial_\xi T_z=0\\
\beta \partial_t c_s -\frac{dh}{dt}\frac{\partial c_s}{\partial \xi}+\partial_\xi(c_s v_s)= 0,\\
\beta \partial_t c_\pm -\frac{dh}{dt}\frac{\partial c_\pm}{\partial \xi}+\partial_\xi(c_\pm v_s)= -\partial_\xi j_\pm,\\
\beta j_\pm= - \frac{D_\pm c_\pm}{D_0}\partial_\xi \mu_\pm,
\end{gather}
so that assuming $\mu_m\sim O(1)$ at the leading order in $\beta$ we obtain:
\begin{gather}
0= - \partial_\xi \mu_\pm ,
\end{gather}
which corresponds to constant chemical potential in the layer:
\begin{gather}
\mu^{bath}_+\equiv \lim_{\xi\rightarrow \infty}\mu^{bath}_+= \Phi_0^{bath}+\ln c_0 \Rightarrow c_+ = c_0\exp[-(\Phi-\Phi_0^{bath})-p\mathcal{G}]\\
\mu^{bath}_-\equiv \lim_{\xi\rightarrow \infty}\mu^{bath}_-= -\Phi_0^{bath}+\ln c_0\Rightarrow c_- = c_0\exp[(\Phi-\Phi_0^{bath})-p\mathcal{G}]\\
 c_s = 1-2c_0\exp[-p\mathcal{G}]\cosh(\Phi-\Phi_0^{bath}).
\end{gather}
Using the above and Equation~(\ref{Poi2.1}):
\begin{equation}
-\epsilon_r\partial_{\xi\xi} \Phi = 2c_0\exp(-p\mathcal{G}) \sinh(-(\Phi-\Phi_0^{bath})).
\end{equation}
We further have that the stress is constant so that:
\begin{equation}
\begin{aligned}
T_z \equiv\lim_{\xi\rightarrow\infty}T_z=0 \Rightarrow \mathcal{G}\exp(p\mathcal{G})\partial_\xi p= 2c_0\sinh(\Phi-\Phi_0^{bath})\partial_\xi \Phi\\
\Rightarrow \exp(p\mathcal{G})-2c_0\cosh(\Phi-\Phi_0^{bath})\equiv const\\
\lim_{\xi\rightarrow\infty}\exp(p\mathcal{G})-2c_0\cosh(\Phi-\Phi_0^{bath}) =1-2c_0\\
 \Rightarrow p = \mathcal{G}^{-1}\ln\left[1+2c_0\left(\cosh(\Phi-\Phi_0^{bath})-1\right)\right] 
\end{aligned}
\end{equation}
so that:
\begin{equation}
-\epsilon_r\partial_{\xi\xi} \Phi =-2c_0 \frac{\sinh((\Phi-\Phi_0^{bath}))}{1+2c_0\left(\cosh(\Phi-\Phi_0^{bath})-1\right)}.
\end{equation}
Finally we have a condition on the fluxes:
\begin{eqnarray}
\begin{aligned}
\frac{\partial }{\partial \xi} \left[c_s\left(v_s -\frac{dh}{dt}\right)\right]=  0 \Rightarrow c_s\left(v_s -\frac{dh}{dt}\right)=\alpha_s(t),\\
\lim_{\xi\rightarrow\infty} \alpha_s(t) = (1-2c_0)\left(v^b_{s,0}-\frac{dh}{dt}\right)\\
 \Rightarrow  c_s \left(v_s-\frac{dh}{dt}\right)=(1-2c_0)\left(v^b_{s,0}-\frac{dh}{dt}\right)
\end{aligned}\\[4mm]
\begin{aligned}
\frac{\partial}{\partial \xi}\left[c_+\left(v_s-\frac{dh}{dt}\right)+j_+\right]= 0\Rightarrow -c_+ \frac{dh}{dt}+c_+v_+=\alpha_{+}(t)\\
\lim_{\xi\rightarrow\infty} \alpha_+(t) = -c_0 \frac{dh}{dt}  +j^b_{+,0} +c_0 v^b_{s,0}\\
\Rightarrow  c_+ \left(\frac{dh}{dt}-v_+\right)=c_0 \left(\frac{dh}{dt}-v^b_{s,0}\right) -j^b_{+,0}
\end{aligned}\\[4mm]
\begin{aligned}
\frac{\partial}{\partial \xi}\left[c_-\left(v_s-\frac{dh}{dt}\right)+j_-\right]= 0\Rightarrow -c_- \frac{dh}{dt}+c_-v_-=\alpha_{-}(t)\\
\lim_{\xi\rightarrow\infty} \alpha_-(t) = -c_0 \frac{dh}{dt}  +j^b_{-,0} +c_0 v^b_{s,0}\\
\Rightarrow  c_- \left(\frac{dh}{dt} -v_-\right)=c_0 \left(\frac{dh}{dt} -v^b_{s,0}\right) -j^b_{+,0}-G(t).
\end{aligned}\\
\end{eqnarray}
Consequently the set of equation governing the inner layer in the bath are:
\begin{gather}
-\epsilon_r\partial_{\xi\xi} \Phi^{bath} = 2c_0\exp(-p^{bath}\mathcal{G}) \sinh(\Phi^{bath}-\Phi_0^{bath})\\
p^{bath} = \mathcal{G}^{-1}\ln\left[1+2c_0\left(\cosh(\Phi-\Phi_0^{bath})-1\right)\right] \\
c^{bath}_+ = c_0\exp(-p^{bath}\mathcal{G})\exp[-(\Phi^{bath}-\Phi_0^{bath})],\\
c^{bath}_- = c_0\exp(-p^{bath}\mathcal{G}) \exp[(\Phi^{bath}-\Phi_0^{bath})],\\
c^{bath}_s = \frac{1-2c_0}{1+2c_0\left(\cosh(\Phi-\Phi_0^{bath})-1\right)}.,\\
j_\pm=c^{bath}_\pm\frac{dh}{dt}- c_0 \frac{dh}{dt}  -\frac{D_\pm}{D_0}\left(\pm c_0 \partial_z \Phi_0+\partial_z c_0\right).
\end{gather}
Note also that by the definition of $\mu_s$, we have:
\begin{equation}
\mu_s= \mathcal{G}p^{bath}+\ln c^{bath}_s=\ln(1-2c_0)= const.
\end{equation}
\subsection{Inner Gel.}
Considering instead the gel:
\begin{gather}
\begin{aligned}
\mu_s = p \mathcal{G} - \frac{\omega^2}{\beta^2} \partial_{\xi\xi}c_s+ \ln c_s+\frac{\chi(1-c_s)+1}{J} , 
\end{aligned}\label{mus2.2}\\[2.5mm]
\mu_\pm = p \mathcal{G} \pm \Phi  + \ln (c_\pm) +\frac{1-\chi c_s}{J} ,\\
-\partial_{\xi\xi} \Phi = c_+-c_-+c_f\,\label{Poi2.2} \\
T_z= -p+\frac{\omega^2}{2\mathcal{G}\beta^2} \left[c_s\partial_{\xi\xi}c_s-\frac{(\partial_\xi c_s)^2}{2}\right]+ \frac{J^2-1}{J}-\frac{1}{2\mathcal{G}} (\partial_\xi \Phi)^2,\label{T2.2}\\
\partial_\xi T_z=0\\
\beta \partial_t c_s -\frac{dh}{dt}\frac{\partial c_s}{\partial \xi}+\partial_\xi (c_s v_n)= - \partial_\xi j_s,\label{gov2.2}\\
\beta \partial_t c_\pm -\frac{dh}{dt}\frac{\partial c_\pm}{\partial \xi}+\partial_\xi (c_\pm v_n)= -\partial_\xi j_\pm,\label{gov2.2_2}\\
\beta j_s =-c_s K  \left(\partial_\xi\mu_s +\sum_i \frac{c_i}{c_s} \partial_\xi \mu_i\right)\\
\beta j_\pm= - \frac{D_\pm c_\pm}{D_0}\partial_\xi \mu_\pm + \beta \frac{c_\pm}{c_s}j_s
\end{gather}
Summing Equations~(\ref{gov2.2})-(\ref{gov2.2_2}), we have:
\begin{equation}
\begin{aligned}
-\frac{dh}{dt}\frac{\partial \sum_m c_m}{\partial \xi}+ \partial_\xi \left(\sum_m c_m v_m\right)= \frac{dh}{dt}\frac{\partial c_n}{\partial \xi}-\partial_\xi (c_n v_n)\\
\Rightarrow  \frac{dh}{dt} c_n -c_n v_n=const(t)=0 \Rightarrow  \frac{dh}{dt} \equiv v_n
\end{aligned}
\end{equation}
so that in the layer the velocity of the network remain constant in space.
\color{red}
As before, we consider the chemical potential and the fluxes to be $O(1)$, so as the tensor $T$. Under the assumption that:
\begin{equation}
\omega^2 >> O(\beta) 
\end{equation}
we have that the Debye layer remains the smaller scale of the problem. Let us consider the following expansion of the solution:
\begin{equation}
c_s(\xi,t;\beta) = \alpha_0(\xi,t) + \beta \alpha_1(\xi,t) + \beta^2 \alpha_2(\xi,t)+ h.o.t\,. 
\end{equation}
Substituting the above into~(\ref{mus2.2}), we obtain:
\begin{eqnarray}
&O(\beta^{-2})& \partial_{\xi\xi}\alpha_0=0 \label{cond1}\\
&O(\beta^{-2})& \partial_{\xi}\alpha_0=0 \Rightarrow \alpha_0=a_0(t) \\
&O(\beta^{-1})& \partial_{\xi\xi}\alpha_1=0 \Rightarrow \alpha_1=a_1(t)\label{cond2},
\end{eqnarray} 
with the boundary condition $\partial_{\xi} \alpha_0-\partial_\xi \alpha_1=0$. So that in the Debye layer $c_s= a_0(t) + \beta a_1(t) + O(\beta^2)$.
\color{black}
We can now impose the continuity of the chemical potential:
\begin{gather}
\color{blue}\mu_s \equiv const = \mu^{bath}_s= \ln(1-2c_0),\\
\begin{aligned}
\mu_+ &\equiv const = \mu^{bath}_+=  \Phi_0^{bath} + \ln c_0 \\&\Rightarrow c_+=c_0\exp\left[-(\Phi-\Phi_0^{bath})-p\mathcal{G}-\frac{1-\chi c_s}{J}\right]
\end{aligned}\\
\begin{aligned}
\mu_- &\equiv const = \mu^{bath}_-= - \Phi_0^{bath} + \ln c_0\\ &\Rightarrow c_-=c_0\exp\left[(\Phi-\Phi_0^{bath})-p\mathcal{G}-\frac{1-\chi c_s}{J}\right]
\end{aligned}
\end{gather}
We also have that the stress tensor is constant so that:
\begin{equation}
T_z\equiv T^{bath}_z=0 \Rightarrow p = \frac{\omega^2}{2\mathcal{G}} a_0\partial_{\xi\xi}\alpha_2+\frac{J^2-1}{J}-\frac{1}{2\mathcal{G}} (\partial_\xi \Phi)^2.
\end{equation}
Using now Equations~(\ref{gov2.2})-(\ref{gov2.2_2}), we obtain:
\begin{equation}
c_s v_s - c_s \frac{dh}{dt}=A_s(t), \qquad c_\pm v_\pm - c_\pm \frac{dh}{dt}=A_\pm(t)
\end{equation}
and imposing the boundary condition at the interface:
\begin{gather}
c_s v_s - c_s \frac{dh}{dt}=(1-2c_0)\left[v^b_{s,0}-\frac{dh}{dt}\right]\\
c_\pm v_\pm - c_\pm \frac{dh}{dt}=c_0\left( v^b_{s,0} - \frac{dh}{dt}\right)  +j^0_{\pm,0}.
\end{gather}

\subsection{Matching with the outer gel}
We now use the following notation:
\begin{equation}
\begin{aligned}
\Phi^{gel}_0=\lim_{z\rightarrow h(t)^-} \Phi^{gel}, \qquad c^{gel}_{m,0}=\lim_{z\rightarrow h(t)^-} c^{gel}_{m} \qquad p_0^{gel}=\lim_{z\rightarrow h(t)^-} p^{gel},\\
J^{gel}_0=\lim_{z\rightarrow h(t)^-} J^{gel}, \qquad j^{gel}_{m,0}=\lim_{z\rightarrow h(t)^-} j^{gel}_m
\end{aligned}
\end{equation}
Matching also with the exterior solution in the bulk we have that:
\begin{gather}
\begin{aligned}
&\lim_{\xi\rightarrow -\infty} T_z = T^{gel}_z =0 \\
p^{gel}_0& =  \frac{(J^{gel}_0)^2-1}{J^{gel}_0}+\frac{\omega^2}{2\mathcal{G}} c_s\partial_{zz}c_s\label{eq1}
\end{aligned}\\
\lim_{\xi\rightarrow -\infty} c_\pm = c^{gel}_{\pm,0} = c_0\exp\left[\mp(\Phi^{gel}_0-\Phi_0^{bath})-p_0^{gel}\mathcal{G}-\frac{1-\chi c^{gel}_{s,0}}{J^{gel}_0}\right].
\end{gather}
\begin{gather}
\lim_{\xi\rightarrow -\infty} c_s v_s - c_s \frac{dh}{dt}= j^{gel}_{s,0} =(1-2c_0)\left[v^b_{s,0}-\frac{dh}{dt}\right]\\
\begin{aligned}
\lim_{\xi\rightarrow -\infty}c_\pm v_\pm - c_\pm \frac{dh}{dt}=j^{gel}_{\pm,0}\\
=c_0\left(v^b_{s,0}-\frac{dh}{dt}\right)+j^b_{\pm,0}.
\end{aligned}
\end{gather}
In the case of $C_f$ homogeneous, then $j^{gel}_{+,0}=j^{gel}_{-,0}$ in the gel we have that $G(t)=0$ in the bath. Summing also the fluxes equations we obtain that $V(t)=0$. So that the problem in the bathroom simplifies to:
\begin{eqnarray}
c\partial_z\Phi^{bath}+\frac{D_+-D_-}{D_++D_-}\partial_zc^{bath}=0,\label{A}\\
\partial_t c + \partial_z\left(cv_s\right)=\frac{2D_+D_-}{D_0(D_++D_-)}\partial_{zz}c,\\[2mm]
v_s=\frac{(D_-+D_+)(D_0-D_++D_-)}{D^2_0}\partial_z c. \label{B}
\end{eqnarray}
Since in the outer layer electro-neutrality must hold:
\begin{equation}
c^{gel}_{+,0}-c^{gel}_{-,0}+z_fc_{f,0} =0 \Rightarrow  \sinh(\Phi_0^{gel}-\Phi_0^{bath})= \frac{C_f}{2c_0 J_0^{gel}}\exp\left(p_0^{gel}\mathcal{G}+\frac{1-\chi c^{gel}_{s,0}}{J^{gel}_0}\right)\label{eq2}
\end{equation}
By definition of $J^{gel}_0=1+C^{gel}_{s,0}+C^{gel}_{+,0}+C^{gel}_{-,0}$ we also have that:
\begin{equation}
\frac{(J^{gel}_0)^2-1}{J^{gel}_0}=c^{gel}_{s,0}+ 2c_0\exp\left(-p_0^{gel}\mathcal{G}-\frac{1-\chi c^{gel}_{s,0}}{J^{gel}_0}\right)\cosh[(\Phi^{gel}_0-\Phi_0^{bath})]\label{eq3}.
\end{equation}
Given the inner expansion of $C_s$ we have that:
\begin{equation}
\partial_\xi c_s = O(\beta^2) \Rightarrow \partial_z c^{gel}_{s} = O(\beta)
\end{equation}
which requires that at the leading order $\partial_z c^{gel}_{s,0} = 0$.
Consequently matching also the chemical potential of the solvent $\mu_s$:
\begin{equation}
\begin{aligned}
\color{red}\omega^2\left(1-\frac{c_s}{2}\right) \partial_{zz}c^{gel}_{s,0}\color{black}= \frac{(J^0_{gel})^2-1}{J^0_{gel}} \mathcal{G} +\frac{\chi(1-c^{gel}_{s,0})+1}{J} + \ln \frac{c^{gel}_{s,0}}{(1-2c_0)}, 
\end{aligned}\\[2.5mm]\label{eq4}
\end{equation}

Given that we have for unknowns $\Phi^{gel}_0$, $p^{gel}_0$, $C^{gel}_{s,0}$ and $J^{gel}_0$, we can solve for them using Equations~(\ref{eq1})-(\ref{eq2})-(\ref{eq3})-(\ref{eq4}). 
To summarise for the bath we have the following boundary condition at the interface:
\begin{eqnarray}
\sinh(\Phi-\Phi_0^{bath})= \frac{C_f}{2c_0 J}\exp\left(p\mathcal{G}+\frac{1-\chi c_s}{J}\right), & z(t)=h^-(t)&\\
\begin{aligned}
p=\frac{J^2-1}{J}\frac{2}{2-c_s}+\frac{c_s}{\mathcal{G}(2-c_s)}\left[\ln\frac{c_s}{1-2c_0}\right.\\
\left.+\frac{\chi(1-c_s)+1}{J}\right]
\end{aligned},& z(t)=h^-(t)&\\
\frac{J-1}{J}=c_s+ 2c_++z_f\frac{C_f}{J}, & z(t)=h^-(t)&\\[4mm]
\partial_z c_{s}=0,&  z(t)=h^-(t),&\\[5mm]
c_{+} = c_0\exp\left[-(\Phi-\Phi_0^{bath})-p\mathcal{G}-\frac{1-\chi c_s}{J}\right], &z(t)=h^-(t),&\\
\begin{aligned}
\omega^2\left(1-\frac{c_s}{2}\right) \partial_{zz}c_s= \frac{J^2-1}{J} \mathcal{G} +\frac{\chi(1-c_s)+1}{J} \\
+ \ln \frac{c_s}{(1-2c_0)}  
\end{aligned}, & z(t)=h^-(t)&
\end{eqnarray}
and the governing equations:
\begin{gather}
\mu_s = p \mathcal{G}  - \omega^2\partial_{zz} c_s+ \ln c_s+\frac{\chi(1-c_s)+1}{J},\\
\mu_\pm = p \mathcal{G} \pm \Phi + \ln (c_\pm) +\frac{1-\chi c_s}{J} ,\\
p=\frac{\omega^2}{2\mathcal{G}} \left[c_s\partial_{zz}c_s-\frac{(\partial_z c_s)^2}{2}\right]+ \frac{J^2-1}{J},\\
j_s =-\frac{c^2_sKD_-}{c_sD_-+z_fc_fD_0K}  \left(\partial_z\mu_s +\left(1+\frac{D_+}{D_-}\right)\frac{c_+}{c_s}\partial_z\mu_+\right)\\
j_+= - \frac{D_+ c_+}{D_0}\partial_z \mu_+ + \frac{c_+}{c_s}j_s\\
v_n = - j_s-2j_+,\\
\partial_t c_s + \partial_z (c_s v_n)=- \partial_z j_s,\\
\partial_t c_+ +\partial_z (c_+ v_n)= -\partial_zj_+,\\
c_-=c_++z_fc_f,\\
\partial_z \Phi = \frac{D_0 j_+}{D_-c_-}- \frac{D_0 j_s}{D_-c_s} + \frac{\partial_z c_-}{c_-}-(1-\chi c_s)\frac{\partial_z J}{J^2}-\frac{\chi}{J}\partial_z c_s+\mathcal{G}\partial_z p,
\end{gather}
with the additional boundary condition at the substrate:
\begin{equation}
j_+= j_s = 0, \qquad \color{red} \partial_z c_s=0 \qquad at\  z=0.
\end{equation}
Finally $c_0$ and $\Phi^{bath}_0$ can be determined using the outer problem in the bath and the matching condition ~(\ref{A})-(\ref{B}):
\begin{eqnarray}
 \partial_z \Phi = \frac{D_--D_+}{D_-+D_+} \frac{\partial_z c}{c}, &z\in\left(h(t),\ell(t)\right)&\\
\partial_t c + v_\infty \partial_z c= \partial_z\left[2\frac{D_+D_-}{D_0(D_++D_-)}  \partial_z c\right], &z\in\left(h(t),\ell(t)\right),&\\
\left(j^{gel}_+-\frac{j_{s,0}^{gel}c}{1-2c}\right) =- 2 \frac{D_+D_-}{D_0(D_++D_-)}\partial_z c, & z=h^+(t),&\\
\frac{d\ell}{dt}=v_\infty=\frac{j^{gel}_{s,0}}{1-2c_0}+\frac{dh}{dt},\\
\Phi(\ell(t),t)=\Phi_\infty=0,\quad c(\ell(t),t)=c_\infty.
\end{eqnarray}
The system can be simplified by substituting for the chemical potentials, pressure and $c_-$:
\begin{gather}
\tilde{k}=\frac{c^2_sKD_-}{c_sD_-+z_fc_fD_0K},\qquad \alpha=\frac{1-z_fC_f}{1+z_fC_f},\\
a_{1,s}(c_s,c_+)=\omega^2\left(\frac{c_s+2c_+}{2}-1+\frac{D_+}{D_-}c_+\right),\\
\scriptsize
a_{2,s}(c_s,c_+)=2\frac{\mathcal{G}(1+J^2)-\chi(1-c_s)-1}{1+z_fC_f}\\
+\frac{1}{c_s}\left(1+\frac{D_+}{D_-}\right)\left(1+2\frac{\mathcal{G}(1+J^2)-1+\chi c_s}{1+z_fC_f}c_+\right),\\[2mm]
\begin{aligned}
a_{3,s}(c_s,c_+)=1-\frac{c_s\chi}{J}+c_s\frac{\mathcal{G}(1+J^2)-1-\chi(1-c_s)}{1+z_fC_f}\\
+\left(1+\frac{D_+}{D_-}\right)\left(c_+\frac{\mathcal{G}(1+J^2)-1+\chi c_s}{1+z_fC_f}-\chi\frac{c_+}{J}\right),
\end{aligned}\\[2mm]
a_{4,s}(c_s,c_+)=\frac{c_+}{c_s}\left(1+\frac{D_+}{D_-}\right),\\
j_s =- \tilde{k} \left(a_{1,s}\partial_{zzz}c_s +a_{2,s}\partial_z c_++a_{3,s}\frac{\partial_z c_s}{c_s}+a_{4,s}\partial_z\tilde{\Phi}\right)\\
\begin{aligned}
j_+= - \frac{D_+ c_+}{D_0}\left(\omega^2c_s\partial_{zzz}c_s+\frac{\partial_zc_+}{c_+}\left(1+\frac{\mathcal{G}(1+J^2)-1+\chi c_s}{1+z_fC_f}2c_+\right)\right.
\\\left.+ \left[\frac{\mathcal{G}(1+J^2)-1+\chi c_s}{1+z_fC_f}-\frac{\chi}{J}\right]\partial_zc_s+\partial_z\tilde{\Phi}\right) +\frac{c_+}{c_s}j_s
\end{aligned}\\
v_n = - j_s-2j_+,\\
\partial_t c_s + \partial_z (c_s v_n)=- \partial_z j_s,\\
\partial_t c_+ +\partial_z (c_+ v_n)= -\partial_zj_+,\\
c_-=c_++z_fc_f,\\
\begin{aligned}
\partial_z \tilde{\Phi} = \frac{D_0 j_+}{D_-c_-}- \frac{D_0 j_s}{D_-c_s} + \partial_z c_s\left(\frac{\alpha+1}{2}(\mathcal{G}(1+J^2)-1+\chi c_s)-\frac{\chi}{J}-\frac{1-\alpha}{2c_-}\right)\\
+\partial_z c_+\left((\alpha+1)(\mathcal{G}(1+J^2)-1+\chi c_s)+\frac{\alpha}{c_-}\right),
\end{aligned}
\end{gather}
\subsection{Fixed domain.}
In order to solve the problem numerically we move to a fixed domain by using the standard transformation:
\begin{equation}
Z= \frac{z}{h(t)}, \quad \frac{d h}{dt}=\left.v_n\right|_{h(t)}
\end{equation}
Consequently, we need to apply the following change of variables:
\begin{eqnarray}
\partial_z(\cdot) = \frac{\partial_Z (\cdot)}{h(t)},\\
\partial_t(\cdot) = \partial_t (\cdot) - \frac{h'(t)Z}{h(t)}\partial_Z(\cdot).
\end{eqnarray}
So that the equations~() turn into:
\begin{gather}
\tilde{j}_s =- \tilde{k} \left(\frac{a_{1,s}}{h^2(t)}\partial_{ZZZ}c_s +a_{2,s}\partial_Z c_++a_{3,s}\frac{\partial_Z c_s}{c_s}+a_{4,s}\partial_Z\Phi\right)\\
\begin{aligned}
\tilde{j}_+= - \frac{D_+ c_+}{D_0}\left(\omega^2\frac{c_s}{h^2(t)}\partial_{ZZZ}c_s+\frac{\partial_Zc_+}{c_+}\left(1+\frac{\mathcal{G}(1+J^2)-1+\chi c_s}{1+z_fC_f}2c_+\right)\right.
\\\left.+ \left[\frac{\mathcal{G}(1+J^2)-1+\chi c_s}{1+z_fC_f}-\frac{\chi}{J}\right]\partial_Zc_s+\partial_Z\Phi\right) +\frac{c_+}{c_s}j_s
\end{aligned}\\[2mm]
\partial_t c_s  -\frac{h'(t)h(t)Z+\tilde{j}_s+2\tilde{j}_+}{h^2(t)}\partial_Zc_s= \frac{c_s-1}{h^2(t)}\partial_Z \tilde{j}_s+\frac{2c_s}{h^2(t)}\partial_Z \tilde{j}_+,\\
\partial_t c_+ -\frac{h'(t)h(t)Z+\tilde{j}_s+2\tilde{j}_+}{h^2(t)}\partial_Zc_+= \frac{2c_+-1}{h^2(t)}\partial_Z \tilde{j}_+ +\frac{c_+}{h^2(t)}\partial_Z \tilde{j}_+ ,\\
c_-=c_++z_fc_f,\\
\begin{aligned}
\partial_Z \Phi = \frac{D_0 \tilde{j}_+}{D_-c_-}- \frac{D_0 \tilde{j}_s}{D_-c_s} + \partial_Z c_s\left(\frac{\alpha+1}{2}(\mathcal{G}(1+J^2)-1+\chi c_s)-\frac{\chi}{J}-\frac{1-\alpha}{2c_-}\right)\\
+\partial_Z c_+\left((\alpha+1)(\mathcal{G}(1+J^2)-1+\chi c_s)+\frac{\alpha}{c_-}\right),
\end{aligned}
\end{gather}
with the boundary conditions:
\begin{eqnarray}
\tilde{j}_s=\tilde{j}_+=0,&  Z=0,\\
\partial_Z c_s =0, & Z=\left\{0,1\right\},\\
\scriptstyle
\frac{\omega^2}{h^2(t)}\left(1-\frac{c_s}{2}\right)\partial_{ZZ}c_s=\frac{J^2-1}{J}\mathcal{G}+\frac{\chi(1-c_s)+1}{J}+ \ln \frac{c_s}{1-2c_0},& Z=1\\
c_+ = c_0 \exp \left[-\left(\Phi-\Phi_0\right)-p\mathcal{G}-\frac{1-\chi c_s}{J}\right], & Z=1,\\
J= \frac{1+z_fC_f}{1-c_s-2c_+}, \\
\Phi-\Phi_0= \sinh^{-1} \left[\frac{z_fc_f}{2c_0}\exp\left(p\mathcal{G}+\frac{1-\chi c_s}{J}\right)\right], & Z=1,\\
\scriptstyle
p=\frac{J^2-1}{J}\frac{2}{2-c_s}+\frac{2}{\mathcal{G}(2-c_s)} \left[\ln\frac{c_s}{1-2c_0}+\frac{\chi(1-c_s)+c_s}{J}\right],& Z=1.
\end{eqnarray}
Finally we have a simple ODE describing the growth of the domain:
\begin{equation}
\frac{d h^2}{dt} = - 2\left(\tilde{j}_s+2\tilde{j}_+\right)_{Z=1}
\end{equation}
Similarly we can rescale the bath domain:
\begin{equation}
Z= \frac{z-h(t)}{\ell(t)-h(t)}+1.
\end{equation}
Consequently, we need to apply the following change of variables:
\begin{eqnarray}
\partial_z(\cdot) = \frac{\partial_Z (\cdot)}{\ell(t)-h(t)},\\
\partial_t(\cdot) = \partial_t (\cdot) + \frac{h'(t)(Z-2)-\ell'(t)(Z-1)}{\ell(t)-h(t)}\partial_Z(\cdot).
\end{eqnarray}
The system of equation can thus be re-written as:
\begin{eqnarray}
\Phi = \underbrace{\frac{D_--D_+}{D_-+D_+}}_{\lambda} \ln\left(\frac{c}{c_\infty}\right), &Z\in\left(1,2\right)&\\[2mm]
\begin{aligned}
w^2\partial_t c + w'& w(2-Z)\partial_Z c=\\ &\partial_Z\left[2\frac{D_+D_-}{D_0(D_++D_-)}  \partial_Z c\right]
\end{aligned}, &Z\in\left(1,2\right),&\\
\frac{w}{h}\left(\tilde{j}_+-\frac{\tilde{j}_sc}{1-2c}\right) =- 2 \frac{D_+D_-}{D_0(D_++D_-)}\partial_Z c, & Z=1,&\\
w'=\frac{\tilde{j}_s}{h(1-2c_0)},\\
w=l(t)-h(t),\\
c(1,t)=c_0, \quad c(2,t)=c_\infty.
\end{eqnarray}

%\input{lagrangian_form}
\end{document}