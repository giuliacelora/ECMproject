\subsection{Energy Balance Inequality.}
\label{sec_ine}
The energy imbalance inequality reads:
\begin{equation}
\frac{\d}{\d t} \left\{\int_R \psi \right\}\leq W(R) + M(R) \label{energyin}
\end{equation}
where $R$ is an arbitrary control volume of the system, $\psi$ is the Helmholtz free energy, $W(R)$ is the rate at which the environment does work on $R$ and $M(R)$ is the inflow of mass due to transport. Considering a control volume $R$ in the reference configuration $\mathcal{B}_0$, the system exchanges mass due to the diffusion of each mobile species, so that $M(R)$ is given by:
\begin{equation}
M(R)= \sum\limits_{m=s,1,\ldots,N} - \int_{\partial R} \mu_m \,\mathbf{J}_m \cdot \mathbf{n} 
\end{equation}
where $\mathbf{n}$ is the unit normal vector to the surface $\partial R$ and $\mu_m$ is the chemical potential associated with each species. Widely used in the thermodynamics of mixture, the chemical potential is a measure of the rate of change in free energy associated with adding one more molecule to a unit volume.

The term $W(R)$, i.e. the rate of work done on the system, is instead decomposed in two contributions, the rate of electrical $W_{el}(R)$ and mechanical work $W_{mec}(R)$. Following \cite{DROZDOVph}, $W_{el}(R)$ is defined as:
\begin{equation}
W_{el}(R) = -\int_{\partial R} \Phi\, \dot{\mathbf{H}}\cdot \mathbf{n}
\end{equation}

Following the work of Gurtin \cite{GURTIN}, we account both for the presence of macro-stresses $\mathbb{S}$ and micro-stresses $\boldsymbol{\xi}$, which arise due to the system heterogeneity \cite{microstress}\footnote{Assuming only interface is between solid and solvent phase.}. As before we only consider the dominant contribution of the solvent while neglecting the solute, so that $W_{mec}(R)$ reads:
\begin{equation}	\centering
%	\begin{subfigure}{0.32\textwidth}
%		\centering
%		\Large
%	\def\svgwidth{0.95\linewidth}
%	\input{latex/images/modelA1.pdf_tex}
%	\caption{Rheological Model A}
%	\label{fig1A}
%	\end{subfigure}
W_{mec}(R) = \int_{\partial R} \left(\boldsymbol{\xi}\cdot \mathbf{n}\right)\dot{C}_s + \int_{\partial R} \mathbb{S}\mathbf{n} \cdot \dot{\mathbf{u}}
\end{equation}
where $\mathbf{u}= \mathbf{x}-\mathbf{X}$ is the displacement vector, which is related to the deformation tensor by $\F=\mathbb{I}-\nabla_0 \mathbf{u}$. Substituting this result back into the formula~(\ref{energyin}) and applying the divergence theorem we obtain the following inequality:
\begin{equation}
\int_R \dot{\psi} - \mathbf{E}\cdot \dot{\mathbf{H}} \, + \, \sum\limits_{i=1}^{N} \left[e \Phi  z_i \dot{C}_i+ \nabla_0 \left(\mu_i \mathbf{J}_i \right)\right] + \nabla_0 (\mu_s \mathbf{J}_s- \boldsymbol{\xi}\dot{C}_s -\mathbb{S}^T\mathbf{\dot{u}}) \leq 0 
\end{equation}

Since this must hold for any choice of the volume $R$, the inequality must hold also locally:
\begin{equation}
\dot{\psi} - \mathbf{E}\cdot \dot{\mathbf{H}} \, + \, \sum\limits_{i=1}^{N} \left[e \Phi  z_i \dot{C}_i+ \nabla_0 \left(\mu_i \mathbf{J}_i \right)\right] + \nabla_0 (\mu_s \mathbf{J}_s- \boldsymbol{\xi}\dot{C}_s -\mathbb{S}^T\mathbf{\dot{u}}) \leq 0. 
\end{equation}
Further accounting for Equations~(\ref{consmass})-(\ref{consmom}), we obtain that:
\begin{equation}
\begin{aligned}
\dot{\psi} - \mathbf{E}\cdot \dot{\mathbf{H}} \, + \, \sum\limits_{i=1}^{N} \left[e \Phi  z_i - \mu_i\right] \dot{C}_i - (\mu_s + \nabla_0 \cdot \boldsymbol{\xi})\,\dot{C}_s -\mathbb{S}:\dot{\F}\\
-\boldsymbol{\xi} \cdot \nabla_0 \, \dot{C}_s + \sum\limits_{m} \nabla_0 \, \mu_m \cdot \mathbf{J}_m \leq 0.
\label{temp2}
\end{aligned} 
\end{equation}

As exhaustively discussed in previous studies \cite{Plasto,GURTIN}, the energy inequality imposes restrictions on the constitutive equation of the free energy $\psi$. Adapting their results to our specific problem, we have that:
\begin{equation}
\psi = \psi (\F,\F_e, C_s, C_i, \nabla_0 \,C_s,\mathbf{H}), \label{temp1}
\end{equation}
which precludes any explicit dependency of $\psi$ on the chemical potential or the viscous deformation gradient $\F_v$. By differentiating the incompressibility condition~(\ref{inc}), we obtain:

\begin{gather}
v_s\dot{C_s} - J \F^{-T}:\dot{\F} =0, \label{temp3}
\end{gather}

If we now substitute~(\ref{temp1}) into~(\ref{temp2}), and include the constraint~(\ref{temp3}) using as Lagrange multipliers $p$, we are left with the augmented form of the energy imbalance inequality:
\begin{equation}
\begin{aligned}
\color{blue}{\left(\frac{\partial \psi}{\partial \nabla_0 C_s}-\boldsymbol{\xi}\right)} \color{black}\cdot \nabla_0 \dot{C}_s + \color{blue}{\left(\frac{\partial \psi}{\partial C_s}-\mu_s-\nabla_0 \cdot \boldsymbol{\xi}+p v\right)}\color{black}\dot{C}_s\\
+ \sum_i\color{blue}\left(\frac{\partial \psi}{\partial C_i} + e\Phi z_i-\mu_i\right) \color{black}\dot{C}_i +\color{blue}\left(\frac{\partial \psi}{\partial \mathbf{H}}-\mathbf{E}\right) \cdot \color{black}\dot{\mathbf{H}}\\
+ \color{blue} \left(\frac{\partial \psi}{\partial \F} - \mathbb{S} - p J \F^{-T}\right): \color{black}\dot{\F}+ \sum_m \nabla_0 \,\mu_m \cdot \mathbf{J}_m \leq 0 . \label{ineq}
\end{aligned}
\end{equation}

\subsection{Construction of the Free Energy.}

Having the general form of $\psi$, Equation~(\ref{temp1}), it remains to construct its precise form. Following a standard approach in $\psi$-depending modeling, we assume that the total free energy can be additively decomposed with each physical mechanisms contributing independently. We here consider six distinct contributions:

\begin{enumerate}
	{\indentitem\item[\textbullet] the energy of the electric field $\psi_1$;}
	{\indentitem \item[\textbullet] the energy of solvent and solutes' molecules not interacting with the solid phase $\psi_2$;}
	{\indentitem\item[\textbullet] the energy of mixing the solid phase with the solution, $\psi_3$;}
	{\indentitem\item[\textbullet] the energy of mixing the solvent with the solutes in solution, $\psi_4$;}
	{\indentitem\item[\textbullet] the interfacial energy between dissimilar phases, $\psi_5$;}
	{\indentitem\item[\textbullet] the energy of the solid phase not interacting with the solution, $\psi_6$.}
\end{enumerate}

Assuming the solid phase to be an ideal and linear dielectric material, with constant permittivity $\epsilon$,the free energy of polarization reads \cite{DROZDOV+,Reviewpolyel}:
\begin{gather}
\psi_1 = \frac{1}{2\epsilon J} \mathbf{H}\F^T \cdot \F \mathbf{H}.
\end{gather}

The specific energy density $\psi_2$ has the standard form:
\begin{equation}
\psi_2 = \sum\limits_{m} \mu^0_m C_m
\end{equation} 
where $\mu^0_m$ denotes the chemical potential of non interacting solvent and ions molecules. According to Flory-Huggins theory \cite{flory,hug} of mixtures, the mixing energy is given by:
\begin{equation}
\psi_3 = \frac{k_B T J}{v_s} \left(\phi_f \ln \phi_f + \chi \phi_f \phi_n\right),\label{mix}
\end{equation}
where $k_B$ is the Boltzmann's constant, $T$ is the temperature and $\chi$ is the Flory-Huggins parameter, which is a measure of the enthalpy of mixing. 

Assuming the solution is dilute, the contribution $\psi_4$ reads \cite{Reviewpolyel,ecm1,ecm2}:
\begin{equation}
\psi_4 = k_B T \sum\limits_{i=1}^{N} C_i \left(\ln \frac{C_i}{ C_s}-1\right).
\end{equation}

For non dilute solution the form proposed by Hong \cite{Reviewpolyel} is:
\begin{equation}
\psi_4 = k_B T \sum\limits_{m} C_m \left(\ln \frac{C_m}{\sum_i C_i+C_s}-1\right).
\end{equation}

As proposed by Hong et al. \cite{Interface}, we include in the energy the effect of interface tension. Assuming that the contribution of mobile ions is negligible, only the solid-solvent interface contributes to the energy:
\begin{equation}
\psi_5 = \frac{\gamma}{2} J \left|\nabla C_s\right|^2,
\end{equation}
where the constant $\gamma$ plays a role analogous to a surface tension.

For the strain energy we consider the gel to be an hyper-elastic Neo-Hookean material:

\begin{equation}
\psi_6(\F) = \frac{G}{2} \left(\F:\F - 3 -2 \ln J\right)\
\end{equation}
where $G$ is the shear constant of the material.

\subsection{Entropy Production $\sigma$.}
\label{ent}

Having specified how the system interacts with its environment, we can now discuss how it dissipates energy. We here consider only the contribution due to the transport of the solution(diffusion of solvent and solutes). The thermodynamic fluxes associated with these two phenomena are $\mathbf{J}_m$, $m=s,1,\ldots,N$, and $\LL_v$. Consequently, the entropy production is of the form:

\begin{equation}
\sigma = \sum_m \zeta_m \cdot \mathbf{J}_m,
\label{dis}
\end{equation}
where $\zeta$s represent the thermodynamic forces associate with each flux. On the other hand, $\nabla_0 \dot{C}_s$, $\dot{C}_s$, $\dot{C}_i$, $\mathbf{\dot{H}}$ and $\dot{\F}$ are the independent physical variables that describe the evolution of reversible process. In order for the energy imbalance inequality~(\ref{ineq}) to hold for any choice of the these fields and given the constraint~(\ref{temp1}) on $\psi$, we have that the terms highlighted in blue in Equation~(\ref{ineq}) must be identically zero. 
\begin{gather}
\boldsymbol{\xi} = \gamma J \,\mathbb{B}^{-1} \,\nabla_0 \,C_s,\label{sys1}\\[2mm]
\begin{aligned}
\mu_s = p v_s + \mu_s^0 - \gamma J \nabla^2 C_s + k_BT&\left[\ln \frac{C_s v_s}{1+C_s v_s} + \frac{1}{1+C_sv_s}\right.\\
&\left.\ \ \ \ \ \ +\frac{\chi}{(1+C_s v_s)^2}-\sum_i \frac{C_i}{C_s}\right], 
\end{aligned}\label{gov1}\\[2.5mm]
\mu_i = \mu^0_i + e\Phi z_i + kT \ln \frac{C_i}{C_s},\label{mu}\\
\mathbf{E} = \frac{1}{\epsilon J} \F^T \F\, \mathbf{H}\, , \qquad -\epsilon J \nabla^2 \Phi = Q\, ,\label{sys2}
\end{gather}
\begin{gather}
\begin{aligned}
\mathbb{T}= -p \mathbb{I} + \underbrace{\gamma \left[\frac{1}{2} |\nabla C_s|^2\mathbb{I} - \nabla C_s \otimes \nabla C_s\right]}_{\mathbb{T}^{kort}}+ \underbrace{\epsilon \left[\frac{1}{2} \,|\nabla \Phi|^2\mathbb{I} -\nabla \Phi \otimes \nabla \Phi\right]}_{\mathbb{T}^{Max}}\\
+ \frac{G}{1+C_sv_s}\left(\mathbb{B}-\mathbb{I}\right),
\end{aligned}
\label{sys3}
\end{gather}

So that the energy imbalance inequality reduces to:
\begin{equation}
\sum_m \nabla_0 \,\mu_m \cdot \mathbf{J}_m \leq 0.
\end{equation}
In the framework of linear non-equilibrium thermodynamics, when considering isothermal transformation, the second law of thermodynamics can be rewrite as:
\begin{equation}
W(R)+M(R)-\frac{\d}{\d t} \left\{\int_R \psi \right\} = T \int_R \sigma \,\d V\, ,
\label{eqCIT}
\end{equation}

Substituting Equations~(\ref{sys1})-(\ref{sys3}) into~(\ref{eqCIT}) and moving from the integral to the local form, we obtain:
\begin{equation}
\begin{aligned}
\sigma = -  \sum_m \frac{1}{T}\nabla_0 \,\mu_m \cdot \mathbf{J}_m. \label{EQen}
\end{aligned} 
\end{equation}

Comparing Equation~(\ref{EQen}) and~(\ref{dis}), it is evident that the thermodynamics forces are:
\begin{equation}
\zeta_m = -\frac{1}{T} \nabla_0 \,\mu_m. \label{vflow1}
\end{equation}
Assuming to be in regime of linear non-equilibrium thermodynamics, we have that forces linearly depend on fluxes:
\begin{equation}
\zeta_k = -\frac{1}{T} \nabla_0 \,\mu_m =\sum_{k=s,1,\ldots,N} H_{mk} \mathbf{J}_m. \label{dif}
\end{equation}

Since the entropy production depends on the deformation, it is more suitable to move from the Lagrangian to the Eulerian coordinates. So that Equation~(\ref{dif}) can be rewritten as:
\begin{equation}
\nabla \,\mu_m  = - \sum_{k=s,1,\ldots,N} T J H_{mk} \mathbb{B}^{-1} \, \mathbf{j}_m= \sum_{k=s,1,\ldots,N} \ell_{mk}\   \mathbf{j}_m. \label{dif2}
\end{equation}
where the coefficient $\ell_{mk}$ are now constant, i.e. they are independent of the deformation. These can be correlated to drag coefficients, which are commonly used in the theory of mixtures \cite{ecm1,ecm2}. Let us first rewrite the fluxes as $\mathbf{j}_m = c_m (\mathbf{v}_m-\mathbf{v}_n)= c_m \bar{\mathbf{v}}_{m}$, where $\mathbf{v}_m$ is the velocity of the $m$-th component in the current configuration, $\mathbf{v}_n$ is the velocity of the network also in the current configuration and  $\bar{\mathbf{v}}_{m}$ is the relative velocity of the $m$-th component with respect to the network. Then Equation~(\ref{dif2}) can be written as:
\begin{eqnarray}
-c_j \nabla \mu_j = \sum_b h_{jb} \bar{\mathbf{v}}_j= \sum_{i\neq j} f_{ji} \left(\bar{\mathbf{v}}_i-\bar{\mathbf{v}}_j\right) + f_{js} (\bar{\mathbf{v}}_s-\bar{\mathbf{v}}_j) + f_{jn} \bar{\mathbf{v}}_j,\label{drag1}\\
-c_s \nabla \mu_s = \sum_i f_{si} \left(\bar{\mathbf{v}}_i-\bar{\mathbf{v}}_s\right)+ f_{sn} \bar{\mathbf{v}}_s,
\end{eqnarray}
where $f_{mi}$ and $h_{mn}$ are the drag coefficients related to the interaction between fluid constituents and the polymer network respectively. Based on the Onsanger's reciprocal relation we have that:
\begin{equation}
f_{mb}=f_{bm}.
\end{equation}
A common assumption in the study of mixture theory is that the solute-solute drag can be neglected so that $f_{ij}=0$ for $i,j=1,\ldots,N$ \cite{ecm1,bookbiophys}. The remaining drag coefficient are instead defined by:
\begin{equation}
f_{sn} = \frac{1}{k}, \ \ f_{js}=\frac{k_BT c_j}{D^0_{j}},\ \  f_{js}+f_{jn}= \frac{k_BT c_j}{D_j}, \label{drag2}
\end{equation}
where $k$ is the hydraulic permeability of the solvent in the network, $D^0_j$ is the diffusion coefficient of the solute in pure solution, while $D_j$ is the diffusion coefficient in the gel.

Using~(\ref{drag1})-(\ref{drag2}), the relative velocities are of the form:
\begin{eqnarray}
\bar{\mathbf{v}}_s = -K  \left(\nabla \mu_s +\sum_i \frac{D_i}{D^0_i} \frac{C_i}{C_s} \nabla \mu_i\right),\label{vbar2}\\
\bar{\mathbf{v}}_j = - \frac{D_j}{k_B T}\nabla \mu_j + \frac{D_j}{D^0_j} \bar{\mathbf{v}}_s, \label{vbar}
\end{eqnarray}
and the coefficient $K$ is defined as:
\begin{equation}
\frac{1}{K} = \frac{1}{c_sk} + \sum_i \frac{k_B T}{D^0_i} \left(1-\frac{D_i}{D^0_i}\right) \frac{c_i}{c_s}.
\end{equation}
If we now move back to the initial configuration, we have that the fluxes $\mathbf{J}_m$ have the form:
\begin{eqnarray}
\mathbf{J}_m = J \F^{-1} \mathbf{j}_m=  \F^{-1} C_m \bar{\mathbf{v}}_s,\label{flux}
\end{eqnarray}
where $\bar{\mathbf{v}}_s$ are defined by Equations~(\ref{vbar2})-(\ref{vbar}). Substituting now Equations~(\ref{flux}) into the conservation of mass laws~(\ref{consmass}), we recover the following governing equations:
\begin{eqnarray}
\partial_t C_s=\nabla_0 \cdot\left[K \F^{-1}\left(C_s\nabla \mu_s +\sum_i \frac{D_i}{D^0_i} C_i \nabla \mu_i\right)\right],\label{gov2}\\
\partial_t C_i= \nabla_0\cdot\left[\frac{D_i}{k_B T}C_i\F^{-1}\nabla \mu_i -\frac{D_i}{D^0_i} \frac{C_i}{C_s} \mathbf{J}_s\right].\label{gov3}
\end{eqnarray}

\subsection{Evolution Equation.}

In order to get a better physical insight into the behaviour, we first rewrite the solvent chemical potential as:
\begin{gather}
\mu_s = \mu^0_s + k_B T \left(\frac{p v_s}{k_BT} +\Pi_{osm}-\sum_i \frac{C_i}{C_s} -\frac{\gamma J}{k_B T}\Pi_{grad}\right)\label{mu2},\\
\Pi_{osm}=\ln \frac{C_s v_s}{1+C_s v_s} + \frac{1}{1+C_sv_s}+\frac{\chi}{(1+C_s v_s)^2},\\
\Pi_{grad} = \nabla^2 C_s,
\end{gather}
where $p$ represents the pore pressure, $\Pi_{osm}$ is the osmotic pressure of the solution and $\Pi_{grad}$ is the pressure due to interface energy. If we now substitute into Equations~(\ref{gov2}) the chemical potentials~(\ref{mu2})-(\ref{mu}), which yields to:
\begin{equation}
\begin{aligned}
\partial_t C_s=\nabla_0 \cdot\left\{K\F^{-1}\left[C_s v_s\nabla p - \color{red}\gamma C_s \nabla J\Pi_{grad}\color{black}+\sum_i \frac{D_i}{D^0_i} C_i e z_i \nabla \Phi\right.\right.\\
\left.\left.+ k_B T\color{teal} \left(C_s \nabla \Pi_{osm}+ \sum_i\left(1-\frac{D_iC_i}{D^0_iC_s}\right) \nabla C_s - \sum_i\left(1-\frac{D_i}{D^0_i}\right) \nabla C_i \right)\color{black}\right]\right\}.\label{long}
\end{aligned}
\end{equation}

The above equation shows that the solvent transport is driven by pressure gradient, osmotic pressure gradient (light-blue term in Equation~(\ref{long})), electric potential gradient and the additional composition gradient (red term in Equation~(\ref{long})), which is here first introduced in the context of polyelectrolytes. In the absence of solutes ($C_i\equiv 0$), we recover the same model presented by Hennessy et. al~\cite{sarah}. 
Similarly we can rewrite Equation~(\ref{gov3}) as:
\begin{equation}
\scriptsize
\partial_t C_i = \nabla_0 \cdot \left[D_i\F^{-1}\left(\underbrace{\nabla C_i}_{\text{diffusion}} +\underbrace{\frac{eC_iz_i}{k_B T} \nabla \Phi}_{\text{electric}}\right)-\underbrace{\frac{D_i C_i}{C_s}\F^{-1}\nabla C_s}_{\text{osmotic pressure}}-\underbrace{\frac{D_i C_i}{D^0_iC_s}\mathbf{J}_s}_{\text{advenction}}\right]\label{long2}
\end{equation}

\subsection{Possible simplification of the model.}

If we consider  the case of small ions, when we can neglect the friction between the ions and the polymer chains, i.e. $D_{i}=D^0_i$. Under this assumption the system of equation simplifies as $K=c_sk$. Following the work of Hennesy et al., we consider the hydraulic permeability $k$ to be:

\begin{equation}
k = \frac{D_0(1+v_sC_s)^\beta}{k_B T c_s} \Rightarrow K=  \frac{D_0(1+v_sC_s)^\beta}{k_B T}.
\end{equation}

The governing equation thus reduces to:
\begin{eqnarray}
\begin{aligned}
\partial_t C_s=\nabla_0 \cdot\left\{\frac{D_0(1+v_sC_s)^\beta}{k_B T}\F^{-1}\left[C_s v_s\nabla p - \gamma C_s \nabla  \left(J\nabla^2 C\right)\right.\right.\\
\left.\left.+\sum_i C_i e z_i \nabla \Phi+ k_B T \left(C_s \nabla \Pi_{osm}+ \sum_i\left(1-\frac{C_i}{C_s}\right) \nabla C_s \right)\right]\right\}.
\end{aligned}\label{A}
\end{eqnarray}
\begin{eqnarray}
\partial_t C_i = \nabla_0 \cdot\left[D_i\F^{-1}\left(\nabla C_i +\frac{eC_iz_i}{k_B T} \nabla \Phi\right)-\frac{D_i C_i}{C_s}\F^{-1}\nabla C_s-\frac{C_i}{C_s}\mathbf{J}_s\right]\label{B}
\end{eqnarray}

\section{1D full model}

Considering a 1D constrained swelling, the deformation gradient tensor is of the form:
\begin{equation}
\F= \begin{bmatrix}
1 & 0 &0\\
0 & 1 &0\\
0 & 0 &J(t,Z)\\
\end{bmatrix}.\label{deffree}                                                                
\end{equation}
where in the dilute case we have $J=1+v_sC_s$.
Similarly all the variables will just depend on the time $t$ and the $Z$ coordinates, the tensor will preserve the same diagonal form, identify by the index $1,2,3$. Given that the set of state and governing equation for the model are given by:

\begin{gather}
\partial_Z S_3 = 0,\\
S_3 = -p - \frac{\gamma}{2} \frac{(\partial_Z C)^2}{J^2} - \frac{\epsilon}{2} \frac{(\partial_Z \Phi)^2}{J^2}+ \frac{G}{J}\left(J^2-1\right),\\[2mm]
-\epsilon \partial_Z \left(J^{-1} \partial_Z \Phi\right)= Q,\\[2mm]
Q= e\left(\sum\limits_{i} z_i C_i + z_fC_f\right)\, ,\\[2mm]
\begin{aligned}
J_s=-\frac{D_0J^{\beta-2}}{k_B T}\left[ - \gamma C_s \partial^2_Z  \left(\frac{\partial_Z C_s}{J}\right)+\sum_i C_i e z_i \partial_Z \Phi+ C_s v_s\partial_Z p\right.\\[2mm]
\left.+ k_B T \left(C_s \partial_Z \Pi_{osm} + \sum_i \left(1-\frac{C_i}{C_s}\right) \partial_Z C_s \right)\right]\, ,
\end{aligned}\\[2mm]
\Pi_{osm}=\ln \frac{C_s v_s}{J} + \frac{1}{J}+\frac{\chi}{J^2},\\[2mm]
\partial_t C_s = -\partial_Z J_s\,,\\[2mm]
\partial_t C_i = \partial_Z \left[\frac{D_i}{J^2}\left(\partial_Z C_i +\frac{eC_iz_i}{k_B T} \partial_Z \Phi\right)-\frac{D_i C_i}{J^2C_s}\partial_Z C_s-\frac{C_i}{C_s}J_s\right]\, .
\end{gather}
 
 