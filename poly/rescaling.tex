\documentclass[12pt]{extarticle}
\usepackage[utf8]{inputenc}
\usepackage[utf8]{inputenc}
\usepackage{tikz}
\usetikzlibrary{er,positioning,bayesnet}
\usepackage{multicol}
\usepackage{xcolor}
\usepackage{algpseudocode,algorithm,algorithmicx}
\usepackage{hyperref}
\usepackage[inline]{enumitem} % Horizontal lists
\usepackage{cite}
\usepackage{amsmath}
\usepackage{amssymb}
\usepackage{graphicx}
\usepackage{multirow}
\usepackage{booktabs}
\usepackage[labelfont=bf,font=scriptsize]{caption}
\usepackage[font=scriptsize]{subcaption}
\usepackage{mhchem}
\newcommand{\indentitem}{\setlength\itemindent{20pt}}
\newcommand{\F}{\ensuremath{\mathbb{F}}}
\newcommand{\B}{\ensuremath{\mathbb{B}}}
\newcommand{\LL}{\ensuremath{\mathbb{L}}}
\renewcommand{\d}{\ensuremath{\text{d}}}
\title{Debye Layer Analysis.}
\author{}

\begin{document}
\section{Non-dimensionalization of the problem.}
\subsection{The Gel.}
We assume the following form for the diffusion coefficients:
\begin{equation}
D_i=D_i^0,\quad k = \frac{D_0(1+v_sC_s)^\beta}{k_B T c_s} \Rightarrow K=  \frac{D_0(1+v_sC_s)^\beta}{k_B T}.
\end{equation}
To simplify the notation, we also assume that the interface parameter $\gamma(C_s)=\gamma_0 f(C_s)$, where $f:[0,\infty]\rightarrow(0,\infty]$ is an analytical function and $\gamma_0$ is a constant. 
Using the following re-scaling for the variables in the model:
\begin{equation*}
\begin{aligned}
k_BT\hat{\mu} + \mu_0 -k_BT= \mu, \qquad \hat{C}_i = vC_i, \qquad \hat{\Phi} = \frac{\Phi e}{k_B T}, \qquad  G_1\hat{p}= p\\
G_1\hat{\mathbb{T}}=\mathbb{T}, \qquad\hat{\mathbf{x}} L =\mathbf{x}, \qquad \hat{t}\tau=t, \qquad e\hat{Q} =v Q, \qquad \hat{K} = \frac{k_BT}{D_0}K\\
\hat{\mathbf{j}}=\frac{vL}{D_0}\mathbf{j}, \qquad \hat{\LL}= \tau \LL, \qquad \tau=\frac{L^2}{D_0}
\end{aligned}
\end{equation*}
where $L$ is the characteristic size of the gel. We identify the following non-dimensional parameter:
\begin{equation*}
\beta=\frac{L_d}{L}, \qquad L_d= \sqrt{\frac{\epsilon k_B Tv}{L^2e^2}}, \qquad \omega= \frac{L_{int}}{L}, \qquad L_{int}=\sqrt{\frac{\gamma_0}{vk_BT}}, \qquad \mathcal{G}=\frac{vG}{k_BT}.
\end{equation*}
We thus identify three length scales which we consider to be related as follows:
\begin{equation}
L_d<<L, \quad L_d<<L_{int} \Rightarrow \beta<<\omega
\end{equation}

Using the eulerian framework the model can be thus written as:

\begin{gather}
\begin{aligned}
\mu_s = p \mathcal{G} - \omega^2 J \left[f(C_s)\nabla^2 C_s+\frac{f'(C_s)}{2}\left|\nabla C_s\right|^2\right]\\
 + \left[\ln \frac{C_s}{1+C_s} + \frac{1}{1+C_s}
 +\frac{\chi}{(1+C_s)^2} + \ln \frac{C_s}{J-1} \right], 
\end{aligned}\\[2.5mm]
\mu_\pm = p \mathcal{G} \pm \Phi + \ln \frac{C_\pm}{J-1} ,\\
-\beta^2 \nabla^2 \Phi = c_+-c_-+z_fc_f\, 
\end{gather}
\begin{gather}
\begin{aligned}
\mathbb{T}= -p \mathbb{I} +\frac{\omega^2f(C_s) }{\mathcal{G}} \left[\frac{1}{2} |\nabla C_s|^2\mathbb{I} - \nabla C_s \otimes \nabla C_s\right]+ J^{-1}\left(\mathbb{B}-\mathbb{I}\right)\\
+ \frac{\beta^2}{\mathcal{G}} \left[\frac{1}{2} \,|\nabla \Phi|^2\mathbb{I} -\nabla \Phi \otimes \nabla \Phi\right],
\end{aligned}\\
\nabla \cdot \mathbb{T}=\mathbf{0}\\
\partial_t c_s + \nabla \cdot(c_s \mathbf{v}_n)=- \nabla \cdot\mathbf{j}_s,\\
\partial_t c_\pm + \nabla \cdot(c_\pm \mathbf{v}_n)= -\nabla\cdot\mathbf{j}_\pm,\label{flux1.1}\\
\mathbf{j}_s =-c_sK  \left(\nabla \mu_s +\sum_i \frac{c_i}{c_s} \nabla \mu_i\right)\\
\mathbf{j}_\pm= - \frac{D_\pm c_\pm}{D_0}\nabla \mu_\pm + \frac{c_\pm}{c_s}\mathbf{j}_s
\end{gather}
where $c_\ell=C_\ell/J$ is the concentration of the $\ell$-th component in the mixture.
Expanding the solution in terms of $\beta$ at the leading order we recover electro-neutrality:
\begin{equation}
c_- - c_+=z_f c_f. \label{neu1.1}
\end{equation}
Subtracting equations~(\ref{flux1.1}) and using~(\ref{neu1.1}):
\begin{equation}
\partial_t c_f +\nabla \cdot(c_f \mathbf{v}_n)= \nabla\cdot(\mathbf{j}_+-\mathbf{j}_-)
\end{equation}
Using the fact that $c_f=C_f/J$ with $C_f$ being uniform and constant and the additional equality:
\begin{equation}
\partial_t J + \mathbf{v}_n \nabla J=J\nabla \cdot\mathbf{v}_n
\end{equation}
we obtain the condition:
\begin{equation}
0= \nabla\cdot(\mathbf{j}_+-\mathbf{j}_-),
\end{equation}
Consequently at the leading order, the 1D problem in the gel we have:
\begin{gather}
\begin{aligned}
\mu_s = p \mathcal{G}  - \omega^2 J \left[f(C_s)\partial_{zz}^2 C_s+\frac{f'(C_s)}{2}\left(\partial_z C_s\right)^2\right]\\
+ \left[\ln \frac{C_s}{1+C_s} + \frac{1}{1+C_s}+\frac{\chi}{(1+C_s)^2} + \ln \frac{C_s}{J-1} \right], 
\end{aligned}\\[2.5mm]
\mu_\pm = p \mathcal{G} \pm \Phi + \ln \frac{C_\pm}{J-1} ,\\
T_z= -p -\frac{\omega^2 f(C_s)}{2\mathcal{G}} (\partial_z C_s)^2+ J^{-1}\left(J-1\right),\\
\partial_z T_z=0\\
C_- = C_+ + z_f C_f,\\
\partial_t c_s + \partial_z (c_s v_n)=- \partial_z j_s,\\
\partial_t c_+ +\partial_z (c_+ v_n)= -\partial_zj_+,\\
0= \partial_z(j_+-j_-) \Rightarrow j_+=j_-,\\
j_s =-c_sK  \left(\partial_z\mu_s +\sum_i \frac{c_i}{c_s} \partial_z \mu_i\right)\\
j_+= - \frac{D_+ c_+}{D_0}\partial_z \mu_+ + \frac{c_+}{c_s}j_s
\end{gather}

\subsection{The bath.}
For the bath we have the following system of equations:
\begin{gather}
\begin{aligned}
\mu_s = p \mathcal{G} + \ln c_s, 
\end{aligned}\\[2.5mm]
\mu_\pm = p \mathcal{G} \pm \Phi z_i + \ln c_\pm ,\\
-\epsilon_r\beta^2\partial_{zz} \Phi = c_+-c_-\, \\
T_z= -p-\frac{\epsilon_r\beta^2 }{2\mathcal{G}} (\partial_z \Phi)^2,\\
\partial_z T_z=0\\
\partial_t c_s = - \partial_z j_s,\\
\partial_t c_\pm = -\partial_z j_\pm,\label{fluxes1.2}\\
j_s =-c_s \frac{D^{bath}}{D_0}  \left(\partial_z\mu_s +\sum_i \frac{c_i}{c_s} \partial_z \mu_i\right)\\
j_\pm= - \frac{D_\pm c_\pm}{D_0}\partial_z \mu_\pm +  \frac{c_\pm}{c_s}j_s
\end{gather}
where the fluxes are of the form $j_m=c_m v_m$.
When looking at the leading order in $beta$, we again obtain the electro-neutrality condition:
\begin{equation}
c_-=c_+=c.
\end{equation}
Again subtracting~(\ref{fluxes1.2}) and using the above condition we have:
\begin{equation}
\partial_z(j_+-j_-)=0.\label{temp1.2}
\end{equation}
At the leading order we also have:
\begin{equation}
T_z =-p \quad \Rightarrow p=const=0.
\end{equation}
So that the fluxes can be written as:
\begin{gather}
j_s =-\frac{D^{bath}}{D_0}  \left(\partial_zc_s + 2 \partial_z c\right)=0\\
j_+= - \frac{D_+}{D_0}\left(c\partial_z\Phi + \partial_z c\right),\\
j_-= - \frac{D_-}{D_0}\left(-c\partial_z\Phi + \partial_z c\right)
\end{gather}
where we have used that $c_s+2c_0=1$, i.e. no void condition.
So that~(\ref{temp1.2}) can be rewritten as:
\begin{gather}
\partial_z\left(\frac{c}{D_0}(D_-+D_+)\partial_z\Phi+\frac{1}{D_0}(D_+-D_-)\partial_zc\right)=0\\
j_+= - \frac{D_+}{D_0}\left(c\partial_z\Phi + \partial_z c\right),\\
j_-= - \frac{D_-}{D_0}\left(-c\partial_z\Phi + \partial_z c\right)
\end{gather}
Note that in the simplified case $D_-=D_+$ the above set reduces to:
\begin{gather}
\partial_z\left(c^{bath}\partial_z\Phi^{bath}\right)=0\\
\partial_t c^{bath} = \frac{D}{D_0} \partial_{zz} c^{bath}
\end{gather}
and we denote $\Phi^{bath}_0=\lim_{z\rightarrow h(t)^+} \Phi^{bath}$, $c^{bath}_0=\lim_{z\rightarrow h(t)^+} c^{bath}$.
\section{Inner Layer.}
The interface between the gel and the bath is moving in eulerian coordinates. This is identified by $z=h(t)$ where:
\begin{equation}
h(t)-1=\int_0^1{C^{gel}_s+C^{gel}_++C^{gel}_-}dZ = \int_0^{h(t)} c^{gel}_s+c^{gel}_++c^{gel}_- dz.
\end{equation}
Taking the time-derivative of the above equation we have:
\begin{equation}
\begin{aligned}
\frac{d h}{dt}=\int_0^{h(t)}\partial_z\left(c^{gel}_s+c^{gel}_++c^{gel}_-\right)dz + \frac{d h}{dt}\sum_m c^{gel}_m(h(t),t)\\
\Rightarrow \frac{d h}{dt}=-\frac{\sum_m c_m v_m}{1-\sum_m c_\alpha}=-\phi_n^{-1}\sum_m c_m v_m=v_n(h(t),t)
\end{aligned}
\end{equation}
where $phi_n$ is the volume fraction of the network. In order to move in the inner layer we now consider the following change of variable:
\begin{equation}
z= h(t)+ \beta \xi
\end{equation}
so that the derivative in time and space become:
\begin{equation}
\frac{\partial}{\partial z} = \beta^{-1}\frac{\partial}{\partial \xi}, \qquad \frac{\partial}{\partial t}= \frac{\partial}{\partial t} - \beta^{-1} \frac{d h}{dt} \frac{\partial}{\partial \xi}
\end{equation}
Before moving to analyse the bath, we need to impose the boundary condition at the interface $h(t)$:
\begin{equation}
\begin{aligned}
\left[\phi\right]^+_-=0, \qquad \left[\mu_i\right]^+_-=0, \qquad \epsilon_r \partial_z \phi^+=\partial_z \phi^-\\
\left[T_z\right]^+_-=0, \qquad \left[c_i\left(v_i-\frac{dh}{dt}\right)\right]^+_-=0, \qquad\left.\partial_z C_s \right|_{h^-(t)}=0.
\end{aligned}
\end{equation}
\subsection{Bath.}
We now study the inner layer  in the bath.
\begin{gather}
\begin{aligned}
\mu_s = p \mathcal{G} + \ln c_s, 
\end{aligned}\\[2.5mm]
\mu_\pm = p \mathcal{G} \pm \Phi z_i + \ln c_\pm ,\\
-\epsilon_r\partial_{\xi\xi} \Phi = c_+-c_-\, \label{Poi2.1}\\
T_z= -p-\frac{\epsilon_r }{2\mathcal{G}} (\partial_z \Phi)^2,\\
\partial_\xi T_z=0\\
\beta \partial_t c_s -\frac{dh}{dt}\frac{\partial c_s}{\partial \xi}= - \partial_\xi j_s,\\
\beta \partial_t c_\pm -\frac{dh}{dt}\frac{\partial c_\pm}{\partial \xi}= -\partial_\xi j_\pm,\\
\beta j_s =-c_s \frac{D^{bath}}{D_0}  \left(\partial_\xi\mu_s +\sum_i \frac{c_i}{c_s} \partial_\xi \mu_i\right)\\
\beta j_\pm= - \frac{D_\pm c_\pm}{D_0}\partial_\xi \mu_\pm + \beta \frac{c_\pm}{c_s}j_s
\end{gather}
so that assuming $\mu_m\sim O(1)$ at the leading order in $\beta$ we obtain:
\begin{gather}
0=\partial_\xi\mu_s +\sum_i \frac{c_i}{c_s} \partial_\xi \mu_i\\
0= - \partial_\xi \mu_\pm 
\end{gather}
which corresponds to constant chemical potential in the layer:
\begin{gather}
\mu^{bath}_s\equiv \lim_{\xi\rightarrow \infty}\mu^{bath}_s= \ln (1-2c_0) \Rightarrow c_s = (1-2c_0)\exp(-p\mathcal{G})\\
\mu^{bath}_+\equiv \lim_{\xi\rightarrow \infty}\mu^{bath}_+= \Phi_0^{bath}+\ln c_0 \Rightarrow c_+ = c_0\exp[-(\Phi-\Phi_0^{bath})-p\mathcal{G}]\\
\mu^{bath}_-\equiv \lim_{\xi\rightarrow \infty}\mu^{bath}_-= -\Phi_0^{bath}+\ln c_0\Rightarrow c_- = c_0\exp[(\Phi-\Phi_0^{bath})-p\mathcal{G}]
\end{gather}
Using the above and Equation~(\ref{Poi2.1}):
\begin{equation}
-\epsilon_r\partial_{\xi\xi} \Phi = 2c_0\exp(-p\mathcal{G}) \sinh(-(\Phi-\Phi_0^{bath})).
\end{equation}
We further have that the stress is constant so that:
\begin{equation}
\begin{aligned}
T_z \equiv\lim_{\xi\rightarrow\infty}T_z=0 \Rightarrow \mathcal{G}\exp(p\mathcal{G})\partial_\xi p= 2c_0\sinh(\Phi-\Phi_0^{bath})\partial_\xi \Phi\\
\Rightarrow \exp(p\mathcal{G})-2c_0\cosh(\Phi-\Phi_0^{bath})\equiv const\\
\lim_{\xi\rightarrow\infty}\exp(p\mathcal{G})-2c_0\cosh(\Phi-\Phi_0^{bath}) =1-2c_0\\
 \Rightarrow p = \mathcal{G}^{-1}\ln\left[1+2c_0\left(\cosh(\Phi-\Phi_0^{bath})-1\right)\right] 
\end{aligned}
\end{equation}
so that:
\begin{equation}
-\epsilon_r\partial_{\xi\xi} \Phi =-2c_0 \frac{\sinh((\Phi-\Phi_0^{bath}))}{1+2c_0\left(\cosh(\Phi-\Phi_0^{bath})-1\right)}
\end{equation}
Finally we have a condition on the fluxes:
\begin{eqnarray}
\begin{aligned}
\frac{dh}{dt}\frac{\partial c^{bath}_s}{\partial \xi}=  \partial_\xi j_s \Rightarrow c^{bath}_s \frac{dh}{dt}-j_s=\alpha_s(t),\\
\lim_{\xi\rightarrow\infty} \alpha_s(t) = (1-2c_0)\frac{dh}{dt} \Rightarrow  c^{bath}_s \left(v_s-\frac{dh}{dt}\right)=(1-2c_0)\frac{dh}{dt}
\end{aligned}\\[4mm]
\begin{aligned}
\frac{dh}{dt}\frac{\partial c^{bath}_+}{\partial \xi}= \partial_\xi j_+\Rightarrow c^{bath}_+ \frac{dh}{dt}-j_+=\alpha_{+}(t)\\
\lim_{\xi\rightarrow\infty} \alpha_+(t) = c_0 \frac{dh}{dt}  +\frac{D_+}{D_0}\left(c_0 \partial_z \Phi_0+\partial_z c_0\right) \\
\Rightarrow  c^{bath}_+ \left(v_+-\frac{dh}{dt}\right)=c_0 \frac{dh}{dt}  +\frac{D_+}{D_0}\left(c_0 \partial_z \Phi_0+\partial_z c_0\right)
\end{aligned}\\[4mm]
\begin{aligned}
\frac{dh}{dt}\frac{\partial c^{bath}_-}{\partial \xi}= \partial_\xi j_-\Rightarrow c^{bath}_- \frac{dh}{dt}-j_-=\alpha_{-}(t)\\
\lim_{\xi\rightarrow\infty} \alpha_-(t) = c_0 \frac{dh}{dt}  +\frac{D_-}{D_0}\left(-c_0 \partial_z \Phi_0+\partial_z c_0\right) \\
\Rightarrow  c^{bath}_- \left(v_--\frac{dh}{dt}\right)=c_0 \frac{dh}{dt}  +\frac{D_-}{D_0}\left(-c_0 \partial_z \Phi_0+\partial_z c_0\right)
\end{aligned}\\
\end{eqnarray}
Consequently the set of equation governing the inner layer in the bath are:
\begin{gather}
-\epsilon_r\partial_{\xi\xi} \Phi^{bath} = 2c_0\exp(-p^{bath}\mathcal{G}) \sinh(\Phi^{bath}-\Phi_0^{bath})\\
p^{bath} = \mathcal{G}^{-1}\ln\left[1+2c_0\left(\cosh(\Phi-\Phi_0^{bath})-1\right)\right] \\
c^{bath}_+ = c_0\exp(-p^{bath}\mathcal{G})\exp[-(\Phi^{bath}-\Phi_0^{bath})],\\
c^{bath}_- = c_0\exp(-p^{bath}\mathcal{G}) \exp[(\Phi^{bath}-\Phi_0^{bath})],\\
 c^{bath}_s = (1-2c_0)\exp(-p^{bath}\mathcal{G})\\
j_s=c^{bath}_s\frac{dh}{dt}+(1-2c_0)\frac{dh}{dt},\\
j_\pm=c^{bath}_\pm\frac{dh}{dt}+ c_0 \frac{dh}{dt}  +\frac{D_\pm}{D_0}\left(\pm c_0 \partial_z \Phi_0+\partial_z c_0\right).
\end{gather}
\subsection{Inner Gel.}
Considering instead the gel:
\begin{gather}
\begin{aligned}
\mu_s = p \mathcal{G} - \frac{\omega^2}{\beta^2} J \left[f(C_s)\partial_{\xi\xi} C_s+\frac{f'(C_s)}{2}\left(\partial_\xi C_s\right)^2\right]\\
+ \left[\ln \frac{C_s}{1+C_s} + \frac{1}{1+C_s}+\frac{\chi}{(1+C_s)^2} + \ln \frac{C_s}{J-1} \right], 
\end{aligned}\label{mus2.2}\\[2.5mm]
\mu_\pm = p \mathcal{G} \pm \Phi  + \ln \frac{C_\pm}{J-1} ,\\
-\partial_{\xi\xi} \Phi = c_+-c_-+c_f\,\label{Poi2.2} \\
T_z= -p-\frac{\omega^2 f(C_s)}{2\beta^2\mathcal{G}} (\partial_\xi C_s)^2+ J^{-1}\left(J-1\right)-\frac{1}{2\mathcal{G}} (\partial_\xi \Phi)^2,\label{T2.2}\\
\partial_\xi T_z=0\\
\beta \partial_t c_s -\frac{dh}{dt}\frac{\partial c_s}{\partial \xi}+\partial_\xi (c_s v_n)= - \partial_\xi j_s,\label{gov2.2}\\
\beta \partial_t c_\pm -\frac{dh}{dt}\frac{\partial c_\pm}{\partial \xi}+\partial_\xi (c_\pm v_n)= -\partial_\xi j_\pm,\label{gov2.2_2}\\
\beta j_s =-c_s K  \left(\partial_\xi\mu_s +\sum_i \frac{c_i}{c_s} \partial_\xi \mu_i\right)\\
\beta j_\pm= - \frac{D_\pm c_\pm}{D_0}\partial_\xi \mu_\pm + \beta \frac{c_\pm}{c_s}j_s
\end{gather}
Summing Equations~(\ref{gov2.2})-(\ref{gov2.2_2}), we have:
\begin{equation}
\begin{aligned}
-\frac{dh}{dt}\frac{\partial \sum_m c_m}{\partial \xi}+ \partial_\xi \left(\sum_m c_m v_m\right)= \frac{dh}{dt}\frac{\partial c_n}{\partial \xi}-\partial_\xi (c_n v_n)\\
\Rightarrow  \frac{dh}{dt} c_n -c_n v_n=const(t)=0 \Rightarrow  \frac{dh}{dt} \equiv v_n
\end{aligned}
\end{equation}
so that in the layer the velocity of the network remain constant in space.
As before, we consider the chemical potential and the fluxes to be $O(1)$. Under the assumption that:
\begin{equation}
\omega^2 f(C_s)>> O(\beta) \text{ and } \omega^2 f'(C_s)>> O(\beta)
\end{equation}
we have that the Debye layer remains the smaller scale of the problem. Let us consider the following expansion of the solution:
\begin{equation}
C_s(\xi,t;\beta) = \alpha_0(\xi,t) + \beta \alpha_1(\xi,t) + \beta^2 \alpha_2(\xi,t)+ h.o.t\,. 
\end{equation}
Substituting the above into~(\ref{mus2.2}), we obtain:
\begin{eqnarray}
&O(\beta^{-2})& \ \omega^2 J_0\left(f(\alpha_0)\partial_{\xi\xi}\alpha_0+ \frac{f'(\alpha_0)}{2}(\partial_\xi \alpha_0)^2\right)=0\label{cond1}\\
&O(\beta^{-1})& \ \begin{aligned}
f(\alpha_0)\partial_{\xi\xi}\alpha_1+ f'(\alpha_0)\left(\alpha_1\partial_{\xi\xi}\alpha_0+\partial_\xi \alpha_0\partial_\xi \alpha_1\right)
\\+f''(\alpha_0)\alpha_1(\partial_\xi \alpha_0)^2=0 \label{cond2}
\end{aligned}
\end{eqnarray} 
with the boundary condition $\partial_\xi \alpha_0=\partial_\xi \alpha_1=0$. 
As we are assuming that $f(\alpha_0)>0$, then (\ref{cond1}) reduces to:
\begin{equation}
\partial_{\xi}\left[ \sqrt{f(\alpha_0)}\partial_\xi \alpha_0\right]=0 \Rightarrow \ \partial_\xi \alpha_0\equiv 0\  \Rightarrow \ \alpha_0= a_0(t).
\end{equation} 
Using now Equation~(\ref{cond2}), we obtain:
\begin{equation}
f(\alpha_0)\partial_{\xi\xi}\alpha_1 = 0  \Rightarrow \alpha_1 \equiv a_1(t).
\end{equation} 
so that in the Debye layer $C_s= a_0(t) + \beta a_1(t) + O(\beta^2)$.
We can now impose the continuity of the chemical potential:
\begin{gather}
\mu_s \equiv const = \mu^{bath}_s= \ln(1-2c_0),\\
\begin{aligned}
\mu_+ &\equiv const = \mu^{bath}_+=  \Phi_0^{bath} + \ln c_0 \\&\Rightarrow c_+=\frac{(J-1)}{J}c_0\exp[-(\Phi-\Phi_0^{bath})-p\mathcal{G}]
\end{aligned}\\
\begin{aligned}
\mu_- &\equiv const = \mu^{bath}_-= - \Phi_0^{bath} + \ln c_0\\ &\Rightarrow c_-=\frac{(J-1)}{J}c_0\exp[(\Phi-\Phi_0^{bath})-p\mathcal{G}]
\end{aligned}
\end{gather}
We also have that the stress tensor is constant so that:
\begin{equation}
T_z\equiv T^{bath}_z=0 \Rightarrow p = \frac{J-1}{J}-\frac{1}{2\mathcal{G}} (\partial_\xi \Phi)^2.
\end{equation}
Using now Equations~(\ref{gov2.2})-(\ref{gov2.2_2}), we obtain:
\begin{equation}
c_s v_s - c_s \frac{dh}{dt}=A_s(t), \qquad c_\pm v_\pm - c_\pm \frac{dh}{dt}=A_\pm(t)
\end{equation}
and imposing the boundary condition at the interface:
\begin{gather}
c_s v_s - c_s \frac{dh}{dt}=(1-2c_0)\frac{dh}{dt}\\
c_\pm v_\pm - c_\pm \frac{dh}{dt}=c_0 \frac{dh}{dt}  +\frac{D_\pm}{D_0}\left(\pm c_0 \partial_z \Phi^{bath}_0+\partial_z c_0\right).
\end{gather}

\subsection{Matching with the outer gel}
We now use the following notation:
\begin{equation}
\begin{aligned}
\Phi^{gel}_0=\lim_{z\rightarrow h(t)^-} \Phi^{gel}, \qquad c^{gel}_{m,0}=\lim_{z\rightarrow h(t)^-} c^{gel}_{m} \qquad p_0^{gel}=\lim_{z\rightarrow h(t)^-} p^{gel},\\
J^{gel}_0=\lim_{z\rightarrow h(t)^-} J^{gel}, \qquad j^{gel}_{m,0}=\lim_{z\rightarrow h(t)^-} j^{gel}_m
\end{aligned}
\end{equation}
Matching also with the exterior solution in the bulk we have that:
\begin{gather}
\begin{aligned}
&\lim_{\xi\rightarrow -\infty} T_z = T^{gel}_z =0 \\
p^{gel}_0& = \frac{J^{gel}_0-1}{J^{gel}_0}\label{eq1}
\end{aligned}\\
\lim_{\xi\rightarrow -\infty} c_\pm = c^{gel}_{\pm,0} = \frac{(J^{gel}_0-1)}{J^{gel}_0}c_0\exp[\mp(\Phi^{gel}_0-\Phi_0^{bath})-p_0^{gel}\mathcal{G}].
\end{gather}
\begin{gather}
\lim_{\xi\rightarrow -\infty} c_s v_s - c_s \frac{dh}{dt}= j^{gel}_{s,0} =(1-2c_0)\frac{dh}{dt}\\
\begin{aligned}
\lim_{\xi\rightarrow -\infty}c_\pm v_\pm - c_\pm \frac{dh}{dt}=j^{gel}_{\pm,0}\\
=c_0 \frac{dh}{dt}  +\frac{D_\pm}{D_0}\left(\pm c_0 \partial_z \Phi^{bath}_0+\partial_z c_0\right).
\end{aligned}
\end{gather}
As $j^{gel}_{+,0}=j^{gel}_{-,0}$ in the gel we have that:
\begin{eqnarray}
0= - \left(\frac{D_++D_-}{D_0}\right)c_0\partial_z \Phi^{bath}_0 + \left(\frac{D_--D_+}{D_0}\right)\partial_z c_0\label{A}\\
2 \frac{d h}{dt} =- \left(\frac{D_+-D_-}{D_0}\right)c_0\partial_z \Phi^{bath}_0 - \left(\frac{D_-+D_+}{D_0}\right)\partial_z c_0\label{B}\\
\frac{d h}{dt} = \frac{j_{s,0}^{gel}}{1-2c_0},
\end{eqnarray}
where the second is obtained summing all of the above and using the saturation condition plus $c_nv_n=-\sum c_mv_m$. 
Since in the outer layer electro-neutrality must hold:
\begin{equation}
c^{gel}_{+,0}-c^{gel}_{-,0}+c_{f,0} =0 \Rightarrow  \sinh(\Phi_0^{gel}-\Phi_0^{bath})= \frac{C_f}{2c_0 (J_0^{gel}-1)}\exp(p_0^{gel}\mathcal{G})\label{eq2}
\end{equation}
By definition of $J^{gel}_0=1+C^{gel}_{s,0}+C^{gel}_{+,0}+C^{gel}_{-,0}$ we also have that:
\begin{equation}
J^{gel}_0=1+C^{gel}_{s,0}+ 2c_0\exp(-p_0^{gel}\mathcal{G})\cosh[(\Phi^{gel}_0-\Phi_0^{bath})]\label{eq3}.
\end{equation}
Given the inner expansion of $C_s$ we have that:
\begin{equation}
\partial_\xi C_s = O(\beta^2) \Rightarrow \partial_z C^{gel}_{s} = O(\beta)
\end{equation}
which requires that at the leading order $\partial_z C^{gel}_{s,0} = 0$.
Consequently matching also the chemical potential of the solvent $\mu_s$:
\begin{equation}
\begin{aligned}
\omega^2 J^0_{gel} f(C^{gel}_{s,0})\partial_{zz}^2 C^{gel}_{s,0}= \frac{J^0_{gel}-1}{J^0_{gel}} \mathcal{G}  + \ln \frac{C^{gel}_{s,0}}{1+C^{gel}_{s,0}} +\\ \frac{1}{1+C^{gel}_{s,0}}+\frac{\chi}{(1+C^{gel}_{s,0})^2} + \ln \frac{C^{gel}_{s,0}}{(J^0_{gel}-1)(1-2c_0)}, 
\end{aligned}\\[2.5mm]\label{eq4}
\end{equation}

Given that we have for unknowns $\Phi^{gel}_0$, $p^{gel}_0$, $C^{gel}_{s,0}$ and $J^{gel}_0$, we can solve for them using Equations~(\ref{eq1})-(\ref{eq2})-(\ref{eq3})-(\ref{eq4}). 
To summarise for the bath we have the following boundary condition at the interface:
\begin{gather}
\sinh(\Phi_0^{gel}-\Phi_0^{bath})= \frac{C_f}{2c_0 (J_0^{gel}-1)}\exp\left(\frac{J^{gel}_0-1}{J^{gel}_0}\mathcal{G}\right)\\
J^{gel}_0=1+C^{gel}_{s,0}+ 2c_0\sqrt{\exp\left(-2\frac{J^{gel}_0-1}{J^{gel}_0}\mathcal{G}\right)+\frac{C^2_f}{4c^2_0 (J_0^{gel}-1)^2}}\\[4mm]
\partial_z C^{gel}_{s,0}=0,\\[5mm]
c^{gel}_{+,0} = \frac{(J^{gel}_0-1)}{J^{gel}_0}c_0\exp[-(\Phi^{gel}_0-\Phi_0^{bath})-\frac{(J^{gel}_0-1)}{J^{gel}_0}\mathcal{G}],\\
\begin{aligned}
\omega^2 J^0_{gel} f(C^{gel}_{s,0})\partial_{zz}^2 C^{gel}_{s,0}= \frac{J^0_{gel}-1}{J^0_{gel}} \mathcal{G}  + \ln \frac{C^{gel}_{s,0}}{1+C^{gel}_{s,0}} +\\ \frac{1}{1+C^{gel}_{s,0}}+\frac{\chi}{(1+C^{gel}_{s,0})^2} + \ln \frac{C^{gel}_{s,0}}{(J^0_{gel}-1)(1-2c_0)}, 
\end{aligned}
\end{gather}
and the governing equations:
\begin{gather}
\begin{aligned}
\mu_s = p \mathcal{G}  - \omega^2 J \left[f(C_s)\partial_{zz}^2 C_s+\frac{f'(C_s)}{2}\left(\partial_z C_s\right)^2\right]\\
+ \left[\ln \frac{C_s}{1+C_s} + \frac{1}{1+C_s}+\frac{\chi}{(1+C_s)^2} + \ln \frac{C_s}{J-1} \right], 
\end{aligned}\\[2.5mm]
\mu_\pm = p \mathcal{G} \pm \Phi + \ln \frac{C_\pm}{J-1} ,\\
p=-\frac{\omega^2 f(C_s)}{2\mathcal{G}} (\partial_z C_s)^2+ \frac{J-1}{J},\\
C_- = C_+ + z_f C_f,\\
j_s =-c_sK  \left(\partial_z\mu_s +\sum_i \frac{c_i}{c_s} \partial_z \mu_i\right)\\
j_+= - \frac{D_+ c_+}{D_0}\partial_z \mu_+ + \frac{c_+}{c_s}j_s\\
\partial_t c_s + \partial_z (c_s v_n)=- \partial_z j_s,\\
\partial_t c_+ +\partial_z (c_+ v_n)= -\partial_zj_+,\\
\partial_z \Phi = \frac{D_0 j_+}{D_-c_-}- \frac{D_0 j_s}{D_-c_s}j_s + \frac{\partial C_-}{C_-}-\frac{\partial J}{J-1},
\end{gather}
with the additional boundary condition at the surface:
\begin{equation}
j_+= j_s = 0.
\end{equation}
Finally $c_0$ and $\Phi^{bath}_0$ can be determined using the outer problem in the bath and the matching condition ~(\ref{A})-(\ref{B}):
\begin{gather}
c^{bath} \frac{D_-+D_+}{D_0} \partial_z \Phi^{bath} + \frac{D_--D_+}{D_0} \partial_z c^{bath}=0\\
\partial_t c^{bath} = -\partial_z\left[\frac{D_+}{D_0} c \partial_z \Phi +\frac{D_+}{D_0} \partial_z c\right]\\
0= - \left(\frac{D_++D_-}{D_0}\right)c\partial_z \Phi^{bath} + \left(\frac{D_--D_+}{D_0}\right)\partial_z c \qquad \qquad z=h(t)\\
\frac{2j_{s,0}^{gel}}{1-2c} =- \left(\frac{D_+-D_-}{D_0}\right)c\partial_z \Phi^{bath} - \left(\frac{D_-+D_+}{D_0}\right)\partial_z c \quad z=h(t)\\
+ \text{boundary condition at the edge of the bath/far field}
\end{gather}
If we assume that $D_+=D_-=D$ then:
\begin{gather}
\partial_z \Phi^{bath}=0 \Rightarrow \Phi^{bath}=\Phi^{bath}(\infty)=0\\
\partial_t c_0 = -\frac{D}{D_0} \partial_{zz}c_0\\
j_{s,0}^{gel}=-(1-2c_0)\frac{D}{D_0} \partial_z c_0 \quad z=h(t)\\
j_{\pm}^{gel}=-\frac{D}{D_0}c_\pm \partial_z\left(p\mathcal{G}\pm\Phi-\ln(J-1)\right)-\frac{D}{D_0}\partial_z C_\pm+\frac{c_\pm}{c_s}j_s\\ 
j_+=j_- \Rightarrow  \partial_z \Phi = \frac{c_f}{c_++c_-} \left(\mathcal{G}p-\frac{\partial_z J}{J-1}-\frac{D_0}{Dc_s}j_s\right)\\
\begin{aligned}
j_s = - \frac{Kc_s D(2c_++c_f)}{Kc_f^2D_0+Dc_s(2c_++c_f)}\left[\left(c_s+\frac{4c_+(c_++c_f)}{2c_++c_f}\right)\mathcal{G}\partial_z p\right.\\
\left.+\frac{c_f^2}{2c_++c_f}\frac{\partial_z J}{J-1}-c_s \omega^2 \partial_z(J\partial_{zz}C_s)+\frac{1+C_s-2\chi C_s}{C_s(1+C_s)^3}\partial_z C_s\right]
\end{aligned}
\end{gather}
If we now consider the boundary:
\begin{equation}
\begin{aligned}
j^{gel}_{s,0} = - \frac{2K Dc_s(2c_++c_f)}{Kc_f^2D_0+Dc_s(2c_++c_f)}\left\{\left[\left(-C_s+\frac{4c_+(c_++c_f)}{2c_++c_f}\right)\mathcal{G}\right.\right.\\
\left.\left.+\frac{C_f^2}{(2c_++c_f)(J-1)}-C_sf(C_s)\right.\right.\\
\left.\left.+C_ s\ln\left((1-2c_0)(J-1)\right)\right]\frac{\partial_z C_+}{J^2}-\omega^2J\partial_{zzz}C_s\right\}
\end{aligned}
\end{equation}
so that the boundary condition for $c_0$ is:
\begin{equation}
\begin{aligned}
\partial_z c_0= \frac{2K c_s\frac{(2c_++c_f)}{1-2c_0}}{Kc_f^2+D/D_0c_s(2c_++c_f)}\left\{\left[\left(-C_s+\frac{4c_+(c_++c_f)}{2c_++c_f}\right)\mathcal{G}\right.\right.\\
\left.\left.+\frac{C_f^2}{(2c_++c_f)(J-1)}-C_sf(C_s)\right.\right.\\
\left.\left.+C_s\ln\left((1-2c_0)(J-1)\right)\right]\frac{\partial_z C_+}{J^2}-\omega^2J\partial_{zzz}C_s\right\}
\end{aligned}
\end{equation}
\end{document}