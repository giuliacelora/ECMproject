\documentclass[12pt]{extarticle}
\usepackage[utf8]{inputenc}
\usepackage[utf8]{inputenc}
\usepackage{tikz}
\usetikzlibrary{er,positioning,bayesnet}
\usepackage{multicol}
\usepackage{xcolor}
\usepackage{algpseudocode,algorithm,algorithmicx}
\usepackage{hyperref}
\usepackage[inline]{enumitem} % Horizontal lists
\usepackage{cite}
\usepackage{amsmath}
\usepackage{amssymb}
\usepackage{graphicx}
\usepackage{multirow}
\usepackage{booktabs}
\usepackage[labelfont=bf,font=scriptsize]{caption}
\usepackage[font=scriptsize]{subcaption}
\usepackage{mhchem}
\newcommand{\indentitem}{\setlength\itemindent{20pt}}
\newcommand{\F}{\ensuremath{\mathbb{F}}}
\newcommand{\B}{\ensuremath{\mathbb{B}}}
\newcommand{\LL}{\ensuremath{\mathbb{L}}}
\renewcommand{\d}{\ensuremath{\text{d}}}
\title{Polyelectrolytes Model Derivation}
\author{}

\begin{document}
\section{Polyelectrolytes Gel Model Derivation.}
\label{modeldev}
\subsection{Conservation Law.}
\label{conslaw}
We here consider the gel as a three-phase medium composed of a solid polymer network with fixed charges, a solvent (i.e. water molecules, interstitial fluid) and solutes (freely moving charges). 
\begin{figure}[h!]
	\centering
	\def\svgwidth{0.9\linewidth}
	\input{images/deformation.pdf_tex}
	\caption{Sketch of the dry and current state of the gel.}
	\label{Above}
\end{figure}

As the tissue deforms, the material element originally located at $\mathbf{X}$ in the initial configuration $\mathcal{B}_0$ is displaced to the point $\mathbf{x}$ in the current configuration $\mathcal{B}_t$, see Figure \ref{Above}. Such a transformation is described by the deformation gradient tensor $\F= \partial \mathbf{x}/\partial \mathbf{X}$; the information about the change in volume is encoded in $J= \det \F$, while $\mathbf{u}= \mathbf{x}-\mathbf{X}$ is the displacement vector. As in \cite{sarah}, we consider the reference, or initial, state $\mathcal{B}_0$, which is stress free, to be equivalent to the dry state of the gel, i.e. only solid phase present. Since we assume the solid phase to be incompressible, any change in the volume can only be related to the migration of solvent and solute molecules, whose nominal \footnote{variable value in the reference configuration.} concentrations will be denoted by $C_s$ and $C_i$ respectively, $i=1,\ldots,N$ with $N$ being the number of free ion species. This lead to the molecular incompressibility condition:

\begin{equation}
 J= 1 + v_s C_s +\sum\limits_{i=1}^{N} v_i C_i
 \label{comp}
\end{equation}
where $v_m$ are the characteristic molecular volume of each species in the solution. When considering the interstitial fluid, the contribution of ions to the volume can be neglected \cite{ecm1,ecm2} so that Equation~(\ref{comp}) reduces to:

\begin{equation}
J=1+v_s C_s.
\label{inc}
\end{equation} 

Consequently, the volume fractions of fluid $\phi_f$ and solid $\phi_n$ phases in the swollen gel are defined as:
\begin{equation}
\phi_f = \frac{v_sC_s}{1+v_sC_s}, \qquad \phi_n = \frac{1}{1+v_sC_s}.
\end{equation}
where again we are neglecting the contribution of ions to the total volume.
While $C_m$ denotes the number of each molecule per unit volume in the initial configuration for the $m$-th species in the solution, the actual concentration in the current state is denoted by $c_m=C_m/J$. Throughout the derivation of the model, we will be using the index $i=1,\ldots,N$ to denote the ionic species only, while $m\in\left\{s,1,\ldots,N\right\}$ refers to all mobile species, i.e. both the solvent and solutes.

Mass conservation must apply to all mobile species and in the initial configuration this reads:
\begin{equation}
\dot{C}_m + \nabla_0 \cdot \mathbf{J}_m = 0, \label{consmass}
\end{equation}
where $\mathbf{J}_m$ is the nominal flux per unit area in the dry state, $\dot{C}_m$ is the derivative of $C_m$ with respect to time, i.e.  $\dot{C}_m\equiv\partial_t C_m$ and $\nabla_0$ denotes the gradient in the Lagrangian coordinates $\mathbf{X}$. Their counterparts in the actual configuration are denoted by $\mathbf{j}_m$ and $\nabla$ and are defined according to the following rules:
\begin{equation}
\mathbf{J}_m = J \F^{-1} \mathbf{j}_m, \qquad \nabla_0 (\cdot) = \F^{T} \nabla(\cdot).
\end{equation}

When considering tissues or hydrogels, inertial and gravitational effects are commonly neglected, so that the conservation of momentum for the gel reads:

\begin{gather}
\nabla_0 \cdot \mathbb{S}=0\label{consmom},
\end{gather}

where $\mathbb{S}$ is the first Piola-Kirchoff tensor, which represents the stress state of the gel in the initial configuration. The counterpart in the current configuration is the Cauchy stress tensor $\mathbb{T}$, which is related to $\mathbb{S}$ as follows:

\begin{equation}
\mathbb{T} = J^{-1}\mathbb{S}\F^T.\label{TS}
\end{equation}

The presence of free moving ions generates an electric field which is denoted by $\mathbf{E}$ and $\mathbf{e}$ in the initial and current configuration respectively. Introducing the electrostatic potential $\Phi$, we have that:
\begin{equation}
\mathbf{E}= -\nabla_0 \, \Phi, \hspace{8mm} \mathbf{e}= - \nabla \, \Phi.
\label{Phi}
\end{equation}

As in \cite{Reviewpolyel}, we consider the gel to be a dielectric material\footnote{a material that does not conduct electricity but can be polarized in the presence of an electric field.}. Consequently, the presence of the electric field generates an electric displacement $\mathbf{H}$\footnote{the vector field that accounts for both the electric field and the polarization of the dielectric material.}, which must obey Gauss law of electrostatics:
\begin{equation}
\nabla_0 \cdot \mathbf{H}= Q,
\label{gauss}
\end{equation}
where $Q$ is the local total charge, which accounts for both fixed and moving charges:
\begin{equation}
Q = e\left(\sum\limits_{i} z_i C_i+z_f C_{f}\right)\, , 
\end{equation}
where $e$ is the elementary charge, $C_f$ is the concentration of fix charges and $z_m$ is the valence of the corresponding charged species. As for above, we can move from nominal quantities to the corresponding value in the current configuration by applying the following rules:

\begin{eqnarray}
\mathbf{H} = J \F^{-1}\mathbf{h},\\
\mathbf{E} = \F^T \mathbf{e},
\end{eqnarray}
where $\mathbf{h}$ is the electric displacement in the current configuration.
\subsection{Energy Balance Inequality.}
\label{sec_ine}

As mentioned in Section \ref{secNET}, in a thermodynamically consistent model, the free energy $\psi$ can not be chosen arbitrarily but needs to satisfy the energy imbalance inequality,~(\ref{energyin}). In this section, we focus on the right-hand side of the inequality which specify how the system exchanges energy and mass with the environment. Considering a control volume $R$ in the reference configuration $\mathcal{B}_0$, the system exchanges mass due to the diffusion of each mobile species, so that $M(R)$ is given by:
\begin{equation}
M(R)= \sum\limits_{m=s,1,\ldots,N} - \int_{\partial R} \mu_m \,\mathbf{J}_m \cdot \mathbf{n} 
\end{equation}
where $\mathbf{n}$ is the unit normal vector to the surface $\partial R$ and $\mu_m$ is the chemical potential associated with each species. Widely used in the thermodynamics of mixture, the chemical potential is a measure of the rate of change in free energy associated with adding one more molecule to a unit volume.

The term $W(R)$, i.e. the rate of work done on the system, is instead decomposed in two contributions, the rate of electrical $W_{el}(R)$ and mechanical work $W_{mec}(R)$. Following \cite{DROZDOVph}, $W_{el}(R)$ is defined as:
\begin{equation}
W_{el}(R) = -\int_{\partial R} \Phi\, \dot{\mathbf{H}}\cdot \mathbf{n}
\end{equation}

Following the work of Gurtin \cite{GURTIN}, we account both for the presence of macro-stresses $\mathbb{S}$ and micro-stresses $\boldsymbol{\xi}$, which arise due to the system heterogeneity \cite{microstress}. As before we only consider the dominant contribution of the solvent while neglecting the solute, so that $W_{mec}(R)$ reads:
\begin{equation}	\centering
%	\begin{subfigure}{0.32\textwidth}
%		\centering
%		\Large
%	\def\svgwidth{0.95\linewidth}
%	\input{latex/images/modelA1.pdf_tex}
%	\caption{Rheological Model A}
%	\label{fig1A}
%	\end{subfigure}
W_{mec}(R) = \int_{\partial R} \left(\boldsymbol{\xi}\cdot \mathbf{n}\right)\dot{C}_s + \int_{\partial R} \mathbb{S}\mathbf{n} \cdot \dot{\mathbf{u}}
\end{equation}
where $\mathbf{u}= \mathbf{x}-\mathbf{X}$ is the displacement vector, which is related to the deformation tensor by $\F=\mathbb{I}-\nabla_0 \mathbf{u}$. Substituting this result back into the formula~(\ref{energyin}) and applying the divergence theorem we obtain the following inequality:
\begin{equation}
\int_R \dot{\psi} - \mathbf{E}\cdot \dot{\mathbf{H}} \, + \, \sum\limits_{i=1}^{N} \left[e \Phi  z_i \dot{C}_i+ \nabla_0 \left(\mu_i \mathbf{J}_i \right)\right] + \nabla_0 (\mu_s \mathbf{J}_s- \boldsymbol{\xi}\dot{C}_s -\mathbb{S}^T\mathbf{\dot{u}}) \leq 0 
\end{equation}

Since this must hold for any choice of the volume $R$, the inequality must hold also locally:
\begin{equation}
\dot{\psi} - \mathbf{E}\cdot \dot{\mathbf{H}} \, + \, \sum\limits_{i=1}^{N} \left[e \Phi  z_i \dot{C}_i+ \nabla_0 \left(\mu_i \mathbf{J}_i \right)\right] + \nabla_0 (\mu_s \mathbf{J}_s- \boldsymbol{\xi}\dot{C}_s -\mathbb{S}^T\mathbf{\dot{u}}) \leq 0. 
\end{equation}
Further accounting for Equations~(\ref{consmass})-(\ref{consmom}), we obtain that:
\begin{equation}
\begin{aligned}
\dot{\psi} - \mathbf{E}\cdot \dot{\mathbf{H}} \, + \, \sum\limits_{i=1}^{N} \left[e \Phi  z_i - \mu_i\right] \dot{C}_i - (\mu_s + \nabla_0 \cdot \boldsymbol{\xi})\,\dot{C}_s -\mathbb{S}:\dot{\F}\\
-\boldsymbol{\xi} \cdot \nabla_0 \, \dot{C}_s + \sum\limits_{m} \nabla_0 \, \mu_m \cdot \mathbf{J}_m \leq 0.
\label{temp2}
\end{aligned} 
\end{equation}

As exhaustively discussed in previous studies \cite{Plasto,GURTIN}, the energy inequality imposes restrictions on the constitutive equation of the free energy $\psi$. Adapting their results to our specific problem, we have that:
\begin{equation}
\psi = \psi (\F,\F_e, C_s, C_i, \nabla_0 \,C_s,\mathbf{H}), \label{temp1}
\end{equation}
which precludes any explicit dependency of $\psi$ on the chemical potential or the viscous deformation gradient $\F_v$. By differentiating the incompressibility condition~(\ref{inc}) and~(\ref{Jv}), we obtain:

\begin{gather}
v_s\dot{C_s} - J \F^{-T}:\dot{\F} =0, \label{temp3}\\
\mathbb{I}:\LL_v=0. \label{temp4}
\end{gather}

If we now substitute~(\ref{temp1}) into~(\ref{temp2}), and include the constraint~(\ref{temp3})-(\ref{temp4}) using as Lagrange multipliers $p$ and $p_v$ respectively, we are left with the augmented form of the energy imbalance inequality:
\begin{equation}
\begin{aligned}
\color{blue}{\left(\frac{\partial \psi}{\partial \nabla_0 C_s}-\boldsymbol{\xi}\right)} \color{black}\cdot \nabla_0 \dot{C}_s + \color{blue}{\left(\frac{\partial \psi}{\partial C_s}-\mu_s-\nabla_0 \cdot \boldsymbol{\xi}+p v\right)}\color{black}\dot{C}_s\\
+ \sum_i\color{blue}\left(\frac{\partial \psi}{\partial C_i} + e\Phi z_i-\mu_i\right) \color{black}\dot{C}_i +\color{blue}\left(\frac{\partial \psi}{\partial \mathbf{H}}-\mathbf{E}\right) \cdot \color{black}\dot{\mathbf{H}}\\
+ \color{blue} \left(\frac{\partial \psi}{\partial \F} + \frac{\partial \psi}{\partial\F_e}\F_v^{-1}- \mathbb{S} - p J \F^{-T}\right): \color{black}\dot{\F}+ \sum_m \nabla_0 \,\mu_m \cdot \mathbf{J}_m \\
- \left(\F_e^T\frac{\partial \psi}{\partial \F_e}-p_v\mathbb{I}\right):\mathbb{L}_v\leq 0 . \label{ineq}
\end{aligned}
\end{equation}

Note that in deriving~(\ref{ineq}), we have also made us of the following identity:
\begin{equation}
\dot{\F}=\dot{\F}_e\F_v+\F_e\dot{\F}_v \Longrightarrow \dot{\F}_e=\dot{\F}\F_v^{-1}-\F_e \LL_v.
\end{equation}
\subsection{Construction of the Free Energy.}

Having the general form of $\psi$, Equation~(\ref{temp1}), it remains to construct its precise form. Following a standard approach in $\psi$-depending modeling, we assume that the total free energy can be additively decomposed with each physical mechanisms contributing independently. We here consider six distinct contributions:

\begin{enumerate}
	{\indentitem\item[\textbullet] the energy of the electric field $\psi_1$;}
	{\indentitem \item[\textbullet] the energy of solvent and solutes' molecules not interacting with the solid phase $\psi_2$;}
	{\indentitem\item[\textbullet] the energy of mixing the solid phase with the solution, $\psi_3$;}
	{\indentitem\item[\textbullet] the energy of mixing the solvent with the solutes in solution, $\psi_4$;}
	{\indentitem\item[\textbullet] the interfacial energy between dissimilar phases, $\psi_5$;}
	{\indentitem\item[\textbullet] the energy of the solid phase not interacting with the solution, $\psi_6$.}
\end{enumerate}

Assuming the solid phase to be an ideal and linear dielectric material, with constant permittivity $\epsilon$,the free energy of polarization reads \cite{DROZDOV+,Reviewpolyel}:
\begin{gather}
\psi_1 = \frac{1}{2\epsilon J} \mathbf{H}\F^T \cdot \F \mathbf{H}.
\end{gather}

The specific energy density $\psi_2$ has the standard form:
\begin{equation}
\psi_2 = \sum\limits_{m} \mu^0_m C_m
\end{equation} 
where $\mu^0_m$ denotes the chemical potential of non interacting solvent and ions molecules. According to Flory-Huggins theory \cite{flory,hug} of mixtures, the mixing energy is given by:
\begin{equation}
\psi_3 = \frac{k_B T J}{v_s} \left(\phi_f \ln \phi_f + \chi \phi_f \phi_n\right),\label{mix}
\end{equation}
where $k_B$ is the Boltzmann's constant, $T$ is the temperature and $\chi$ is the Flory-Huggins parameter, which is a measure of the enthalpy of mixing. Different is the approach of Xue et al. in \cite{ecm1,ecm2}. In these studies, the authors assume only the mixing of GAGs with solvent, while neglecting the collagen. Since we are considering GAGs and the collagen network as as a unique solid phase and we could not find any evidence that collagen does not mix with water, we have chosen the more general form~(\ref{mix}).

As the interstitial fluid is well approximated by a dilute solution, the contribution $\psi_4$ reads \cite{Reviewpolyel,ecm1,ecm2}:

\begin{equation}
\psi_4 = k_B T \sum\limits_{i=1}^{N} C_i \left(\ln \frac{C_i}{ C_s}-1\right).
\end{equation}

As proposed by Hong et al. \cite{Interface}, we include in the energy the effect of interface tension. Despite having been neglected in many models for hydrogel swelling, this term plays a role in the transient poroelastic relaxation of the material, when boundaries between solvent-rich and solvent-poor regions can emerge \cite{sarah,Interface}. Again we assume that the contribution of mobile ions is negligible, so that only the solid-solvent interface contributes to the energy:
\begin{equation}
\psi_5 = \frac{\gamma}{2} J \left|\nabla C_s\right|^2,
\end{equation}
where the constant $\gamma$ plays a role analogous to a surface tension.

Finally, we need to specify the strain energy $\psi_6$ which depends on the particular constitutive model used to describe the material. As mentioned in the Introduction the Standard Linear Solid (SLS), see Figure \ref{SLS}, is commonly used to describe soft material in the regime of small deformation. However, when account for large deformation, as in the case of swelling, soft material present a non-linear behaviour. For this reason we consider the model in Figure \ref{fig1A}, which is a generalization of the 1D SLS to 3D problems with non-linear elastic response. 

\begin{figure}
	\begin{subfigure}{0.32\textwidth}
		\centering
		\large
		\def\svgwidth{0.9\linewidth}
		\input{latex/images/modelA1.pdf_tex}
		\caption{}
		\label{fig1A}
	\end{subfigure}
	\hspace{20mm}
	\begin{subtable}{0.375\textwidth}
		\hspace{-15mm}
		\begin{tabular}{|c | c | c|}	
			\hline
			\multirow{2}{*}{\textbf{ Element } }& \textbf{ Constitutive } & \multirow{2}{*}{\textbf{ Deformation }} \\
			& \textbf{Properties} &\\
			\hline	
			\multirow{2}{*}{ spring 1 } & Isotropic  & volumetric\\
			&Neo-Hookean spring& + deviatoric\\
			\hline
			\multirow{2}{*}{ spring 2 } & Isotropic  & volumetric\\
			&Neo-Hookean spring &+ deviatoric\\ 
			\hline
			\multirow{2}{*}{dashpot}  & Isotropic  & 	\multirow{2}{*}{deviatoric}\\
			& Linear dashpot & \\
			\hline
		\end{tabular}
		\caption{}
	\end{subtable}
	\caption{(a) Schematic representation of the non-linear rheological model for ECM; (b) Table summarizing the major properties of the model components.}
\end{figure}

The strain energy can thus be decomposed into the sum of the contributions from spring $1$ and spring $2$:

\begin{equation}
\psi_6 = \psi_{6.1}(\F) + \psi_{6.2}(\F_e).
\end{equation}

As in \cite{ecm2}, we consider the spring to be isotropic and hyper-elastic (Neo-Hookean) which are characterised by the following form of the free-energy:

\begin{eqnarray}
\psi_{6.1}(\F) = \frac{G^A_1}{2} \left(\F:\F - 3 -2 \ln J\right)\\
\psi_{6.2}(\F_e) = \frac{G^A_2}{2} \left(\F_e:\F_e - 3 -2 \ln J_{e}\right)\label{hyp}
\end{eqnarray}
where $G_{1/2}$ stands for the shear modulus associated with each spring, $J_e= \det \F_e$, while $J$ is as defined in the previous sections. As derived in \cite{floryprinciples}, the hyper-elastic model~(\ref{hyp}) can be correlated to the microscopic properties of a polymer network, under the assumption of Gaussian chains and affine deformation. Other thermodynamically consistent form of the stretching energy have been proposed in the literature \cite{BERGSTROM1998931,boyce2,doi}. These have been also derived by statistical arguments but starting from different network models.

\subsection{Entropy Production $\sigma$.}
\label{ent}

Having specified how the system interacts with its environment, we can now discuss how it dissipates energy. As mentioned in Section~(\ref{kin}), there are two contributions: transport (diffusion of solvent and solutes) and viscosity. The thermodynamic fluxes \footnote{See Section \ref{secNET}} associated with these two phenomena are $\mathbf{J}_m$, $m=s,1,\ldots,N$, and $\LL_v$. Consequently, using Equation~(\ref{2law}), we obtain:

\begin{equation}
\sigma = \sum_m \zeta_m \cdot \mathbf{J}_m + \zeta_v : \LL_v,
\label{dis}
\end{equation}
where $\zeta$s represent the thermodynamic forces associate with each flux. On the other hand, $\nabla_0 \dot{C}_s$, $\dot{C}_s$, $\dot{C}_i$, $\mathbf{\dot{H}}$ and $\dot{\F}$ describe the evolution of reversible process. This implies that their value can be controlled and arbitrarily chosen, by carefully tune the condition of an experiment, while the energy imbalance inequality~(\ref{ineq}) continue to hold. Given the constraint~(\ref{temp1}) on $\psi$, this can only happen if the terms highlighted in blue in Equation~(\ref{ineq}) are identically zero. As shown in the Appendix [TO DO], this leads to the following system of equations:
\begin{gather}
\boldsymbol{\xi} = \gamma J \,\mathbb{B}^{-1} \,\nabla_0 \,C_s,\label{sys1}\\[2mm]
\begin{aligned}
\mu_s = p v_s + \mu_s^0 - \gamma J \nabla^2 C_s + k_BT&\left[\ln \frac{C_s v_s}{1+C_s v_s} + \frac{1}{1+C_sv_s}\right.\\
&\left.\ \ \ \ \ \ +\frac{\chi}{(1+C_s v_s)^2}-\sum_i \frac{C_i}{C_s}\right], 
\end{aligned}\label{gov1}\\[2.5mm]
\mu_i = \mu^0_i + e\Phi z_i + kT \ln \frac{C_i}{C_s},\label{mu}\\
\mathbf{E} = \frac{1}{\epsilon J} \F^T \F\, \mathbf{H}\, , \qquad -\epsilon J \nabla^2 \Phi = Q\, ,\label{sys2}
\end{gather}
\begin{gather}
\begin{aligned}
\mathbb{T}= -p \mathbb{I} + \underbrace{\gamma \left[\frac{1}{2} |\nabla C_s|^2\mathbb{I} - \nabla C_s \otimes \nabla C_s\right]}_{\mathbb{T}^{kort}}+ \underbrace{\epsilon \left[\frac{1}{2} \,|\nabla \Phi|^2\mathbb{I} -\nabla \Phi \otimes \nabla \Phi\right]}_{\mathbb{T}^{Max}}\\
+ \frac{G^A_1}{1+C_sv_s}\left(\mathbb{B}-\mathbb{I}\right) + \frac{G^A_2}{1+C_sv_s}\left(\mathbb{B}_e-\mathbb{I}\right),
\end{aligned}
\label{sys3}
\end{gather}

As discussed in Section \ref{secNET}, in the framework of linear non-equilibrium thermodynamics, when considering isothermal transformation, the second law of thermodynamics can be rewrite as Equations~(\ref{eqCIT}). Using the same argument as in Section \ref{sec_ine} and Equation~(\ref{dis}), we can rewrite Equations~(\ref{eqCIT}) in differential form, and substituting Equations~(\ref{sys1})-(\ref{sys3}), we obtain:
\begin{equation}
\begin{aligned}
\sigma = -  \sum_m \frac{1}{T}\nabla_0 \,\mu_m \cdot \mathbf{J}_m + \frac{1}{T}\left( \F_e^T\frac{\partial \psi}{\partial \F_e}-p_v\mathbb{I}\right):\mathbb{L}_v\label{EQen}
\end{aligned} 
\end{equation}

Equating Equation~(\ref{EQen}) and~(\ref{dis}), it is evident that the thermodynamics forces are:
\begin{gather}
\zeta_m = \frac{1}{T} \nabla_0 \,\mu_m, \label{vflow1}\\
\zeta_v = \frac{1}{T} \left( \F_e^T\frac{\partial \psi}{\partial \F_e}-p_v\mathbb{I}\right) = \frac{1}{T} \left[G^A_2(\mathbb{C}_e-\mathbb{I})-p_v\mathbb{I}\right].
\end{gather}
Assuming to be in regime of linear non-equilibrium thermodynamics, we can use the identity~(\ref{lin}) to couple fluxes and forces. However, considering the symmetry constraint from \textit{Curie's law}\footnote{Macroscopic causes can not have more element of symmetries than the effect they cause \cite{CIT}} , there can be no coupling between fluxes and forces of different tensorial nature. Consequently, we are left with the following force-flux relation:
\begin{gather}
\LL_v = L_{vv} \zeta_v,\label{vflow2}\\
\mathbf{J}_m = \sum_{k=s,1,\ldots,N} L_{mk} \zeta_k. \label{dif}
\end{gather}

%Combining Equation~(\ref{vflow1}) and (\ref{vflow2}), and imposing that condition~(\ref{Jv}) is satisfied, we can characterise the viscous flow by the following relation:
%\begin{equation}
%\LL_v = L_{vv}T^{-1}\text{DEV}\left[\F_e^T\frac{\partial \psi}{\partial \F_e}\right] = \eta^{-1}\text{DEV}\left[\F_e^T\frac{\partial \psi}{\partial \F_e}\right] ,
%\end{equation}
%where $\eta$ represent the viscosity of the material and $\text{DEV}\left[\cdot\right] = \cdot-1/3\, \text{tr}(\cdot)$ is the deviatoric component of the tensor in the brackets. 

%The dissipative contribution due to the relative movement of phases has been largely studied in the literature \cite{ecm1,ecm2}. Starting from Equation~(\ref{dif}) and standard arguments we can get to the following definition for the fluxes:
As described in Appendix \ref{apenergy}, starting from Equations~(\ref{vflow1})-(\ref{dif}) and with common consideration from the theory of mixture, we can derive the following system of time dependent equations:
\begin{eqnarray}
\partial_t C_s=\nabla_0 \cdot\left[K C_s \F^{-1}\left(c_s\nabla \mu_s +\sum_i \frac{D_i}{D^0_i} c_i \nabla \mu_i\right)\right],\label{gov2}\\
\partial_t C_i= \nabla_0\cdot\left[\frac{D_i}{k_B T}C_i\F^{-1}\nabla \mu_i -\frac{D_i}{D^0_i} \frac{C_i}{C_s} \mathbf{J}_s\right],\label{gov3} \\
\dot{\mathbb{B}}_e =\LL\mathbb{B}_e + \mathbb{B}_e \LL^T - \frac{1}{\tau_R} \,\mathbb{B}_e\text{DEV}[\mathbb{B}_e].\label{Be}
\end{eqnarray}
where the parameters are macroscopic phenomenological coefficients, that can either be estimated experimentally or derived from the nano/microscopic properties of the different phases \cite{ecm1,ecm2}. To sum up the governing equations for the Model A are Equations~(\ref{gov1})-(\ref{sys3}) together with the flow rules~(\ref{gov2})-(\ref{Be}). The analogous system of equation for model B can be found in Appendix \ref{modelB}. 
\subsection{Evolution Equation.}

In order to get a better physical insight into the behaviour, we first rewrite the solvent chemical potential as:
\begin{gather}
\mu_s = \mu^0_s + k_B T \left(\frac{p v_s}{k_BT} +\Pi_{osm}-\sum_i \frac{C_i}{C_s} -\frac{\gamma J}{k_B T}\Pi_{grad}\right)\label{mu2},\\
\Pi_{osm}=\ln \frac{C_s v_s}{1+C_s v_s} + \frac{1}{1+C_sv_s}+\frac{\chi}{(1+C_s v_s)^2},\\
\Pi_{grad} = \nabla^2 C_s,
\end{gather}
where $p$ represents the pore pressure, $\Pi_{osm}$ is the osmotic pressure of the solution and $\Pi_{grad}$ is the pressure due to interface energy. If we now substitute into Equations~(\ref{gov2}) the chemical potentials~(\ref{mu2})-(\ref{mu}), which yields to:
\begin{equation}
\begin{aligned}
\partial_t C_s=\nabla_0 \cdot\left\{K\F^{-1}\left[C_s v_s\nabla p - \color{red}\gamma C_s J\nabla \Pi_{grad}\color{black}+\sum_i \frac{D_i}{D^0_i} C_i e z_i \nabla \Phi\right.\right.\\
\left.\left.+ k_B T\color{teal} \left(C_s \nabla \Pi_{osm}+ \sum_i\left(1-\frac{D_iC_i}{D^0_iC_s}\right) \nabla C_s - \sum_i\left(1-\frac{D_i}{D^0_i}\right) \nabla C_i \right)\color{black}\right]\right\}.\label{long}
\end{aligned}
\end{equation}

The above equation shows that the solvent transport is driven by pressure gradient, osmotic pressure gradient (green term in Equation~(\ref{long})), electric potential gradient and the additional composition gradient (red term in Equation~(\ref{long})), which is here first introduced in the context of polyelectrolytes. In the absence of solutes ($C_i\equiv 0$), we recover the same model presented by Hennessy et. al~\cite{sarah}. If we further assume that $|\nabla \Pi_{grad}|<<1$ and take the limit $v_sC_S\rightarrow\infty$, Equations~(\ref{long}) reduces to Darcy's law for the flow in a porous media. The model for polyelectrolytes proposed by Hong \cite{Reviewpolyel} correspond instead to the limit $D^0_i\rightarrow\infty$, i.e. mobile species can move freely in pure solution, and  $|\nabla \Pi_{grad}|<<1$.

Similarly we can rewrite Equation~(\ref{gov3}) as:
\begin{equation}
\scriptsize
\partial_t C_i = \nabla_0 \left[D_i\F^{-1}\left(\underbrace{\nabla C_i}_{\text{diffusion}} +\underbrace{\frac{eC_iz_i}{k_B T} \nabla \Phi}_{\text{electric}}\right)-\underbrace{\frac{D_i C_i}{C_s}\F^{-1}\nabla C_s}_{\text{osmotic pressure}}-\underbrace{\frac{D_i C_i}{D^0_iC_s}\mathbf{J}_s}_{\text{advenction}}\right]\label{long2}
\end{equation}

In the case of ions, the driving forces of transport are the diffusion of ions, the electric field, the osmotic pressure due to the mixing with the solvent and the advection term (due to the relative movement of ions with respect to the solvent). In the limit $D^0_i\rightarrow0$, where the ions can freely move in the pure solution, we recover the formulation of Hong in \cite{Reviewpolyel}, which is commonly used in the description of polyelectrolytes. In the dilute limit, i.e. when the concentration of ions is much smaller than the solvent  concentration $C_i<<C_s$, we can drop both the osmotic and advection term so to recover the well-known Nernst-Planck equation \cite[see Equation (6.67)]{Reviewpolyel}.

As mentioned in the introduction, one the major aspect of interest is the visco-elastic contribution to the evolution of the system. Even though in the transport Equation~(\ref{long})-(\ref{long2}) there is no direct reference to it, the transport is indirectly coupled to the visco-relaxation through the pressure gradient. Taking the divergence of Equation~(\ref{sys3}) and using the conservation of momentum~(\ref{consmom}), we can identify components to the pressure gradient:

\begin{equation}
\nabla p= \nabla \cdot \mathbb{T}^{kort} + \nabla \cdot \mathbb{T}^{Max} + G_1^A\underbrace{ \nabla \cdot \left[\frac{\B-\mathbb{I}}{1+v_sC_s}\right]}_{\substack{\text{swelling}\\\text{+ deviatoric}\\\text{ deformation}}} + \,\color{red} G_2^A \underbrace{\nabla \cdot\left[\frac{\B_e-\mathbb{I}}{1+v_sC_s}\right]}_{\color{black}\substack{\text{swelling}\\\text{+ deviatoric}\\\text{deformation}\\\text{+ viscous relaxation}}}\color{black}
\end{equation}

Differently from the standard Neo-Hookean model for polyelectrolytes, we have an additional term highlighted in red, which accounts for the energy stored in the second spring of model A. Its time evolution, as determined by Equation~(\ref{Be}) is driven by the swelling and macroscopic deformation and a relaxation term which capture the viscous nature of the solid phase:
\begin{equation}
\dot{\mathbb{B}}_e=\underbrace{\LL\mathbb{B}_e + \mathbb{B}_e \LL^T}_{\substack{\text{swelling}\\\text{+ deviatoric}\\\text{deformation}}} - \underbrace{\frac{1}{\tau_R} \,\mathbb{B}_e\text{DEV}[\mathbb{B}_e]}_{\substack{\text{viscous}\\\text{relaxation}}}\label{Be2}
\end{equation}


Despite the introduction of the non-linear terms, there is an apparent analogy between the above Equation and the ODE describing the standard linear solid, see Equation~(SLS) in the Introduction. The term $\dot{\epsilon} \leftrightarrow \LL$ is related to the strain experienced by spring 1, while $\epsilon_e  \leftrightarrow \mathbb{B}_e$ is the variable describing the strain on the 2. In the limit $\tau_R\rightarrow \infty$, we have that $\mathbb{B}_e\equiv\mathbb{B}$. We thus recover the standard Neo-Hookean hyper-elastic model for hydrogels with shear modulus given by $G^A_1+G^A_2$. Similar result are obtained for model B. The governing equation of Model B is equivalent to~(\ref{Be2}), except for the fact that it is not influence by purely volumetric deformation:
\begin{equation}
\dot{\bar{\mathbb{B}}}_e=\underbrace{\bar{\LL}\bar{\mathbb{B}}_e + \bar{\mathbb{B}}_e \bar{\LL}^T}_{\substack{\text{deviatoric}\\\text{deformation}}} - \underbrace{\frac{1}{\tau_R} \,\bar{\mathbb{B}}_e\text{DEV}[\bar{\mathbb{B}}_e]}_{\substack{\text{viscous}\\\text{relaxation}}}
\end{equation}
while the pressure gradient is defined as:
\begin{equation}
\begin{aligned}
\nabla p= \nabla \cdot \mathbb{T}^{kort} + \nabla \cdot \mathbb{T}^{Max} + G_1^B\underbrace{ \nabla \cdot \left[\frac{\text{DEV}[\bar{\mathbb{B}}]}{1+v_sC_s}\right]}_{\substack{\text{+ deviatoric}\\\text{ deformation}}} \\
+ \,\color{red} G_2^B \underbrace{\nabla \cdot\left[\frac{\text{DEV}[\bar{\mathbb{B}}_e]}{1+v_sC_s}\right]}_{\color{black}\substack{\text{ deviatoric}\\\text{deformation}\\\text{+ viscous relaxation}}}\color{black}+ G_{vol}\underbrace{\nabla \left(\frac{J^{2/3}-1}{J}\right)}_{swelling}.
\end{aligned}
\end{equation}

As we will discuss in the next section, when considering instead a purely volumetric deformation, such as for free swelling, i.e. $\F=\F_{vol}=J^{1/3}\mathbb{I}$, we have that the two model are equivalent given that $G_{vol}\equiv G_A +G_B$. Before comparing the model in different experimental setting, we have briefly summarised in Table~(\ref{summary}) the variables and governing equation for both models. In order to have a close solution, the proper boundary and initial conditions needs also to be assigned depending on the specific problem considered.

\vspace{3mm}
\begin{table}
	\centering
	\begin{tabular}{c|c|c}
		\hline\addlinespace[2pt]
		Variable &  \hspace{1pt} Model A \hspace{1pt} & \hspace{1pt} Model B\hspace{1pt} \\
		\hline
		\hline\addlinespace[2.5pt]
		Solvent Concentration &  \multicolumn{2}{c}{$C_s$~(\ref{long})}\\[2.5pt]
		Chemical Potential of & \multicolumn{2}{c}{ $\mu_s$~(\ref{gov1})}\\
		the solvent &\multicolumn{2}{c}{}\\[2pt]
		Ionic Concentrations &\multicolumn{2}{c}{$C_i$~(\ref{long2})}\\[3pt]
		Chemical Potential of & \multicolumn{2}{c}{$\mu_i$~(\ref{mu}) }\\
		the ions &\multicolumn{2}{c}{ }\\[2pt]
		\hline\addlinespace[2pt]
		Local Volumetric Change & \multicolumn{2}{c}{$J$~(\ref{inc})}\\[2.5pt]
		Cauchy Stress Tensor & 	\multicolumn{2}{c}{$\mathbb{T}$~(\ref{consmom})}\\[2.5pt]
		Deformation Gradient & $\F$~(\ref{dec1})& $\F$~(\ref{dec2})\\[2.5pt]
		Pore Pressure & $p$~(\ref{sys3}) & $p$~\ref{sys2B})\\[2.5pt]
		Left-Cauchy Tensor& \multirow{2}{*}{$\mathbb{B}_e$~(\ref{Be})} & \multirow{2}{*}{$\mathbb{\bar{B}}_e$~(\ref{BeB})}\\
		for spring 2&&\\
		Viscous Velocity & \multirow{2}{*}{$\LL_v$~(\ref{Lv1})}&  \multirow{2}{*}{$\bar{\LL}_v$~(\ref{Lv2})}\\
		Gradient Tensor&&\\
		\hline\addlinespace[2pt]
		Electric Field & \multicolumn{2}{c}{$\Phi$~(\ref{sys2})}\\
		\hline 
		\hline
	\end{tabular}
	\vspace{3mm}
	\caption{List of Variables involved in the problem, with reference to the corresponding governing equations.}
	\label{summary}
\end{table}

\section{Elasticity: Equilibrium Behaviour.}
In this section, we will be focusing on the equilibrium behaviour of the two model under free and compressed swelling. Given the large number of parameter in the models, such experiment allows to estimate a subset of them. While previous model proposed in the literature have been arbitrarily chosen how to decompose the deformation gradient $\F$, we here shows that such choice need to consider equal to a constitutive equation and thus tested experimentally. 
%
\subsection{Evolution Equation.}

In order to get a better physical insight into the behaviour, we first rewrite the solvent chemical potential as:
\begin{gather}
\mu_s = \mu^0_s + k_B T \left(\frac{p v_s}{k_BT} +\Pi_{osm}-\sum_i \frac{C_i}{C_s} -\frac{\gamma J}{k_B T}\Pi_{grad}\right)\label{mu2},\\
\Pi_{osm}=\ln \frac{C_s v_s}{1+C_s v_s} + \frac{1}{1+C_sv_s}+\frac{\chi}{(1+C_s v_s)^2},\\
\Pi_{grad} = \nabla^2 C_s,
\end{gather}
where $p$ represents the pore pressure, $\Pi_{osm}$ is the osmotic pressure of the solution and $\Pi_{grad}$ is the pressure due to interface energy. If we now substitute into Equations~(\ref{gov2}) the chemical potentials~(\ref{mu2})-(\ref{mu}), which yields to:
\begin{equation}
\begin{aligned}
\partial_t C_s=\nabla_0 \cdot\left\{K\F^{-1}\left[C_s v_s\nabla p - \color{red}\gamma C_s \nabla J\Pi_{grad}\color{black}+\sum_i \frac{D_i}{D^0_i} C_i e z_i \nabla \Phi\right.\right.\\
\left.\left.+ k_B T\color{teal} \left(C_s \nabla \Pi_{osm}+ \sum_i\left(1-\frac{D_iC_i}{D^0_iC_s}\right) \nabla C_s - \sum_i\left(1-\frac{D_i}{D^0_i}\right) \nabla C_i \right)\color{black}\right]\right\}.\label{long}
\end{aligned}
\end{equation}

The above equation shows that the solvent transport is driven by pressure gradient, osmotic pressure gradient (light-blue term in Equation~(\ref{long})), electric potential gradient and the additional composition gradient (red term in Equation~(\ref{long})), which is here first introduced in the context of polyelectrolytes. In the absence of solutes ($C_i\equiv 0$), we recover the same model presented by Hennessy et. al~\cite{sarah}. 
Similarly we can rewrite Equation~(\ref{gov3}) as:
\begin{equation}
\scriptsize
\partial_t C_i = \nabla_0 \cdot \left[D_i\F^{-1}\left(\underbrace{\nabla C_i}_{\text{diffusion}} +\underbrace{\frac{eC_iz_i}{k_B T} \nabla \Phi}_{\text{electric}}\right)-\underbrace{\frac{D_i C_i}{C_s}\F^{-1}\nabla C_s}_{\text{osmotic pressure}}-\underbrace{\frac{D_i C_i}{D^0_iC_s}\mathbf{J}_s}_{\text{advenction}}\right]\label{long2}
\end{equation}

\subsection{Possible simplification of the model.}

If we consider  the case of small ions, when we can neglect the friction between the ions and the polymer chains, i.e. $D_{i}=D^0_i$. Under this assumption the system of equation simplifies as $K=c_sk$. Following the work of Hennesy et al., we consider the hydraulic permeability $k$ to be:

\begin{equation}
k = \frac{D_0(1+v_sC_s)^\beta}{k_B T c_s} \Rightarrow K=  \frac{D_0(1+v_sC_s)^\beta}{k_B T}.
\end{equation}

The governing equation thus reduces to:
\begin{eqnarray}
\begin{aligned}
\partial_t C_s=\nabla_0 \cdot\left\{\frac{D_0(1+v_sC_s)^\beta}{k_B T}\F^{-1}\left[C_s v_s\nabla p - \gamma C_s \nabla  \left(J\nabla^2 C\right)\right.\right.\\
\left.\left.+\sum_i C_i e z_i \nabla \Phi+ k_B T \left(C_s \nabla \Pi_{osm}+ \sum_i\left(1-\frac{C_i}{C_s}\right) \nabla C_s \right)\right]\right\}.
\end{aligned}\label{A}
\end{eqnarray}
\begin{eqnarray}
\partial_t C_i = \nabla_0 \cdot\left[D_i\F^{-1}\left(\nabla C_i +\frac{eC_iz_i}{k_B T} \nabla \Phi\right)-\frac{D_i C_i}{C_s}\F^{-1}\nabla C_s-\frac{C_i}{C_s}\mathbf{J}_s\right]\label{B}
\end{eqnarray}

\section{1D full model}

Considering a 1D constrained swelling, the deformation gradient tensor is of the form:
\begin{equation}
\F= \begin{bmatrix}
1 & 0 &0\\
0 & 1 &0\\
0 & 0 &J(t,Z)\\
\end{bmatrix}.\label{deffree}                                                                
\end{equation}
where in the dilute case we have $J=1+v_sC_s$.
Similarly all the variables will just depend on the time $t$ and the $Z$ coordinates, the tensor will preserve the same diagonal form, identify by the index $1,2,3$. Given that the set of state and governing equation for the model are given by:

\begin{gather}
\partial_Z S_3 = 0,\\
S_3 = -p - \frac{\gamma}{2} \frac{(\partial_Z C)^2}{J^2} - \frac{\epsilon}{2} \frac{(\partial_Z \Phi)^2}{J^2}+ \frac{G}{J}\left(J^2-1\right),\\[2mm]
-\epsilon \partial_Z \left(J^{-1} \partial_Z \Phi\right)= Q,\\[2mm]
Q= e\left(\sum\limits_{i} z_i C_i + z_fC_f\right)\, ,\\[2mm]
\begin{aligned}
J_s=-\frac{D_0J^{\beta-2}}{k_B T}\left[ - \gamma C_s \partial^2_Z  \left(\frac{\partial_Z C_s}{J}\right)+\sum_i C_i e z_i \partial_Z \Phi+ C_s v_s\partial_Z p\right.\\[2mm]
\left.+ k_B T \left(C_s \partial_Z \Pi_{osm} + \sum_i \left(1-\frac{C_i}{C_s}\right) \partial_Z C_s \right)\right]\, ,
\end{aligned}\\[2mm]
\Pi_{osm}=\ln \frac{C_s v_s}{J} + \frac{1}{J}+\frac{\chi}{J^2},\\[2mm]
\partial_t C_s = -\partial_Z J_s\,,\\[2mm]
\partial_t C_i = \partial_Z \left[\frac{D_i}{J^2}\left(\partial_Z C_i +\frac{eC_iz_i}{k_B T} \partial_Z \Phi\right)-\frac{D_i C_i}{J^2C_s}\partial_Z C_s-\frac{C_i}{C_s}J_s\right]\, .
\end{gather}
 
In the limit of $v_sC_s$
 
\newpage
\bibliographystyle{plain}
\bibliography{ref}
\end{document}